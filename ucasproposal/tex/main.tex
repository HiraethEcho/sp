
%---------------------------------------------------------------------------%
%->> Main content
%---------------------------------------------------------------------------%
\section{选题的背景及意义}

\subsection{选题背景}
双有理代数几何的目标之一是按双有理等价类分类代数簇,并选取选择恰当的代表元。
极小模型纲领 (Minimal model program)是构造代表元的一种方法,并且猜想每一个代数簇都双有理等价于一个极小模型 (minimal model) 或一个森纤维空间 (Mori fibre space),但这样的代表元有时并不唯一,于是自然的问题就是不同代表元之间的关系。

近十年来,具有klt奇点的高维代数簇对的极小模型纲领取得了一系列巨大的进展,尤其是对大边界除子的标量极小模型纲领终结。这些结果可以用来研究代数簇的性质、构造代数簇对的各种模型,而不只是极小模型。特别的,这些方法可以用来分析同一个双有理等价类的不同代表元之间的关系,比如平转连接极小模型,和Sarkisov分解森纤维空间的双有理映射。

历史上,Sarkisov为了分类直纹曲面而首先提出来关于曲面的森纤维空间的分解,并且指出对三维空间的相应结果。Corti给出了对终端型奇点三维代数簇的证明,后来又被Bruno和Matsuki推广到klt奇点的情况,并且探讨了高维空间的推广。

\subsection{选题意义}
通过将不同代表元之间的双有理映射分解成一些基础的双有理映射,可以反应代表元之间的关系。另外,基础的双有理映射可以保持代数簇一些性质,使得某一个代表元的性质可以继承到令一个代表元上。另一方面,基础的映射更容易分析其性质,这些映射的复合可以得到整个映射的性质。例如,仿射空间的自同构可以延拓成射影空间的双有理映射,而双有理映射正是森纤维空间。通过Sarkisov分解得到双有理映射的性质,进一步可以得到仿射空间自同构群的性质。

\section{国内外本学科领域的发展现状与趋势}
目前对任意维数的具有klt奇点的$\mathbb{Q}$-分解代数簇对的Sarkisov分解成立,并且有两种不同的方法。对这个问题有多种不同的推广,例如考虑一般化代数簇对 (generalized pair),低维数下正特征的情况、低维数下更差奇点性质 (例如lc奇点)的情况等。但这些方法还没有比较,而且除了曲面上最原始的方法外,缺少例子。

\section{课题主要研究内容、预期目标}
主要学习极小模型纲领和Sarkisov纲领,理解不同方法。
预期将这些证明中的细节补全,并且互相比较,计算例子,梳理成文。尝试在叶层化的情况下讨论Sarkisov纲领。

\section{拟采用的研究方法、技术路线、实验方案及其可行性分析}
主要采用文献学习法,广泛阅读相关专题文献。三种方法都有相关文献,并且有各种推广,在学习之后通过具体例子的构造和计算来加深理解。
叶层化代数簇的极小模型纲领也已经有一些结果,可以尝试推广。
代数簇对的相关结果是已知的,研究重点是比较方法,并通过尝试推广加深理解,有较强可行性。

\section{已有科研基础与所需的科研条件}
已具有基本的双有理代数几何的知识,了解极小模型纲领的过程。仍需要进一步学习关于极小模型的结论,包括极小模型纲领的终结性,和各种模型的有限性。另外还需要学习叶层化代数簇的基本知识。
\section{研究工作计划与进度安排}

首先学习近十年来双有理代数几何的重要发展,即模型有限性和极小模型纲领终结性等。中期学习具体的Sarkiosv纲领的证明,将细节补充完整,并且做具体计算,比较各种方法。最后尝试在叶层化代数簇对上推广结果。


\nocite{*}% 使文献列表显示所有参考文献(包括未引用文献)
%---------------------------------------------------------------------------%

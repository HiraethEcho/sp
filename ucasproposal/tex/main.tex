
%---------------------------------------------------------------------------%
%->> Main content
%---------------------------------------------------------------------------%
\section{选题的背景及意义}

\subsection{选题背景}
双有理代数几何的目标之一是按双有理等价类分类代数簇,并选取选择恰当的代表元。
极小模型纲领 (Minimal model program)是构造代表元的一种方法,并且猜想每一个代数簇都双有理等价于一个极小模型 (minimal model) 或一个森纤维空间 (Mori fibre space),但这样的代表元有时并不唯一,于是自然的问题就是不同代表元之间的关系。
近年来,具有klt奇点的高维代数簇对的极小模型纲领取得了一系列巨大的进展,尤其是

\subsection{选题意义}
通过将不同代表元之间的双有理映射分解成一些基础的双有理映射,可以反应代表元之间的关系。

\section{国内外本学科领域的发展现状与趋势}

趋势趋势趋势趋势趋势趋势趋势趋势趋势趋势趋势趋势趋势趋势趋势趋势趋势趋势趋势趋势趋势趋势趋势趋势趋势趋势趋势趋势趋势趋势

\section{课题主要研究内容、预期目标}

目标目标目标目标目标目标目标目标目标目标目标目标目标目标目标目标目标目标目标目标目标目标目标目标目标目标目标目标目标目标目标目标目标目标目标目标目标目标目标目标

\section{拟采用的研究方法、技术路线、实验方案及其可行性分析}

方法方法方法方法方法方法方法方法方法方法方法方法方法方法方法方法方法方法方法方法方法方法方法方法方法方法方法方法方法方法方法方法方法方法方法方法方法方法方法方法

\section{已有科研基础与所需的科研条件}

条件条件条件条件条件条件条件条件条件条件条件条件条件条件条件条件条件条件条件条件条件条件条件条件条件条件条件条件条件条件条件条件条件条件条件条件条件条件条件条件条件条件条件条件条件条件条件条件条件条件

\section{研究工作计划与进度安排}

计划计划计划计划计划计划计划计划计划计划计划计划计划计划计划计划计划计划计划计划计划计划计划计划计划计划计划计划计划计划计划计划计划计划计划计划计划计划计划计划计划计划计划计划计划计划计划计划计划计划


\nocite{*}% 使文献列表显示所有参考文献(包括未引用文献)
%---------------------------------------------------------------------------%


%---------------------------------------------------------------------------%
%->> Main content
%---------------------------------------------------------------------------%
\section{学位论文进展情况,存在的问题,已取得的阶段性成果}

\subsection{论文进展情况}
学位论文进展顺利,已如期完成研究目标。具体实现的研究进展和阶段性成果如下:
\begin{enumerate}
  \item 学习了标量极小模型终结性定理,和极小模型有限性定理;
  \item 学习了Corti的原始证明,和对klt奇点情况的推广;
  \item 学习了Hacon提出的用有限模型方法;
  \item 比较了三种方法的异同。
\end{enumerate}
\subsection{存在的问题}
还有下列问题没有解决:
\begin{enumerate}
  \item 首先是原始方法在高维的推广。这种方法的终结性分为两个部分;
    其一是依赖于翻转 (flip)的终结,目前只在低维数成立,或者运行关于某丰沛除子标量的MMP。
    但是归纳地构造Sarkisov连接时,每一个连接都需要运行一次MMP,即使每一个MMP中的翻转会终结,不同的MMP的翻转连在一起,就不再是同一标量MMP,所以终结性不能保证。
    其二是在第一和第二型连接复合的终结性,此时需要局部算术典范阈值的有限性。对于没有边界的情况,在低维情况下是成立的。但是考虑有边界除子的代数簇对时,需要的不再是简单的算术典范阈值,而是更复杂的奇点条件。一个可能的方法是建立局部算术典范阈值的半连续性,然后通过拓扑空间的诺特性质来得到相应的结果,但这需要进一步学习相关知识。
  \item 其次是例子的计算。前两种方法可以用相同的例子分别计算,相关计算量不算太大。但是第三种方法需要计算大量的公共解消的丰沛模型,计算量巨大,并且细节扰动太多,使得结果十分复杂,难以体现这种方法的中心思想。因此需要计算其他例子。
  \item 叶层化代数簇的MMP还有一些结果没有被推广,因此不能简单的按照普通的三种方法直接推广。可以尝试将叶层化代数簇对约化成普通的代数簇对,得到较弱的分解,但是会失去一些性质。
\end{enumerate}
\section{下一步工作计划的内容}
接下来尝试为有限模型法构造新的例子,使得计算相对简单,结果相对清楚。可以考虑首先构造公共解消,然后计算它的丰沛模型,接着构造森纤维空间和双有理映射。

除此之外,继续学习叶层化MMP的细节,做两种尝试:其一是直接使用原方法,把叶层化MMP缺失的部分补上;其二是将问题约化为普通代数簇对的情况,然后还原出叶层化的结构。

预计答辩时间为5月中旬。
\section{已取得的研究成果列表}

无。
%---------------------------------------------------------------------------%

\documentclass{article}

\usepackage{amsfonts}
\usepackage[all]{xy}
\usepackage{amssymb}
\usepackage{amsmath}
\usepackage{mathrsfs}
\usepackage{amsthm}
\usepackage{enumerate}
\usepackage[hidelinks]{hyperref}
\usepackage{ulem}

\usepackage{geometry}
\geometry{a4paper,left=2cm,right=2cm,top=2cm,bottom=2cm}

\newtheorem{defn}{Definition}[section]
\newtheorem{prop}[defn]{Proposition}
\newtheorem{lem}[defn]{Lemma}
\newtheorem{thm}[defn]{Theorem}
\newtheorem{cor}[defn]{Corollary}
\newtheorem{rmk}[defn]{Remark}
\newtheorem{fact}[defn]{Fact}
\newtheorem{problem}{Problem}
\newtheorem*{ques}{Question}

\setcounter{section}{0}

\title{Examples}
\author{wyz}

\begin{document}
  \maketitle
  \tableofcontents
\section{application}
Consider the example: $ X=\mathbb{P}^2 $ with coordinates $ (x_0:x_1:x_2) $ and $ X'=\mathbb{P}^2 $ with coordinates $ (y_0:y_1:y_2) $. Take a rational map $ \Phi:X\dashrightarrow X' $ defined by 
$$ \Phi:(x_0:x_1:x_2)\dashrightarrow (x_0^2:x_0x_1:x_1^2+x_0x_2) $$
Take $ \mathcal{H}'=|\mathcal{O}(1)| $, then clearly $ \mathcal{H}=\left<x_0^2,x_0x_1,x_1^2+x_0x_2\right>\subset |\mathcal{O}(2)| $  and $ \mathrm{Bs}\,\mathcal{H}=\{(0:0:1)\} $. Furthermore, consider the affine plane $ x_0\neq 0 $ and $ y_0\neq 0 $, then
$$ \begin{array}{rcl}
  \mathbb{A}^2&\to&\mathbb{A}^2\\
  (x_1,x_2)&\mapsto&(x_1,x_2+x_1^2)
\end{array} $$
is an automorphism of $ \mathbb{A}^2 $. Denote $ B=\{x_0=0\} $.
\subsection{common resolution}
First we may find the common resolution, by blowing up base locus on $ X $ 3 times:

Let $ \pi_1:V_1\to X $ be the blowing up of $ X $ at point $ P=(0:0:1) $. Consider the affine plane $ x_2\neq 0 $, then $ P=(0,0) $ and
$$ \xymatrix{
  \mathrm{Bl}_P\mathbb{A}^2\ar[r]\ar[rd]&\mathbb{A}^2\times \mathbb{P}^1,(x_0,x_1)\times (u,v)\ar[d]\\
  &\mathbb{A}^2,(x_0,x_1)	} $$
Where $ \mathrm{Bl}_P\mathbb{A}^2\subset \mathbb{A}^2\times \mathbb{P}^1 $ is defined by $ x_0v-x_1u=0 $. Conside the affine plane $ v=1 $, then $ u=\frac{x_0}{x_1} $ and we can describe the map as follows:
$$ \begin{array}{rcccccc}
  U_1=\mathbb{A}^2&\to&\mathrm{Bl}_P\mathbb{A}^2&\to&\mathbb{A}^2&\to&\mathbb{P}^2 \\
  (u=\frac{x_0}{x_1},x_1)&\mapsto&(x_1\cdot \frac{x_0}{x_1},x_1)\times (u,1)&\mapsto&(x_1\cdot \frac{x_0}{x_1},x_1)&\mapsto &((\frac{x_0}{x_1})^2x_1:\frac{x_0}{x_1}\cdot x_1:x_1+\frac{x_0}{x_1})
\end{array} $$
The exceptional divisor $ E_1 $ is defined by $ x_1=0 $ on $ U_1 $ with coordinates $ (\frac{x_0}{x_1},x_1) $. The strict transform of $ B $ (also denoted by $ B $) is defined by $ u=\frac{x_0}{x_1}=0 $ on $ U_1 $. We have a rational map $ V_1\dashrightarrow X' $ with base point $ P_1=(0,0)=E_1\cap B\in U_1 $. The exceptional divisor $ E_1 $ and $ B $ both maps to a point $ P'=(0:0:1) $. On $ V_1 $, $ E_1 $ intersects $ B $ transversally. And $ E_1^2=-1 $, $ B^2=0 $.

Let $ \pi_2:V_2\to V_1 $ be the blowing up of  $ V_1 $ at $ P_1=E_1\cap B $ again, and we can focus on the affine open subset $ U_2\cong \mathbb{A}^2 $. We can describe the rational may as follows:
$$ \begin{array}{rccccc}
  U_2=\mathbb{A}^2&\to&U_1&\to&\mathbb{P}^2 \\
  (\frac{x_0}{x_1^2},x_1)&\mapsto&(x_1\cdot \frac{x_0}{x_1^2},x_1)&\mapsto &((\frac{x_0}{x_1^2})^2x_1^2:\frac{x_0}{x_1^2}\cdot x_1:\frac{x_0}{x_1^2}+1)
\end{array} $$
The exceptional divisor $ E_2 $ is defined by $ x_1=0 $ on $ U_2 $ with coordinates $ (\frac{x_0}{x_1^2},x_1) $. We have a rational map $ V_2\dashrightarrow X' $ with base point $ P_2=(-1,0)\in U_2 $. The exceptional divisor $ E_2 $  maps to a point $ P'=(0:0:1) $. On $ V_2 $, $ E_1 $ intersects $ E_2 $; $ E_2 $ intersects $ E_1$ and $B $;  $ B $ intersects $ E_2 $, all transversally. And $ E_1^2=-2 $, $ E_2^2= B^2=-1 $.

Let $ \pi_3:V_3\to V-2 $ be blowing up of  $ V_2 $ at $ P_2\in E_2-E_1-B $ again, and we can focus on the affine open subset $ U_2\cong \mathbb{A}^2 $. We can describe the rational may as follows:
$$ \begin{array}{rccccc}
  U_3=\mathbb{A}^2&\to&U_2&\to&\mathbb{P}^2 \\
  (a,b)=((\frac{x_0}{x_1^2}+1)/x_1,x_1)&\mapsto&(x_1\cdot \frac{x_0}{x_1^2},x_1)&\mapsto &((ab-1)^2b,ab-1,a)=(\frac{x_0^2}{x_1^3}:\frac{x_0}{x_1^2}:\frac{x_0+x_1}{x_1^3})
\end{array} $$
Then $ E_3 $ maps to $ B' $ (thus we may write $ B'=E_3 $), and $ E_1,E_2,B $ maps to the point $ P'=(0:0:1) $. Then $ V=V_3 $ is the common resolution. On $ V_3 $, $ E_1 $ intersects $ E_2 $; $ E_2 $ intersects $ E_1,E_3,B $; $ E_3 $ intersects $ E_2 $; $ B $ intersects $ E_2 $, all transversally. And $ E_1^2=E_2^2=-2 $, $ E_3^2=B^2=-1 $.

Then $ V=V_3 $ is the common resolution, where $ \sigma:V\to X $ is the composition of 3 blowing ups above, and $ \sigma':V\to X' $ is blowing-down curves in the order of $ B,E_2,E_1 $. We have projection formulas:
$$ \begin{array}{rcl}
  K_V&=&\sigma^*K_X+E_1+2E_2+3E_3\\
  &=&\sigma'^*K_{X'}+E_1+2E_2+3B
\end{array} $$
$$ \sigma^*B=B+E_1+2E_2+2E_3 $$
$$ \sigma'^*E_3=E_3+E_1+2E_2+2B $$
\subsection{Set up}
Consider the pair $ (X,bB) $ and $ (X',b'B') $, then 
$$ \begin{array}{rcccccccccc}
  K_V&=&\sigma^*(K_X+bB)&+&(-b)B&+&(1-b)E_1&+&(2-2b)E_2&+&(3-2b)E_3\\
  &=&\sigma'^*(K_{X'}+b'B')&+&(3-2b')B&+&(1-b')E_1&+&(2-2b')E_2&+&(-b')E_3
\end{array} $$
If the pair is related, then we have 
$$ \left\{\begin{array}{lcl}
  3-2b'&\geqslant&-b\\
  3-2b&\geqslant&-b'
  \end{array} \right.$$
This holds for all $ b,b'\in [0,1] $. Take $ e $ such that $ b,b',b-1,b'-1,2b-2,2b'-2\leqslant e\leqslant 1 $(in fact only $ b,b'\leqslant e\leqslant 1 $ ), consider the pair $ (V,B_V=bB+eE_1+eE_2+b'B') $.
In particular, if we take $ b=b'=e=0 $, then this is the ordinary Sarkisov program of smooth surface; take $ b=b'=e=1 $, this is method of Takahashi. We have ramification formulas:
$$ \begin{array}{rcccccccccc}
  K_V+B_V&=&\sigma^*(K_X+bB)&+&0&+&(1-b+e)E_1&+&(2-2b+e)E_2&+&(3-2b+b')E_3\\
  &=&\sigma'^*(K_{X'}+b'B')&+&(3-2b'+b)B&+&(1-b'+e)E_1&+&(2-2b'+e)E_2&+&0
\end{array} $$
Set $ \mathcal{H}'=|\mathcal{O}(1)| $ with $ \mu'= \frac{1}{3-b'}$, then $ \mathcal{H}=|\mathcal{O}(2)| $ and $ \mu=\frac{2}{3-b} $.
$$ \sigma^*H=H_V+E_1+2E_2+3E_3 $$
Then
$$ K_V+B_V+cH_V=\sigma^*(K_X+bB+cH)+(1-b+e-c)E_1+(2-2b+e-2c)E_2+(3-2b+b'-3c)E_3\\ $$
and 
$$ \lambda=\max\{ \frac{1}{1-b+e},\frac{2}{2-2b+e},\frac{3}{3-2b+b'}\} $$
\subsection{Run the program}
\begin{enumerate}[(A)]
  \item If $ 2b+2b'\geqslant 3e>0 $,then
  \begin{enumerate}[(1)]
    \item  $ \lambda=\frac{2}{2-2b+e} $, and $ E_2 $ is the crepant divisor. And $ X_1=T(2,1) $. In fact,
    \begin{itemize}
      \item $ V_2\to Z_0 $ is contraction of curve $ E_1 $ (which is a $ (K+B+\frac{1}{\lambda}H) $-MMP over $ X $);
      $$ E_1.(K_{V_2}+bB+eE_1+eE_2+\frac{1}{\lambda}H_{V_2})=-e<0 $$
      \item $ Z_0\to X $ is a contraction of $ E_2 $ (which is a  $ (K+B) $-MMP over $ X $);
      $$ E_2.(K_{Z_0}+eE_2+bB)=E_2.(K_{V_2}+eE_2+\frac{e}{2}E_1+bB)=-1-\frac{e}{2}+b<0 $$
      \item $ Z_0\to X_1 $ is contraction of curve $ B $ (which is a $ (K+B+\frac{1}{\lambda}H) $-MMP over pt)
      $$ B.(K_{Z_0}+eE_2+bB+\frac{1}{\lambda}H_{Z_0})=B.(K_{V_2}+eE_2+\frac{e}{2}E_1+bB+\frac{1}{\lambda}H_{V_2})=-1+e-b<0 $$
    \end{itemize} 
    And
    $$ \xymatrix{&&&V_3\ar[ld]\ar@{=}[r]&V\ar[ddd]\ar@{=}[r]&V_3'\ar[rd]\\
      &&V_2\ar[ld]\ar[rd]&&&&V_2'\ar[rd]\\
      &V_1\ar[ld]&&Z_0\ar[llld]\ar[rd]&&&&V_1'\ar[rd]\\
      X_0&&&&X_1&&&&X'
      } $$
    For $ \sigma_1:V\to (X_1,eE_2) $, compute the sarkisov degree:
    $$ \begin{array}{rcl}
      K_V&=&\sigma_1^*K_{X_1}+E_3+B \\
      \sigma_1^*E_2&=&E_2+\frac{1}{2}E_1+E_3+B\\
      \sigma_1^*H_1&=&H_V+E_3\\
      K_V&=&\sigma_1^*(K_{X_1}+eE_2)-\frac{e}{2}E_1-eE_2+(1-e)E_3+(1-e)B\\
      K_V+B_V&=&\sigma_1^*(K_{X_1}+eE_2)+\frac{e}{2}E_1+(1+b'-e)E_3+(1+b-e)B\\
    \end{array} $$
    Then $ \lambda_1=\frac{1}{1+b'-e} \leqslant \lambda $ and $ E_3 $ is the crepant divisor.
    \item The second link is of type II: $ E_3 $ is the crepant divisor on $ Z_1 $, and
    \begin{itemize} 
      \item $ V\to V'_2\to Z_1 $ is contraction of curve $ B $ and $ E_1 $ (which is a $ (K+B+\frac{1}{\lambda_1}H) $-MMP over $ X_1 $);
      $$ B.(K_V+B_V+\frac{1}{\lambda_1}H_V)=e-b-1<0 $$
      $$ E_1.(K_{V_2'}+eE_1+eE_2+b'E_3+\frac{1}{\lambda_1}H_{V_2'})=-e<0 $$
      \item $ Z_1\to X_1 $ is a contraction of $ E_3 $ (which is a  $ (K+B) $-MMP over $ X $);
      $$ E_3.(K_{Z_1}+eE_2+b'E_3)=E_3.(K_{V_2'}+eE_2+\frac{e}{2}E_1+b'E_3)=-1-b'<0 $$
      \item $ Z_1\to X_2 $ is contraction of curve $ E_2 $ (which is a $ (K+B+\frac{1}{\lambda_1}H) $-MMP over pt)
      $$ E_3.(K_{Z_1}+eE_2+b'E_3+\frac{1}{\lambda_1}H_{Z_1})=E_3.(K_{V_2'}+eE_2+\frac{e}{2}E_1+b'E_3+(1+b-e)H_{V_2'})=-1+b'-\frac{e}{2}<0 $$
    \end{itemize} 
    $$ \xymatrix{&&&V_3\ar[ld]\ar@{=}[r]&V\ar[ddd]\ar@{=}[r]&V_3'\ar[rd]\\
      &&V_2\ar[ld]\ar[rd]&&&&V_2'\ar[rd]\ar[dl]\\
      &V_1\ar[ld]&&Z_0\ar[llld]\ar[rd]&&Z_1\ar[rd]\ar[ld]&&V_1'\ar[rd]\\
      X_0&&&&X_1&&X_2\ar@{.>}[rr]&&X'
    } $$
  Consider $ V_2'\to Z_1\to X_2 $ and $ V_2'\to V_1'\to X' $, by rigidity lemma, $ X_2\cong X' $.
  \end{enumerate}
  \item If $ 2b+2b'< 3e $, then
  \begin{enumerate}[(1)]
    \item  $ \lambda=\frac{3}{3-2b+b'} $, and $ E_3 $ is the crepant divisor. First run $ (K+B+\frac{1}{\lambda}H) $-MMP on $ V $ over $ X $:
    $$ E_1.(K_V+B_V+\frac{3-2b+b'}{3}H_V)=-e<0 $$
    Let $ q_1:V\to W $ be the contraction of $ E_1 $, then $ q_1^*K_W=K_V,q_1^*E_2=E_2+\frac{1}{2}E_1,q_1^*H_W=H_V $;
    $$ E_2.(K_W+B_W+\frac{3-2b+b'}{3}H_w)=b+b'-\frac{3e}{2}<0 $$
    then $ q_2:W\to Z_0 $ is the contraction of $ E_2 $
    Let $ \tau:V\to Z_0 $, then
    $$ \begin{array}{rcl}
      K_V&=&\tau^*K_{Z_0} \\
      \tau^*E_3&=&E_3+\frac{1}{3}E_1+\frac{2}{3}E_3\\
      \tau^*B&=&B+\frac{1}{3}E_1+\frac{2}{3}E_3\\
      \tau^*H_{Z_0}&=&H_V\\
    \end{array} $$
    Then $ Z_0\to X $ is a contraction of $ E_3 $, which is a $ (K_{Z_0}+B_{Z_0}) $-MMP
    $$ E_3.(K_{Z_0}+B_{Z_0})=-1-\frac{1}{3}b'+\frac{2}{3}b<0 $$
    and $ q_0:Z_0\to X_1 $ is contraction of $ B $, which is a $ (K_{Z_0}+B_{Z_0}+\frac{1}{\lambda}H_{Z_0}) $-MMP
    $$ B.(K_{Z_0}+B_{Z_0}+\frac{1}{\lambda}H_{Z_0})=-1-\frac{1}{3}b+\frac{2}{3}b'<0 $$
    $$ \xymatrix{
      &V\ar[d]^{q_1}&\\
      &W\ar[d]^{q_2}&\\
      &Z_0\ar[dl]_p\ar[dr]^{q_0}\\
      X&&X_1  } $$
      \item $ \sigma_1:V\to X_1 $ and 
      $$ \begin{array}{rcl}
        K_V&=&\sigma_1^*K_{X_1}+E_1+2E_2+3B \\
        \sigma_1^*E_3&=&E_3+E_1+2E_2+2B\\
        \sigma_1^*H_1&=&H_V\\
        K_V+B_V+cH_V&=&\sigma_1^*(K_{X_1}+b'E_3+cH_1)+(1-b'+e)E_1+(2-b'+e)E_2+(3-2b'+b)B\\
      \end{array} $$
    Therefore, $ \lambda_1=0 $, and $ \mu_1=\frac{1}{3-b'} $. By NFI, or by rigidity lemma w.r.t. $ V\to X_1 $ and $ V\to X' $, we have $ X_1\cong X' $.
    
    This is different from Takahashi's process, since $ (Z_0,E_3+B) $ is not dlt, and $ V $ is not a toric variety in that way. However,  $ V_2 $ is the toric variety w.r.t. fan generated by $ (-1,-1),(1,0),(0,1)(1,1),(2,1) $, where $ B(1,1)=E_1 $ and $ B(2,1)=E_2 $. Then affine subset $ U_2 $ corresponds to the cone generated by $ (1,0),(2,1) $. Take a coordinate transform $ \frac{x_1}{x_0^2}\mapsto \frac{x_1}{x_0^2}+1 $, which is not a toric map, but is an automorphism. Blow up the point $ P_2 $, this is a toric map (locally); $ V\to V_2' $ is contraction of $ B $, and $ V_2' $ is a toric variety.
  \end{enumerate}
  \item If $ e=b=b'=0 $:
  \begin{enumerate}[(1)]
    \item First link is of type I, and $ Z_0=X_1=\mathbb{F}_1 $ with exceptional divisor $ E_1=E_{Z_0} $;
    \item Second link is of type II, with $ Z_1=V_2 $ and $ E_{Z_1}=E_2 $, and $ Z_1\to X_2\cong \mathbb{F}_2 $ is contraction of $ B $;
    \item Third link is of type II, where $ V\to Z_2 $ is contraction of $ B $, and $ E_{Z_2}=E_3 $, and $ Z_2\to X_3\cong \mathbb{F}_1 $ is contraction of $ E_2 $;
    \item Last link is of type III,  and $ \mathbb{F}_1=X_3\to X_4=\mathbb{P}^2 $ is contraction of  $ E_1 $.
  \end{enumerate}
We can track it as following:
$$ \xymatrix{
  &&&V\ar[ld]\ar[rd]&&&&\\
&&Z_1=V_2\ar[ld]\ar[rd]&&Z_2=V_2'\ar[ld]\ar[rd]\\
&X_1=\mathbb{F}_1\ar[d]\ar[ld]&&X_2=\mathbb{F}_2\ar[d]&&X_3=\mathbb{F}_1\ar[d]\ar[rd]\\
X=\mathbb{P}^2\ar[d]&\mathbb{P}^1\ar[ld]\ar@{=}[rr]&&\mathbb{P}^1\ar@{=}[rr]&&\mathbb{P}^1\ar[rd]&X_4=\mathbb{P}^2\ar[d]\\
pt&&&&&&pt} $$
And the key divisors are 
$$ \xymatrix{
  &&&(B,E_1,E_2,E_3)\ar[ld]\ar[rd]&&&&\\
  &&(B,E_1,E_2)\ar[ld]\ar[rd]&&(E_1,E_2,E_3)\ar[ld]\ar[rd]\\
  &(B,E_1)\ar[ld]&&(E_1,E_2)&&(E_1,E_3)\ar[rd]\\
  (B)&&&&&&(E_3)} $$
\end{enumerate}

\section{Set the map}
Let \(X=X'= \mathbb{P}^3 \) with coordinates \([x_0: x_1 x_2: x_3]\) and \([x_0: x_1 x_2: x_3]\). Consider the affine space defined by \(x_0 \neq 0\) and \(y_0 \neq 0\), and the map $ \Phi=\Phi_0 $:
\[
(x_1,x_2,x_3)\mapsto (x_1,x_2+x_1^2,x_3)
\]
which is a birational map for \(X=\mathbb{P}^3\):
\[
[x_0,x_1,x_2,x_3]\mapsto [x_0^2,x_0x_1,x_0x_2+x_1^2,x_0x_3]
\]
Let $ B_0=\{x_0=0\} $ be a boundary of $ X $, the base locus is
\[ L_0=\{x_0=x_1=0\}\subset B_0 \]
Note that $ B_0\setminus L_0  $ maps to a point $ Q=[0:0:1:0]\in X' $.

Let $ L $ be a general line in $ X $ s.t. $ L\cap L_1=\emptyset $.

Furthermore, we may consider a plane $ S=S_0=\{x_3=0\}\subset X $.
\section{Common resolution}
We will blow up for 3 times:
\subsection{Blow up $ L_0 $}
\textbf{Compute blowing up}
Let $ \pi_1:V_1\to X  $ be the blowing up on $ X $ along $ L_0 $, we may consider the affine piece $ U_0=\{x_2=1\}\cong \mathbb{A}^3 $:
\[ \xymatrix{
	\mathrm{Bl}_{L_0}\mathbb{A}^3\ar[r]\ar[rd]&\mathbb{A}^3\times \mathbb{P}^1,(x_0,x_1)\times (u,v)\ar[d]\\
	&\mathbb{A}^2,(x_0,x_1,x_3)	} \]
Here $ V_1 $ is defined by $ ux_1-vx_0=0 $. This blow up induces a birational map $ \Phi_1:V_1\to X' $.
\begin{itemize}
	\item On the $ u=1 $ piece, the map is
	\[ [x_0^2:x_0^2v:x_0^2v^2+x_0:x_0x_3]=[x_0:x_0v:x_0v^2+1:x_3] \]
	and is base point free;
	\item On the $ v=1 $ piece, the map is
	\[
	[x_1^2u^2:x_1^2+ux_1:ux_1x_3]=[x_1u^2:x_1u:x_1+u:ux_3]
	\]
	with base locus
	\[ L_1=\{u=x_1=0\}=E_1\cap B_1 \]
	where $ B_1=\{u=0\} $ is strict transform of $ B_0 $ and $ E_1=\{x_1=0\} $ is exceptional diviosr.
	\item In fact, $ \mathbb{P}^1 $-bundle $ E_1\to L_1 $ is a $ \mathbb{P}^1 $-bundle. Furthermore, let  $ F_1 $ be a fibre of $ \mathbb{P}^1 $-bundle $ E_1\to L_1 $. Also, $ B_1\to B_0 $ is blowing up of $ B_0 $ along $ L_0\cap B_0=L_0 $, which is an isomorphism.
\end{itemize}
\textbf{Compute morphism}
$ E_1 $ is defined by $ x_1=0 $ in the chart $ x_2=1,v=1 $, i.e. $ \mathbb{A}^3 $ with coordinate $ (u,x_1,x_3) $, and the map is $ [x_1u^2:x_1u:x_1+u:ux_3] $. Then $ E_1\setminus L_1 $ maps to a line $ [0:0:u:ux_3]=\{y_0=y_1=0\} $ except the point $ [0:0:0:1] $. However on another chart $ x_2=1,v=1 $, where $ E_1 $ is defined by $ x_0=0 $, and the point is in its image.

\textbf{Compute Intersection numbers}
Since $ L_0 $ is of codimension $ 2 $, we have ramification formular
\[ K_{V_1}=\pi_1^* K_X+E_1. \]
Since $ \operatorname{mult}_{L_0}B_0=1 $, we have
\[ B_1+E_1=\pi^*B_0. \]
By projection formula and properties of fibres, we have
\[ \begin{cases}
	(K-E_1).F_1=0\\
	(K+E_1).F_1=-2\\
	(B + E_1).F_1 = 0
\end{cases} \implies
\begin{cases}
	K.F_1=-1\\
	E_1.F_1=-1\\
	B_1.F_1 = 1 
\end{cases}\]
and
\[
\begin{cases}
	(K-E_1).L_1=-4\\
	(B_1+E_1).L_1=1\\
	(K+B_1).L_1=-3\\
\end{cases}
\implies
\begin{cases}
	K.L_1=-3\\
	B_1.L_1=0\\
	E_1.L_1=1\\
\end{cases}
\]
If $ L' $ is a line s.t.  $ L\cap L_0\neq \emptyset $, let $ B' $ be the plan spaned by lines $ L',L_0 $, and 
\[
\begin{cases}
	(K-E_1).L'=-4\\
	(B_1'+E_1).L'=1\\
	(K+B_1').L'=-3\\
\end{cases}
\implies
\begin{cases}
	K.L'=-3\\
	B_1'.L'=0\\
	E_1'.L'=1\\
\end{cases}
\]
Implies that $ L'\equiv L_1 $; However, for $ L \cap L_0=\emptyset $, then
\[
\begin{cases}
	(K-E_1).L=-4\\
	(B_1+E_1).L=1\\
	B_1.L = 0\\
\end{cases}
\implies
\begin{cases}
	K.L=- 4\\
	B_1.L = 1\\
	E_1.L = 0\\
\end{cases}
\]
This implies that $ L\equiv L_1+F_1 $ 
Note that
\[ (K+E_1).L_1=K_{E_1}.L_1=-2 \implies L_1^2=0\]
where $ L^2_1=(L_1L_1.E_1) $ is self-intersection of $ L_1 $ in $ E_1 $, and implies $ E_1\cong \mathbb{F}_0 $.
\begin{center}
	\begin{tabular}{l|rrr}
		
		& $ K $ & $ B_1 $ &$ E_1 $  \\
		\hline
		$ L\equiv L_1+F_1 $ & $ -4  $ & $ 1 $ & $ 0 $  \\
		
		$ L_1 $ & $ -3 $ & $ 0 $ &$ 1 $  \\
		
		$ F_1 $ & $ -1 $ & $ 1 $ & $ -1 $ \\
		
	\end{tabular}
\end{center}
Then the classes of $ [L_1],[F_1]$ are extremal. 


\subsection{Blow up $ L_1 $}
\textbf{Compute blowing up}
Let $ \pi_2:V_2\to V_1  $ be the blowing up on $ V_1 $ along $ L_1 $, we may consider the affine piece $ U_1=\{x_2=1,v=1\}\cong \mathbb{A}^3 $:
\[ \xymatrix{
	\mathrm{Bl}_{L_1}\mathbb{A}^3\ar[r]\ar[rd]&\mathbb{A}^3\times \mathbb{P}^1,(u,x_1,x_3)\times (s,t)\ar[d]\\
	&\mathbb{A}^2,(u,x_1,x_3)	} \]
Here $ V_2 $ is defined by $ sx_1-tu=0 $. This blow up induces a birational map $ \Phi_2:V_2\to X' $.
\begin{itemize}
	\item On the $ s=1 $ piece, the map is
	\[ [tu^3:tu^2:tu+u:ux_3]=[tu^2:tu:tu+1:x_3] \]
	with base locus $ P=(0,-1,0) $ with coordinate $ (u,t,x_3) $.
	
	On this chart, we have $ E_2=\{u=0\},E'_1=\{t=0\} $ and $ P\in E_2\setminus(B_2\cup E'_1) $.  Denote $ L_2=E_2\cap B_2 $.
	\item On the $ t=1 $ piece, the map is
	\[ [x_1^3s^2:x_1^2s:x_1+x_1s:sx_1x_3]=[x_1^2s^2:x_1s:1+s:sx_3]\]
	with base locus$ P=(-1,0,0) $ with coordinates $ (s,x_1,x_3) $.
	
	On this chart, we have $ E_2=\{x_1=0\} $ and $ P\in E_2\setminus(B_2\cup E'_1) $
	\item In fact, $ \mathbb{P}^1 $-bundle $ E_2\to L_1 $ is a $ \mathbb{P}^1 $-bundle. Furthermore, let  $ F_2 $ be the fibre of $ \mathbb{P}^1 $-bundle $ E_2\to L_1 $ passing through $ P $. Also, $ B_2\to B_1 $ and $ E'_1\to E_1 $ are blowing ups of along $ L_1$, which are an isomorphism.
\end{itemize}
\textbf{Compute morphism}
$ E_2 $ is defined by $ u=0 $ in the chart $ x_2=1,v=1,s=1 $, i.e. $ \mathbb{A}^3 $ with coordinate $ (u,s,x_3) $, and the map is $ [tu^2:tu:t+1:x_3] $. Then $ E_2\setminus P $ maps to a line $ [0:0:t+1:x_3]=\{y_0=y_1=0\} $ except the point $ [0:0:0:1] $. However on the point is in its image.
\textbf{Compute Intersection numbers}
Since $ L_1 $ is of codimension $ 2 $, we have ramification formular
\[ K_{V_2}=\pi_2^* K_{V_1}+E_2. \]
Since $ \operatorname{mult}_{L_1}E_1=1 $ and $ \operatorname{mult}_{L_1}B_1=1 $, we have
\[\begin{cases}
	B_2+E_2=\pi^*B_1\\
	E'_1+E_2=\pi^*E_1
\end{cases}  .\]
By projection formula and properties of fibres, we have
\[ \begin{cases}
	(K-E_2).F_2=0\\
	(B_2+E_2).F_2=0\\
	(K+E_2).F_2=-2 \\
	\text{since $ F_2 $ is a rational curve and $ F_2^2=0 $ as a fibre}\\
	(E_1'+E_2).F_2=0
\end{cases} \implies
\begin{cases}
	K.F_2=-1\\
	B_2.F_2=1\\
	E_1'.F_2=1\\
	E_2.F_2=-1\\
\end{cases}\]
and
\[
\begin{cases}
	(K+E'_1)F_1=-2\\
	(K-E_2).F_1=-1\\
	(B_2+E_2).F_1=1\\
	B_2.F_1=0\\
	\text{since }E'_1\cap B_2=\phi
\end{cases}
\implies
\begin{cases}
	K.F_1=0\\
	B_2.F_1=0\\
	E_1'.F_1=-2\\
	E_2.F_1=1\\
\end{cases}
\]
and 
\[ 
\begin{cases}
	E'_1.L=0\\
	(K-E_2).L=- 4 \\
	(B_2+E_2).L =1\\ 
	(E_1'+E_2).L =0\\ 
\end{cases}
\implies
\begin{cases} 
	K.L=-4\\
	B_2.L = 1\\ 
	E_1'.L=0\\ 
	E_2.L = 0\\
\end{cases} 
\implies L\equiv L_2+F_1+2F_2
\]
and
\[
\begin{cases}
	E'_1.L_2=0\\
	(K-E_2).L_2=-3\\
	(B_2+E_2).L_2=0\\
	(E_1'+E_2).L_2=1\\
	(K + B_2).L_2 =- 3\\
\end{cases}
\implies
\begin{cases}
	K.L_2=-2\\
	B_2.L_2=-1\\
	E_1'.L_2=0\\
	E_2.L_2=1\\
\end{cases}
\]
In fact, this curve is numerically equivalent to rational curves in $ B_2 $; and let $ L_2'=\cap E_1\cap E_2 $ be another curves mapping to $ L_1 $, then 
\[
\begin{cases}
	B_2.L_2'=0\\
	(K-E_2).L_2'=-3\\
	(B_2+E_2).L_2'=0\\
	(E_1'+E_2).L_2'=1\\
	(K + E_1').L_2' =- 2\\
\end{cases}
\implies
\begin{cases}
	K.L_2'=- 3\\
	B_2.L_2' = 0\\
	E_1'.L_2'= 1\\
	E_2.L_2' 0\\
\end{cases}
\implies L_2'\equiv L_2 + F_2.
\]

Note that
\[ (K+E_2).L_2=K_{E_2}.L_2=-1 \implies L_2^2=-1\]
where $ L^2_2=(L_2L_2.E_2) $ is self-intersection of $ L_2 $ in $ E_2 $; furthermore, one can compute $ L'^2=0 $, this implies $ E_2\cong \mathbb{F}_1 $.
\begin{center}
	\begin{tabular}{l|rrrr}
		
		& $ K $ & $ B_2 $ &$ E_1' $ & $ E_2 $ \\
		\hline
		$ L\equiv L_2+F_1+2F_2 $ & $ -4 $ & $ 1 $ &$ 0 $&$ 0   $  \\
		
		$ L_2 $ & $ -2 $ & $ -1 $ &$ 0 $&$ 1 $  \\
		
		$ L_2'\equiv L_2+F_2 $ & $ -3 $ & $ 0 $ &$ 1 $&$ 0 $  \\
		
		$ F_1 $ & $ 0 $ & $ 0 $ &$ -2 $ &$ 1 $ \\
		
		$ F_2 $ & $ -1 $ & $ 1 $ & $ 1 $&$ -1 $ \\
		
	\end{tabular}
\end{center}
Then the classes of $ [L_2],[F_1],[F_2] $ are extremal. 
\subsection{Blow up $ P $}
\textbf{Compute blow up}
This time turns out that $ \Phi_3:V_3=Z\to X' $ is a morphism.

$ E_2'\to E_2 $ identifies with blowing up of $ \mathbb{F}_1 $ at a point on a fibre $ F_2 $ but not on the section $ L_1 $; and $ E_3\to P $ identifies with $ \mathbb{P}^2\to pt $. Denote $ F_3=E_2'\cap E_3 $.
\textbf{Compute morphism}
$ E_3 $ maps to $ B'\subset X' $.
\textbf{Compute Intersection numbers}
Since $ P $ is of codimension $ 3 $, we have ramification formular
\[ K_{V_3}=\pi_3^* K_{V_2}+2E_3. \]
Since $ \operatorname{mult}_{P}E_2=1 $ and $ P\notin E'_1, B_2 $, we have
\[\begin{cases}
	B_3=\pi^*B_2\\
	E'_2+E_3=\pi^*E_2\\
	E_1''=\pi^*E_1'
\end{cases}  .\]
By projection formula and properties of fibres, we have
\[ \begin{cases}
	(K-2E_3).F_3=0\\
	B_3.F_3=0\\
	(K+E_3).F_3=-3 \\
	\text{since $ F_3 $ is a rational curve in }E_3\cong \mathbb{P}^3\\
	(E_3+E_2').F_3=0\\
	E_1''.F_3=0
\end{cases} \implies
\begin{cases}
	K.F_3=-2\\
	B_3.F_3=0\\
	E_1''.F_3=0\\
	E_2'.F_3=1\\
	E_3.F_3=-1
\end{cases}\]
and
\[ \begin{cases}
	(K-2E_3).F_2=-1\\
	(E_3+E_2').F_2=-1\\
	(K+E_2').F_2=-1\\
	B_3.F_2=1\\
	E_1''.F_2=-1\\
\end{cases} \implies
\begin{cases}
	K.F_2=1\\
	B_3.F_2=1\\
	E_1''.F_2=1\\
	E_2'.F_2=-2\\
	E_3.F_2=1\\
\end{cases}\]
and let $ G_2 $ be a fibre of $ E_2\to L_1 $ not passing through $ P $, and identifies with its strict transform in $ E_2'\subset V_3 $
\[ \begin{cases}
	(K-2E_3).G_2=0\\
	(E_3+E_2').G_2=-1\\
	(K+E_2').F_2=-2\\
	B_3.G_2=1\\
	E_1''.G_2=-1\\
\end{cases} \implies
\begin{cases}
	K.G_2=-1\\
	B_3.G_2=1\\
	E_1''.G_2=1\\
	E_2'.G_2=-1\\
	E_3.G_2=0\\
\end{cases}
\implies
G_2\equiv F_2+F_3
\]
and
\[
\begin{cases}
	(K+E_1'').F_1=-2\\
	(K-2E_3).F_1=0\\
	B_3.F_1=0\\
	E_3.F_1=0\\
	E_2'.F_1=1\\
\end{cases}
\implies
\begin{cases}
	K.F_1=0\\
	B_3.F_1=0\\
	E_1''.F_1=-2\\
	E_2.F_1=1\\
	E_3.F_1=0\\
\end{cases}
\]
and
\[
\begin{cases}
	E''_1.L_2=0\\
	(K-2E_3).L_2=-2\\
	B_3.L_2=-1\\
	E_1''.L_2=0\\
	E_3.L_2=0\\
\end{cases}
\implies
\begin{cases}
	K.L_2=-2\\
	B_3.L_2=-1\\
	E_1''.L_2=0\\
	E_2.L_2 = 1\\
	E_3.L_2=0\\
\end{cases}
\]
and
\[
\begin{cases}
	E''_1.L_2' =1\\
	(K-2E_3).L_2'=- 3\\
	B_3.L_2' = 0\\
	E_1''.L_2' = 1\\
	E_3.L_2'=0\\
\end{cases}
\implies
\begin{cases}
	K.L_2'=- 3\\
	B_3.L_2' = 0\\
	E_1''.L_2' = 1\\
	E_2.L_2' = 0 \\
	E_3.L_2'=0\\
\end{cases}
\]
and
\[
\begin{cases}
	E''_1.L=0\\
	(K-2E_3).L=- 4\\
	B_3.L = 1\\
	E_1''.L=0\\
	E_3.L=0\\
\end{cases}
\implies
\begin{cases}
	K.L=- 4\\
	B_3.L =1\\
	E_1''.L=0\\
	E_2.L = 0\\
	E_3.L=0\\
\end{cases}
\implies
L\equiv L_2 + F_1 + 2F_2 + 2F_3
\]
\begin{center}
	\begin{tabular}{l|rrrrr}
		
		& $ K $ & $ B_3 $ &$ E_1'' $ & $ E_2' $&$ E_3 $ \\
		\hline
		
		$ L\equiv L_2 + F_1 + 2F_2 + 2F_3$ & $ -4 $ & $ 1 $ &$ 0 $&$ 0 $ &$ 0 $ \\
		
		$ L_2 $ & $ -2 $ & $ -1 $ &$ 0 $&$ 1 $ &$ 0 $ \\
		
		$ L_2'\equiv L_2+G_2 $ & $ -3 $ & $ 0 $ &$ 1 $&$ 0 $ &$ 0 $ \\
		
		$ F_1 $ & $ 0 $ & $ 0 $ &$ -2 $ &$ 1 $ &$ 0 $\\
		
		$ F_2 $ & $ 1 $ & $ 1 $ & $ 1 $&$ -2 $&$ 1 $ \\
		
		$ G_2\equiv F_2+F_3 $ & $ -1 $ & $ 1 $ & $ 1 $&$ -1 $&$ 0 $ \\
		
		$ F_3 $ & $ -2 $ & $ 0 $ & $ 0 $&$ 1 $&$ -1 $ \\
	\end{tabular}
\end{center}
Then the class of $ [L_2],[F_1],[F_2],[F_3] $ are extremal.
\section{Decomposition}
Using Sarkisov Program for terminal 3-fold without boundary:
\subsection{Set up}
Let $ X=X_0=\mathbb{P}^3 $, and $ H=H_0\in \mathcal{H}=\Phi^*\mathcal{H}' $, where $ \mathcal{H}' $ is linear system of planes on $ X' $. A member $ H\in \mathcal{H}$ is defined by
\[ ax_0^2+bx_0x_1+cx_0x_2+x_1^2+dx_0x_3=0 \]
mapping to the plane defined by
\[ ay_0+by_1+cy_2+dy_3=0 \]
in $ X' $. Therefore $ \mathcal{H}\subset |\mathcal{O}_X(2)| $, and hence
\[ \mu=\mu_0=-\frac{H.L}{K_X.L}=\frac{1}{2} \]
To compute $ \lambda $, we need ramification formulas
\[ K_Z=\pi^*K_X+E_1+2E_2+4E_3 \]
and
\[ H_Z+E_1+2E_2+3E_3=\pi^*H \]
The second is given by
\[ \begin{cases}
	V_1\to X: H\mapsto H_1+E_1\\
	V_2\to V_1: H_1 \mapsto H_2+E_2\\
	V_3\to V_1: H_2\mapsto H_3+E_3
\end{cases} \]
Thus $ \lambda =1 $ and
\[ K_Z+H_Z=\pi^*(K_X+H)+0E_1+0E_2+E_3 \]
which also implies $ e=2 $.
\subsection{First Sarkisov link}
This is $ \lambda > \mu $ case, first need to find a crepant extremal divisorial extraction $ p:Z_0\to X_0 $. In fact, this is exact the first blow up $ \pi_1:V_1\to X $, since
\[ K_{V_1}+H_{V_1}=\pi_1^*(K_X+H) \]
Note that $  K_{V_1}+H_{V_1}\sim K_{V_1}+\pi_1^*H-E_1\sim K_{V_1}+\pi_1^*(2B_0)-E_1\sim K_{V_1}+2B_1+E_1  $, we have
\begin{center}
	\begin{tabular}{c|rr}
		& $ L_1 $ & $ F_1 $ \\
		\hline
		$ K_{V_1}+H_{V_1} $ & $ -2 $ & $ 0 $ \\
	\end{tabular}
\end{center}

Here the class $[L_1]$ is extremal, therefore we have a MFS $ f_1:X_1=Z_0\to S_1$, where $ f_1 $ contracts all curves in the class of $ [L_1] $. This class contains all the rational curves that intersects $ E_1 $ and sections of $ E_1\to L_0 $. In fact, $ S_1\cong\mathbb{P}^1 $ and $ F_1\to S-1 $ is an isomorphism. Any point $ x\in V_1\setminus E_1 $, there is a plane $ B_x $ contains $ L_0 $ and $ \pi(x) $, and $ x $ maps to $ f(B_x'\cap F_1) $.


\subsection{Second Sarkisov link}
$ \mu_1=-\frac{H_1.L_1}{K.L_1}=\frac{1}{3} $ using the intersection number given above. $ \lambda=1 $ and the extremal divisorial extraction is exact $ \pi_2:V_2\to V_1=Z_0=X_1 $, and
\[ K_{V_2}+H_2\sim K_{V_2}+2B_2+E_1+2E_2 \]
and
\begin{center}
	\begin{tabular}{c|rrrr}
		& $ L_2 $ & $ L $ not extremal &$ F_1 $ &$ F_2 $ \\
		\hline
		$ K_{V_2}+H_{2} $ & $ -2 $ &$-2$&  $ 0 $&$ 0 $ \\
	\end{tabular}
\end{center}
Thus 2-ray game on $ Z_1=V_2 $ over $S_1$ w.r.t. $K+H_2$ started with a contraction $ q:Z_1=V_2\to X_2 $ for class of $[L_2]$. In fact, this contracts the divisor $B_2$. And it is followed by a MFS $ f_2:X_2\to S_2=S_1$.
\subsection{third link}
One except this link is the extraction of divisor $ E_3 $, followed by contraction of $ E_2 $.
\subsection{Fourth link}
One except this link the contraction of $ E_3 $, and the program ends.
\end{document}

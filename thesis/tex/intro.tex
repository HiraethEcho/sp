\chapter{绪论}
\section{主要结果}
双有理代数几何的目标之一是按双有理等价类分类代数簇,并选取选择恰当的代表元。极小模型纲领 (Minimal model program)是构造代表元的一种方法,并且猜想每一个代数簇都双有理等价于一个极小模型 (minimal model) 或一个森纤维空间 (Mori fibre space),但这样的代表元有时并不唯一,于是自然的问题就是不同代表元之间的关系。对于极小模型的情形,有以下事实:
\begin{theorem}[平转连接极小模型]\cite[Theorem 1]{flopmin}
  令$(W,B_{W})$为一个终端奇点的$\mathbb{Q}$-分解代数簇对,且 $(X,B_{X}), (Y,B_{Y})$是它的两个极小模型。那么双有理映射$X \dashrightarrow Y$可以分解为一系列$(K_{X}+B_{X})$-平转 (也称为复络, flop)的复合。
\end{theorem}

对于本文主要关注森纤维空间,有以下事实:
\begin{theorem}[Sarkisov分解]\cite[Theorem 1.1]{haconSarkisovProgram2012}\label{main}
  令 $ f:(X, B)\to S$ 和 $f':(X', B')\to S' $ 为两个 MMP-连接的klt奇点的$ \mathbb{Q} $-分解森纤维空间,则有双有理映射 $\Phi$:
  \[ \xymatrix{
      (X,B)\ar[d]_{f}\ar@{.>}[r]^\Phi & (X',B')\ar[d]^{f'}\\
      S & S'} \]
  可以分解为Sarkisov连接映射的复合,即
  \[ \Phi=\Psi_{n}\circ \cdots \circ \Psi_{1} \]
  其中$\Psi_{i}:X_{i}\dashrightarrow X_{i+1} $ 是下列四种Sarkisov连接之一:

  \textbf{第一型Sarkisov连接:}
  \[\xymatrix{
      Z\ar[d]_{p}\ar@{.>}[r]&X_{1}\ar[d]^{f_1}\\
      X\ar[d]_{f}&S_1\ar[dl]^{t}\\
  S &}\]
  \textbf{第二型Sarkisov连接:}
  \[\xymatrix{
      Z\ar[d]_{p}\ar@{.>}[r]&Z'\ar[d]^{q}&\\
      X\ar[d]_{f}&X_1\ar[d]^{f_1}\\
  S\ar[r]^{\sim}&S_1}\]
  \textbf{第三型Sarkisov连接:}
\[ \xymatrix{
    X\ar@{.>}[r]\ar[d]_{f}& Z\ar[d]^q& \\
    S\ar[rd]_{s}         & X_{1}\ar[d]^{f_{1}}&\\
    &S_{1}
    } \]
  \textbf{第四型Sarkisov连接:}
\[ \xymatrix{
      X \ar[d]_f\ar@{.>}[rr]&&X_{1}\ar[d]^{f_1}\\
      S \ar[dr]_{s}&&S_{1} \ar[dl]^{t}\\
      &T &} \]
  其中所有$ f:(X, B)\to S $ 和 $ f_1:(X_1, B_1)\to S_1 $ 都是森纤维空间,所有$p,q$ 都是除子压缩,所有虚线的映射都是翻转 (flip)、平转 (flop)或反向翻转 (inverse flip)的复合。
\end{theorem}
这样的分解称为Sarkisov分解,构造这样Sarkisov分解的方法称为Sarkisov纲领。

本文给出了关于叶层化代数簇对 (foliated pair)的新的结果:
\begin{theorem}[叶层化对的弱Sarkisov分解, 王延泽]\label{mainf}
  令$(W,\mathcal{F}_{W},B_{W})$是具有$F$-dlt奇点的$\mathbb{Q}$-分解叶层化代数簇对,若$\rho:W\dashrightarrow X$ 和$\rho':W \dashrightarrow X'$是两个不同的$(K_{\mathcal{F}_{W}}+B_{W})$-MMP的输出,且$X \to S$和$X' \to S'$是两个森纤维空间。那么双有理映射$\Phi:X \dashrightarrow X'$存在分解:
  \[ X=X_{0}\dashrightarrow X_{1}\dashrightarrow \cdots \dashrightarrow X'. \]
  每一个代数簇$X_{i}$上有边界除子$D_{i}$和压缩态射$f_{i}:X_{i}\to S_{i}$,使得$(X_{i},D_{i})\to S_{i}$是森纤维空间,并且$X_{i} \dashrightarrow X_{i+1}$是定理\ref{main}中定义的Sarkisov连接。
\end{theorem}

\section{问题的历史}
Sarkisov纲领起源于对直纹曲面的分类 \cite{sarkisovBIRATIONALAUTOMORPHISMSCONIC1981,sarkisovCONICBUNDLESTRUCTURES1983},Matsuki和Reid指出了最初的思路,具有终端奇点的三维代数簇 (terminal threefolds)上的Sarkisov纲领的完整证明由Corti\cite{cortiFactoringBirationalMaps}给出。 

对于双有理同构的两个森纤维空间$X\to S$和 $X'\to S'$,选取$X$ 上一个定义双有理映射$\Phi:X \dashrightarrow X'$的线性系 $\mathcal{H}$ (或一个一般的除子 $H \in \mathcal{H}$),那么第一个Sarkisov连接 $\psi_1:X\dashrightarrow X_1$ 由运行一种特殊的极小模型纲领得到,被称为双射线MMP ($2$-ray game),并且取决于 $\mathcal{H}$ ($H$) 的选取。接着用 $\Phi_{1}=\Phi\circ \psi_1^{-1}: X_1 \dashrightarrow X'$替代 $\Phi:X\dashrightarrow X'$并重复这一过程。通过定义Sarkisov次数,并证明在归纳构造中Sarkisov次数下降来说明归纳终结。Bruno和Matsuki \cite{brunoLogSarkisovProgram1995} 将这种方法推广到klt奇点的 $\mathbb{Q}$-分解三维代数簇的情形,并且对于任意维数的$\mathbb{Q}$-分解klt奇点代数簇对,给出了Sarkisov纲领的大纲和所需要的结论。近年来在极小模型纲领中有一些重要进展,例如标量极小模型纲领 (MMP with scaling) 的终结性 \cite{BCHM10},对数典范阈值 (log canonica threshold, lct)的升链条件 (accending chain condition, ACC)\cite{HMX14}, $\delta$-lc Fano代数簇对的有界性 \cite{Bir19,birkarSingularitiesLinearSystems2020},这些进展使得Bruno和Matsuki的大纲部分地可行,剩下的主要问题与对数翻转的终结性和局部对数典范阈值 (local log canonical thresholds)的ACC (或者有限性)有关。本文将这种方法称为下降法。


利用弱典范模型的有限性\cite{BCHM10} (finiteness of weak log canonical models),Hacon \cite{haconMinimalModelProgram2012} 给出了另一种构造Sarkisov纲领的方法,对所有维数都成立。
这种方法也通过双射线MMP来构造Sarkisov连接,但是固定了两个森纤维空间的公共对数解消 $(W,B_W)$ 作为Sarkisov纲领的``屋顶'',使得分解中的每一个森纤维空间都是$W$的某个弱对数典范模型。双标量法的Sarkisov纲领的终结性由弱对数典范模型的有限性推出,这与标量翻转 (flips with scaling)的终结性的证明类似。
刘继豪 \cite{liuSarkisovProgramGeneralized2021} 将这种方法推广到了一般化代数簇对 (generalized pairs)的情况。本文将这种方法称为双标量法。


利用Shokurov的多面体方法 \cite{Sho96,cs11}, Hacon 和 M\textsuperscript{c}Kernan \cite{haconSarkisovProgram2012}给出了一种新方法,不通过双射线MMP构造Sarkisov连接。
令 $W$ 为 $(X,B)\to S$ 和 $(X',B')\to S'$ 的公共对数解消,则在 $W$ 上有两个除子 $D$ 和 $D'$,使得 $S$ 和 $S'$ 是 $W$ 相对于 $D$ 和 $D'$的丰沛模型 (ample model)。进一步,$W$的除子的多面体的边界上有其他除子 $D_{i}$,每个除子对应一个森纤维空间 $X_{i}\to S_{i}$ 和 $W$的丰沛模型$S_{i}$。在多面体边界上有一条路径连接这些除子 $D_{i}$,并且将$\Phi$分解为对应的Sarkisov连接。
Miyamoto \cite{miyamoto2019TheSP} 将这种方法应用到了任意特征代数闭域上的lc对数曲面或 $\mathbb{Q}$-分解对数曲面。 本文将这种方法称为有限模型法。

\section{本文结构}
第一章给出主要的定理,并介绍Sarkisov纲领的发展历史。第二章给出基本概念的定义,MMP相关的定理。

第三、四、五章分别介绍下降法、双标量法和有限模型法的具体内容,每一章都先介绍这种方法需要的定义和引理,接着说明构造每一个Sarkisov连接的方法,最后说明Sarkisov分解的构造。

在第六章,总结比较三种方法,并且每种方法给出一个具体算例。
% 最后介绍一些Sarkisov纲领的简单应用。

第七章介绍了叶层化代数簇对 (foliated pairs),并分析了三种方法在叶层化代数簇对的推广,最后给出定理\ref{mainf}的证明。

\chapter{预备知识}
\section{代数簇对与奇点}

% TODO: contraction definition; resolution; snc; intersection?
定义代数簇对和相关概念:
\begin{definition}
  \begin{itemize}
    \item 一个代数簇对 $(X,B)$ 包括一个正规拟射影代数簇 $X$ 和一个 $\mathbb{R}$-除子 $B=\sum_{i}b_{i}B_{i}$,其中$B_{i}$是素除子,且$0<\leqslant b_{i} \leqslant 1$ ,并且满足
    \[ K_{X}+B \]
          是$\mathbb{R}$-Cartier 除子。 
    \item 若双有理态射 $f:Y\to X$ 满足 $Y$是光滑代数簇,且$\operatorname{Exc}\,f \cup \bigcup_{i}B_{i \text{red}} $ 是横截相交 (normal crossing)的除子,并且$f$ 在$\operatorname{Reg}(X,B)$上是同构, 那么称 $f$ 为 $(X,B)$的算术解消。  
    \item 正规代数簇$X$ 的余维数1的不可约子簇$ E \subset X $ 称为 $X$ \textbf{上}的除子;如果 $f: Y \to X$ 是双有理射影态射,那么 $Y$ 上的除子 $E$ 称为 $X$ \textbf{之上}的除子。  
  \end{itemize}
\end{definition}

定义除子的差异数:
\begin{definition}
  令 $(X, B)$ 为代数簇对, 且 $f: Y\to X$ 是它的算术解消,则有
  \[
    K_{Y}+C=f^*(K_{X}+B)
    \]
  那么除子 $E$ 的差异数 (discrepancy)$a(E;X,B) $定义为
  \[
    a(E;X,B)=-\operatorname{mult}_{E}C
    \]
  进一步,定义$(X,B) $的差异数:
  \[
    \operatorname{discrep}(X, B) := \inf\{a(E; X, B) : E \text{ 是 } X \text{之上的例外除子} \}
  \]
  和整体差异数
  \[
    \operatorname{totdiscrep}(X, B) :=\inf \{a(E; X, B) : E \text{是} X \text{之上的除子}\}.
  \]
\end{definition}
对于代数簇对之间的态射,可以定义
\begin{definition}
 令$f:(Y,C)\to (X,B)$ 为代数簇对之间的态射,在分歧公式中,
 \[ K_{Y}+C + \sum_{i}e_{i}E_{i}=f^{*}(K_{X}+B) \]
 如果$e_{i}=0$,那么除子$E_{i}$称为$f$-相容的。 如果
 \[ K_{Y}+C=f^{*}(K_{X}+B) \]
那么称$f$ 是\textbf{无差别的},也称为\textbf{相容的}  (crepant)。
如果考虑不含边界的代数簇对,即$B=C=0$ ,那么分歧公式
 \[ K_{Y} + \sum_{i}e_{i}E_{i}=f^{*}K_{X} \]
 中满足$e_{i}=0$的除子称为无差别除子。
\end{definition}

通过差异数,定义代数簇对 $(X,B)$ 的奇点性质:
\begin{definition}
  对于代数簇对$(X,B)$,定义奇点性质:
  \begin{itemize}
    \item 如果 $\operatorname{discrep}(X,B)>0$,则称$(X,B) $具有终端奇点 (terminal),也称 $K_{X}+B$具有终端奇点; 
    \item 如果 $\operatorname{discrep}(X,B)\geqslant 0$,则称$(X,B) $具有典范奇点 (canonical),也称 $K_{X}+B$具有典范奇点; 
    \item 如果 $\operatorname{discrep}(X,B)>-1$,则称$(X,B) $具有klt奇点 (kawamata log terminal),也称 $K_{X}+B$具有klt 奇点; 
    \item 如果 $\operatorname{discrep}(X,B)\geqslant -1$,则称$(X,B) $具有lc奇点 (log canonical),也称 $K_{X}+B$具有lc奇点; 
    \item 如果 $\operatorname{discrep}(X,B)\geqslant -1+\delta$,则称$(X,B) $具有$\delta$-lc奇点,也称 $K_{X}+B$具有$\delta$-lc奇点; 
  \end{itemize}
\end{definition}

\section{极小模型纲领}

\begin{theorem}[锥定理]\label{conethm}

令$(X,B)$ 是具有klt奇点的$\mathbb{Q}$-分解代数簇对,且$(K_{X}+B)$不是数值有效的,那么有
\begin{enumerate}
  \item
    \[ \overline{\operatorname{NE}}(X)=\overline{\operatorname{NE}}(X)_{K_{X}+B\geqslant 0} +\sum_{\alpha \in\Lambda} R_{\alpha}\] 
          其中$R_{\alpha}=\mathbb{R}_{\geqslant 0}[C_{\alpha}]$是由有理曲线$C_{\alpha}$生成的$(K_{X}+B) $-负性的极端射线,且$\Lambda$是可列集;
  \item 对任何丰沛除子$A$和$ \epsilon >0 $,有
        \[ \overline{\operatorname{NE}}(X)=\overline{\operatorname{NE}}(X)_{K_{X}+B+\epsilon A\geqslant 0} +\sum_{\alpha \in\Lambda'}R_{\alpha} \]
        其中$\Lambda' \subset \Lambda$是有限子集;
  \item 令 $F \subset \overline{\operatorname{NE}}(X)$是 $(K_{X}+B)$-负性的极端面,那么同构意义下存在唯一的压缩态射
    \[ f=cont_{F}:X \to Z \]
    使得$f_{*}\mathcal{O}_{X}=\mathcal{O}_{Z}$,并且曲线 $C \subset X$被$f$ 压缩当且仅当$[C] \in F$。 
  \item 如果$X$ 上除子 $D$,相对于$Z$ 是数值平凡的,即任意被$f$ 压缩的曲线$C \subset X$,都有$D.C=0$,那么存在$Z$上除子$L_{Z} $使得
    \[ f^{*}L_{Z} = D \]
\end{enumerate}
\end{theorem}
当$F=R$是极端射线时,有$\rho(X)=\rho(Z)+1$。压缩态射$f$有三种情况:
\begin{enumerate}
  \item $\operatorname{Exc}\,f=E$是素除子,此时称为除子压缩;
  \item $\operatorname{codim }(\operatorname{Exc}\,f) \geqslant 2$,则称为小双有理态射 (small birational morphism);
  \item $\dim Z < \dim X, \operatorname{Exc}\,f=X$,此时称$(X,B)\to Z$为森纤维空间 (关于$ (K_{X}+B) $的森纤维空间),压缩态射$f$称为森纤维空间压缩态射。
\end{enumerate}
\textbf{极小模型纲领 (MMP):}
对于$(X,B)=(X_{0},B_{0})$,应用锥定理做关于极端射线的压缩态射$f_{1}:X\to Z$。
\begin{enumerate}
  \item 如果是除子压缩,那么令$X_{1}=Z,B_{1}=f_{1*}B$。在$(X_{1},B_{1})$上重复此过程;
  \item 如果是小双有理态射,那么令$f^{+}_{1}:X_{1}\to Z$为$f_{1}$的翻转 (flip),并记$B_{1}$为$B$ 在$X_{1}$上的严格双有理变换。在$(X_{1},B_{1})$上重复此过程;
  \item 如果是森纤维空间压缩,那么MMP停止。
\end{enumerate}
如果持续做前两种操作,并最终在$(X_{i},B_{i})$上$K_{X_{i}}+B_{i}$是数值有效除子,那么MMP在此终结,且$(X_{i},B_{i})$是极小模型;
如果在森纤维空间$(X_{i},B_{i})\to Z$停止,那么MMP终结于这个森纤维空间。


MMP中出现的代数簇对被称为极小模型的\textbf{结果},将MMP停止处的代数簇称为MMP的\textbf{输出} (要么是极小模型,要么是森纤维空间)。对于极小模型纲领,有如下结果。
\begin{theorem}[标量MMP的终结定理]
  \cite[Corollary 1.4.2]{BCHM10} 令 $ \pi: X\to U $ 为正规拟射影代数簇间的射影态射,且 $(X, B)$ 是  $\mathbb{Q}$-分解的具有 klt奇点的 代数簇对,其中 $K_{X}+B$  $\mathbb{R}$-Cartier 除子, 且$B$ 是 $\pi$-big。若 $C\geqslant0$ 为 $\mathbb{R}$-除子,且$K_{X}+B+C$ 具有 klt奇点 且  $\pi$-数值有效 (nef),那么在 $U$ 上运行 $C$-标量的  $(K_{X}+B)$-MMP,那么这个极小模型纲领将终结。
\end{theorem}

如果$(K_{X}+B)$-MMP终结,那么其输出和$(K_{X}+B)$有下列关系:
\begin{theorem}[极小模型输出]\label{notpseudoeffmfs}
  \cite[Corollary 1.3.3]{BCHM10} 令 $ \pi: X\to U $ 为正规拟射影代数簇间的射影态射,且 $(X, B)$ 是 $\mathbb{Q}$-分解 klt 代数簇对,其中 $K_{X}+B$ 是$\mathbb{R}$-Cartier 除子。若 $K_{X}+B$ 不是 $\pi$-伪有效的,那么运行 $U$ 上的  $(K_{X}+B)$-MMP,将终结于森纤维空间$g:Y\to Z$。
\end{theorem}
注意到如果$ \pi:X \to Y$是双有理射影态射,那么$X$ 上所有除子都是相对于$Y$ 的大除子,于是有以下推论:   
\begin{corollary}\label{extraction}
  \cite[Corollary 13.7]{haconMinimalModelProgram2012} 令 $ (X,B) $ 为 klt 代数簇对, $\mathfrak{C}$是任意差异数满足 $ a(E;X,B)\leqslant 0 $的例外除子 $E$ 的集合,那么有双有理态射 $ f:Z\to X $ 和 $ \mathbb{Q} $-除子 $ B_Z $ 使得:
  \begin{enumerate}
    \item $ (Z,B_Z) $ 是klt代数簇对:
    \item $ E $ 是 $f$-例外除子当且仅当 $ E\in \mathfrak{C} $;
    \item  若 $E \in \mathfrak{C}$则$ \operatorname{mult}_{E}B_Z=-a(E;X,B) $ ,且$ f_*B_Z=B $ 和 $ K_Z+B_Z=f^*(K_X+B) $。
  \end{enumerate}
  特别的,若设 $\mathfrak{C}$ 为所有差异数满足$a(E; X, B)\leqslant 0$的例外除子 $E$ 的集合,那么 $ Z $ 被称为 $X$ 的 \textbf{终端化} (terminalization) ;若取 $\mathfrak{C}$为仅包含一个差异数满足 $a(E; X, B)\leqslant 0$的例外除子,那么 $ f: Z\to X $ 被称为 \textbf{除子解压} (\textbf{divisorial extraction}).
\end{corollary}
对相同的代数簇对$(X,B)$和算术典范除子$K_{X}+B$,对应的$(K_{X}+B)$-MMP可能有不同的输出,它们之间有下列关系:
\begin{definition}
  \cite[Definition 3.3]{brunoLogSarkisovProgram1995}
  如果多个代数簇对 $ \{(X_i,B_i)\} $是从算术光滑的代数簇对 $(W,B_{W})$的 $(K_{W}+B_{W}) $-MMP的不同结果,则称它们为MMP-相关的 ( \textbf{MMP-related} )。 
\end{definition}

\begin{lemma}\label{MMPrelatedConditation}
  \cite[Proposition 3.4]{brunoLogSarkisovProgram1995}
  令 $ \{(X_l,B_l)\} $ 为有限多个互相双有理等价的 $ \mathbb{Q} $-分解 klt 代数簇对,那么下列条件等价:
  \begin{enumerate}
    \item 它们是MMP-相关的;
    \item 存在一个算术光滑代数簇对 $ (W,B_W) $和一组射影双有理态射  $ f_l:W\to  X_l $ 支配每个 $ X_l $,满足 $ f_{l*}B_W=B_l $ 和分歧等式
      \[ K_W+B_W=f_l^*(K_{X_l}+B_l)+\sum_{\text{例外除子}E_{li}}{a_{li}E_{li}} \]
          其中对每个$ f_l $-例外除子$E_{li}$满足 $a_{li}>0$ 的不等式条件;
    \item 对任意两个代数簇对 $ (X,B=\sum_ib_{i }B_i),(X',B'=\sum_{j}b_{j}'B_{j}') $ , 有  $ a(B_i;X',B')\geqslant -b_i $ 且严格不等式成立当且仅当 $ B_i $ 是 $ X' $上的例外除子。同样的,有 $ a(B'_j;X,B)\geqslant -b'_j $ 且严格不等式成立当且仅当$ B'_j $ 是 $ X $上的例外除子。
  \end{enumerate}
\end{lemma}
\begin{proof}
  我们给出  $(3) \implies (2)$的简略证明:令 $W$ 为支配每个代数簇对 $(X_l,B_l=\sum b_{li}B_{li})$ 的算术光滑解消,并有射影双有理态射 $f_l:W\to X_l$,它们例外除子的并$f_{l*}^{-1}B_l\cup E_{li}$是一个横截相交的除子。令 $B_W=\sum_t d_tD_t $,其中  如果 $D_t$ 是$\cup_l f_{l*}^{-1}B_l$中的某个素除子则$d_t = b_{li}$,如果$B_{t}$是每个 $X_{l}$上的例外除子,则  $d_t=1$。由条件(3), 这是定义良好的。那么 $(W,B_{W})$上的分歧等式(ramification formula)中的不等式条件也由(3)得到。
\end{proof}

\section{模型及其有限性}

\begin{definition}
  \cite[\S 2]{haconSarkisovProgram2012} 对有理映射 $f:X\dashrightarrow Y$  若有 $f$的解消$p:W\to X$ 和 $q:W\to Y$ 满足$p$  和 $q$ 都是压缩态射且 $p$ 双有理态射,则称 $f$ 为有理压缩映射( \textbf{rational contraction})  。若 $ q$也是双有理态射, 且每个 $p$-例外除子都是 $q$-例外除子,则称 $f$为双有理压缩映射 (\textbf{birational contraction})。如果 $f^{-1}$ 也是双有理压缩映射,则称 $f$ 为小双有理映射   ( \textbf{small birational map} )。
\end{definition}

\begin{definition}\label{negativemap}
  \cite[Definition 3.6.1]{BCHM10}令 $f:X\dashrightarrow Y$为正规拟射影代数簇间的双有理映射,且 $p:W\to X$ 和 $q:W\to Y$是 $f$的解消。若 $D$是 $X$ 上的$\mathbb{R}$-Cartier 除子,满足  $D_{Y}=f_*D$ 也是 $\mathbb{R}$-Cartier除子,那么如果满足
  \begin{itemize}
    \item $f$ 不解压任何除子(即 $f$ 是双有理压缩 );
    \item $E=p^{*}D-q^*D_Y$ 是  $Y$上的有效除子 (对应的, $\operatorname{Supp}p_*E$ 包含全部 $f$-例外除子)。
  \end{itemize}
 则称$f$为 $D$-非正性的  (\textbf{$D$-non-positive}) ,对应的, $D$-负性的 ( \textbf{$D$-negative)}。
\end{definition}

回顾双有理代数几何中关于模型的定义 \cite{BCHM10}:
\begin{definition}
  \cite[Definition 3.6.5]{BCHM10} 令 $ \pi:(X,D)\to U $为正规拟射影代数簇间的射影态射, $K_{X}+D$ 是 $X$ 上的$\mathbb{R}$-Cartier除子,且$ f: X\dashrightarrow Y $是 $U$ 上的双有理映射。  如果 $ f $ 是 $ (K_X+D) $-非正性的且 $ K_Y+f_*D $ 是 $ U $上半丰沛的,那么称 $Z$ 和 $f$ 为关于 $D$的\textbf{半丰沛模型}(\textbf{semiample model})。

  令 $ g:X\dashrightarrow Z $ 为$ U $上的有理映射,$p:W \to X $ 和 $q:W \to Z $是对 $g$ 的解消,其中 $q$ 是压缩态射 。 若$Z$ 上有 $U$ 上的丰沛除子 $H$ ,且 $p^*(K_{X}+D) \sim_{\mathbb{R},U} q^*H+E$ ,其中 $E$  满足    对 任意的 $B \in |p^*(K_{X}+D)/U|_{\mathbb{R}}$都有 $B\geqslant E$,则称 $Z$ 是 $X$ 关于$D$ 的\textbf{丰沛模型} (\textbf{ample model}) 。
\end{definition}

\begin{remark}
  在原文献和其他文献中的常见定义如下:令 $ \pi:X\to U $为正规拟射影代数簇间的射影态射, $D$ 是 $X$ 上的$\mathbb{R}$-Cartier除子,且$ f: X\dashrightarrow Y $是 $U$ 上的双有理映射。  如果 $ f $ 是 $ D $-非正性的且 $ f_*D $ 相对于 $ U $是半丰沛的,那么称 $Z$ 和 $f$ 为关于 $D$的\textbf{半丰沛模型}(\textbf{semiample model})。

  但本文中只考虑$D=K_{X}+B$是算术典范除子的情况,并且为记号简便做此修改。丰沛模型、弱算术典范模型等都做此修改。
\end{remark}
\begin{definition}\label{models}
  \cite[Definition 3.6.7]{BCHM10} 令 $ \pi:(X,D)\to U $为正规拟射影代数簇间的射影态射,若 $ K_X+D $ 是 lc且$ f:X\dashrightarrow Y $是双有理压缩映射,那么有如下定义:
  \begin{enumerate}
    \item 如果 $f$ 是  $ (K_X+D) $-非正性的且 $ K_Y+f_*D $ 是 $ U $上数值有效的,则  称$ Y $为关于 $D$ 在 $U$上 的  \textbf{弱算术典范模型}(\textbf{weak log canonical model});
    \item 如果 $f$ 是  $ (K_X+D) $-非正性的且 $ K_Y+f_*D $ 是 $ U $上丰沛的,则  称$ Y $为关于 $D$ 在 $U$ 上的  \textbf{算术典范模型}(\textbf{ log canonical model});
    \item 如果 $f$ 是  $ (K_X+D) $-负性的且 $ K_Y+f_*D $ 是 $ U $上数值有效的和$\mathbb{Q}$-分解的,并且具有dlt奇点,则  称$ Y $为关于 $D$ 在 $U$上的  \textbf{算术终端模型}(\textbf{log terminal model})。
  \end{enumerate}
\end{definition}

\begin{lemma}\cite[lemma 3.6.6]{BCHM10}
  令 $\pi:X \to U$ 是正规拟射影代数簇间的射影态射,且 $D$是 $X$ 上的 $\mathbb{R}$-Cartier除子。

  \begin{enumerate}
    \item 如果 $g_{i}:X \dashrightarrow X_{i}, i=1,2$ 是 关于 $D$的  $U$上的两个丰沛模型,那么有同构态射 $h:X_{1}\to X_{2}$ 满足 $g_{2}=h \circ g_{1}$。即丰沛模型在同构意义下唯一。
    \item 如果 $f:X \dashrightarrow Y$是 $U$ 上关于 $D$ 的 半丰沛模型,那么 $U$ 上关于 $D$ 的丰沛模型 $g:X \dashrightarrow  Z$存在,并且 $g=h \circ f$,其中 $h:Y \to Z$是压缩态射, $Z$ 上有对应丰沛除子 $H$  满足$f_*D \sim_{\mathbb{R},U}h^*H$。
    \item  若 $f:X \dashrightarrow Y$  是$U$上双有理映射,那么 $f$是关于$D$ 在 $U$上的丰沛模型当且仅当$f$是关于 $D$ 在 $U$ 上的  半丰沛模型 且 $f_*D$ 在 $U$上丰沛。
  \end{enumerate}
\end{lemma}
根据上述引理,有算术典范模型的等价定义:

\begin{definition}
  令 $ \pi:(X, D)\to U $是正规拟射影代数簇间的射影态射,$ K_X+D $ 有lc奇点且$ f: X\dashrightarrow Y $ 是不解压任何除子的双有理映射。 如果 $ Y $是 关于 $D$ 在 $U$ 上的丰沛模型,那么称之为\textbf{算术典范模型} ( \textbf{log canonical model} ) 。
\end{definition}

进一步,对于边界是大除子的代数簇对,还有
\begin{lemma}\cite[lemma 3.9.3]{BCHM10} 令 $ \pi:(X,B)\to U $是正规拟射影代数簇间的射影态射,且 $(X, B)$是具有klt 奇点的代数簇对,  $B$ 在 $U$上是大除子。如果 $f:X\dashrightarrow Y$是 $U$ 上 弱算术典范模型,那么
  \begin{itemize}
    \item $f$ 是$U$上半丰沛模型;
    \item  $U$ 上的丰沛模型  $g:X \dashrightarrow Z$存在;
    \item  存在压缩态射$h:Y\to Z$和 $Z$ 上在 $U$ 上丰沛的$\mathbb{R}$-除子,使得 
      \[ K_{Y}+f_*B\sim_{\mathbb{R},U} h^*H \]
  \end{itemize}
\end{lemma}
下面给出除子的多面体相关的定义和定理:
\begin{definition}\label{polytopeofdivisor}
  \cite[Definition 1.1.4]{BCHM10} 令 $ \pi: X\to U $ 为正规拟射影代数簇间的射影态射,且 $ V $是 $ \operatorname{WDiv}_{\mathbb{R}}(X) $的定义在有理数上的优先为子射影空间。取定一个 $ \mathbb{R} $-除子 $ A\geqslant 0 $,定义:
  \[
    \begin{aligned}
      \mathcal{L}_A(V)       & =\{D=A+B:B \in V,  K_X+D\, \text{有lc奇点且} B\geqslant0 \} \\
    \mathcal{E}_{A,\pi}(V) & =\{D\in \mathcal{L}_A(V): K_X+D\, \text{ 是 } U \text{上的伪有效除子}\}  \\
    \end{aligned}
  \]
  令 $ f:X \dashrightarrow Y$为 $U$ 上的 双有理压缩映射,定义
  \[ \mathcal{W}_{A,\pi,f}(V)=\{D\in \mathcal{E}_{A}(V): f \text{是   } (X,D) \text{ 在 }U \text{上的弱算术典范模型}\} \]
  令 $g:X\dashrightarrow Z  $ 是 $ U $上的有理压缩映射,定义
  \[ \mathcal{A}_{A,\pi,g}(V)=\{D\in \mathcal{E}_{A}(V): g \text{是} (X,D) \text{ 在 }U \text{上的丰沛模型}\} \]
  进一步,将 $ \mathcal{A}_{A,\pi,g}(V) $ 在 $\mathcal{L}_{A}(V)$中的闭包记作 $ \mathcal{C}_{A,\pi,g}(V) $。

  如果基底 $U$是清楚的,或是一个点,那么我们省略 $\pi$,简单记作 $\mathcal{E}_{A}(V)$ 和 $\mathcal{A}_{A,f}$。
\end{definition}

\begin{theorem}\label{finitewlcm}
  (弱算术典范模型有限性, \cite[Theorem E]{BCHM10}).
  令 $\pi: X\to U$是正规拟射影代数簇间的射影态射,且$A$是一个一般的 在 $U$ 上丰沛的$\mathbb{R}$-除子,且$V \subset \operatorname{WDiv}_{\mathbb{R}}(X)$是定义在有理数上的有限维线性子空间,假设存在具有klt奇点的代数簇对 $(X,\Delta_{0})$。那么存在有限多个 $U$ 上的 双有理映射 $f_{i}:X \dashrightarrow X_{i},1\leqslant i\leqslant l$ ,若某个 $D \in \mathcal{L}_{A}(V)$ 有关于 $D$ 的在 $U$ 上的  弱算术典范模型 $f:X \dashrightarrow  Y$,那么对某个$1\leqslant i\leqslant l$存在同构态射  $h_{i}:X_{i} \to Y$ 使得 $f=h_{i}\circ f_{i}$。
\end{theorem}

\section{叶状结构}

这一节介绍带叶状结构的代数簇对的基本知识。

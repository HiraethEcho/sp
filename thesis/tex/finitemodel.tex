\chapter{有限模型法}\label{thirdmethod}
这一章主要参考 \cite{haconSarkisovProgram2012}。 通过有限模型来构造Sarkisov分解和前两种方法不同,这种方法并不通过运行2-ray game来构造Sarkisov连接。下面简要介绍这种方法的思路:

令 $W$为两个MMP-相关的森纤维空间 $X\to S$ 和 $Y\to T$公共奇点解消。 Take a finite dimensional affine subspace $V$ 取 $\operatorname{WDiv}_{\mathbb{R}}(W)$的有限维仿射子空间$V$和丰沛 $\mathbb{Q}$-除子 $A$,那么 $\{\mathcal{A}_{i} =\mathcal{A}_{A,f_{i}}\} $ 是 $\mathcal{E}_{A}(V)$的一个划分,且每个子集 $\mathcal{A}_{i}$ 对应 $W$的一个丰沛模型,相邻的子集对应的丰沛模型之间有压缩态射 (见 定理\ref{mapbetweenAM})。由引理\ref{keylemma},可以取到一个特殊的子空间 $V$ ,使得 
\begin{enumerate}
  \item $V$ 是2维的有理仿射空间(可以定义在有理数上); 
  \item $S,T$ 是 $W$ 关于 $D_{S},D_{T} \in \mathcal{L}_{A}(V) $的丰沛模型;
  \item $D_{S}$ 和 $D_{T}$ 是$\mathcal{L}_{A}(V)$边界 $\partial \mathcal{L}_{A}(V)$上的两个点,并且将边界分为两个部分。在其中一个部分上,有有限多个线段连接 $D_{S}$ 和 $D_{T}$,并且$D_{i}$是线段的端点,此时每个端点 $ D_{i}$对应一个Sarkisov连接。

    (图 \ref{pic}为一个示例图,其中$D_S=D_0,D_T=D_1$)
  \item 双有理映射$X\dashrightarrow Y$是这些Sarkisov 连接的复合。这就是有限模型法构造的Sarkisov分解。
\end{enumerate}
\begin{remark}
  有限模型法的Sarkisov纲领严格来说并不是一个纲领 (program),和前两种方法不同,不是归纳地构造分解中的森纤维空间$X_{i}\to S_{i}$,而是先找出$W$ 关于$D \in \mathcal{L}_{A}(V)$的丰沛模型,然后选择其中的一些。
\end{remark}


\section{定义和引理}

首先给出 $\mathcal{E}_{A}(V)$ 关于丰沛模型的划分定理:
\begin{theorem}\label{finitemodel}
  \cite[Corollary 1.1.5]{BCHM10} 令 $\pi:X\to U$ 是正规拟射影代数簇间的射影态射,并且 $V \subset \operatorname{WDiv}_{\mathbb{R}}(X)$是有限维有理子空间,假设存在一个除子 $\Delta_{0} \in V$ 使得 $(X,\Delta_{0})$ 具有klt奇点。令 $A$是 相对于 $U$ 的一般的丰沛 $\mathbb{Q}$-除子,并且它和任何$V$ 中的除子都没有共同的素除子作为组成成分。那么
  \begin{enumerate}
    \item   存在有限多个相对于 $U$ 的双有理映射  $f_{i}:X \dashrightarrow   X_{i}$ 使得:
          \[ \mathcal{E}_{A,\pi}(V) =\bigcup_{i}\mathcal{W}_{i} \]
          其中  $\mathcal{W}_{i}=\mathcal{W}_{A,f_{i}}(V)$ 是定义在有理数上的多面体  (rational polytope)。且如果  $f:X \dashrightarrow  Y$ 是关于对某个 $D \in \mathcal{E}_{A,\pi}(V)$相对于 $U$ 的算术终端模型,那么对某个$i$ ,有  $f=f_{i}$ 。即 $X$ 关于$\mathcal{E}_{A,\pi}(V)$的算术终端模型一定是$f_{i}$中的一个。   
    \item   存在有限多个相对于$U$ 的 有理映射 $g_{j}:X \dashrightarrow  Z_{j}$使得: 
          \[ \mathcal{E}_{A,\pi}(V) =\coprod_{j}\mathcal{A}_{j} \]
          其中$ \{\mathcal{A}_j=\mathcal{A}_{A,\pi,g_j}\} $ 是 $ \mathcal{E}_{A}(V) $的划分。 令 $\mathcal{C}_{j}$ 是 $\mathcal{A}_{j}$ 在$\mathcal{L}_{A,\pi}(V)$中的闭包;
    \item  对每个  $f_{i}$ 都存在某个 $g_{j}$ 和态射 $h_{ij}:Y_{i}\to Z_{j}$ 使得 $\mathcal{W}_{i} \subset \mathcal{C}_{j}$。
  \end{enumerate}
\end{theorem}
如果模型对应的多面体相邻,那么模型之间有态射:
\begin{theorem}\label{mapbetweenAM}
  \cite[Theorem 3.3]{haconSarkisovProgram2012} 令 $W$ 是光滑射影代数簇,并且$ V $ 是 $ \operatorname{WDiv}_{\mathbb{R}}(W) $的有限维有理子空间。取定一个有效丰沛$\mathbb{Q}$-除子 $A$,假设$\mathcal{L}_{A}(V)$中有元素 $D_{0}$使得 $K_{W}+D_{0}$ 具有klt奇点并且是大除子。那么存在有限多个有理压缩映射 $ f_i:W\dashrightarrow X_i $使得 
  \begin{enumerate}
    \item $ \{\mathcal{A}_i=\mathcal{A}_{A,f_i}\} $是 $ \mathcal{E}_{A}(V) $的划分。 $ \mathcal{A}_{i} $ 是有限多个有理多面体的内点的并。 令$\mathcal{C}_{i}$ 是 $\mathcal{A}_{i}$ 在 $\mathcal{L}_{A}(V)$中的闭包。如果 $ f_i $ 双有理映射,那么 $ \mathcal{C}_i$ 是有理多面体;
    \item 如果 $ i,j $ 两个指标满足 $ \mathcal{A}_j\cap \mathcal{C}_i\neq \emptyset $ ,那么存在一个压缩态射 $ f_{ij}:X_i\to X_j $使得 $ f_j=f_{ij}\circ f_i $;
    \item 进一步假设 $ V $ 张成 $W$的 Neron-Severi群,即
          \[ V \to \operatorname{NS}(W) \]
          是满射。 如果某个$ \mathcal{C}_i $的一个连通分支$ \mathcal{C} $ 与$ \mathcal{L}_A(V) $的内点有交,那么下列等价:
          \begin{enumerate}
            \item $ \mathcal{C} $ 张成 $ V $;
            \item 对除子 $ D\in \mathcal{A}_i\cap \mathcal{C} $,那么$ f_i $ 是关于$ D $的算术终端模型;
            \item $ f_i $ 是双有理映射,且 $ X_i $ 是 $ \mathbb{Q} $-分解的。
          \end{enumerate}
    \item 假设 $ V $ 张成 $W$的 Neron-Severi群。取两个指标 $ i,j $,使得 $ \mathcal{C}_i $ 张成 $ V $ 并且 $ D $ 是 $ \mathcal{A}_j\cap \mathcal{C}_i $ 的一般点,并且也是 $ \mathcal{L}_A(V) $的内点,那么 $ \mathcal{C}_i $ 在 $D$ 的一个邻域局部同构于$ \overline{\mathrm{NE}}(X_i/X_j)^*\times \mathbb{R}^k $,其中 $ k\geqslant 0 $。并且 $ \rho(X_i/X_j)=\dim  \mathcal{C}_i-\dim \mathcal{C}_j\cap \mathcal{C}_i   $。
  \end{enumerate}
\end{theorem}
对于上述定理后两条的子空间 $V$,需要张成$\operatorname{NS}(W)$,所以有$\dim V \geqslant \rho(W)$。但是通过下列引理,可以取其2维子空间,仍然满足结论;
\begin{lemma}\label{subspace}
  \cite[Corollary 3.4]{haconSarkisovProgram2012} 假设 $V$ 张成$W$的Neron-Severi群。令 $G(r, V)$为$V$ 的 $r$ 维实子空间的 Grassmannian空间,那么存在一个  Zariski稠密的开子集  $U \subset G(r, V)$,使得任何定义在有理数上的子空间  $[V']\in U$,都满足定理 \ref{mapbetweenAM}的(1)-(4)。
\end{lemma}

\begin{proof}
  令 $U \subset G(r, V) $ 是不包含任何多面体 $\mathcal{C}_{i} \subset\mathcal{L}_A(V)$的面(face)的$r$-维子空间  $V' \subsetV$的集合。特别的,  $\mathcal{L}_{A}(V')$的内部包含在 $\mathcal{L}_{A}(V)$的内部。显然任何定义在有理数上的子空间 $[V']\in U$ 都满足定理 \ref{mapbetweenAM}的 (1)-(4)。
\end{proof}
\begin{remark}
  应用上述引理,本章从此开始,总假设$V$ 是满足定理\ref{mapbetweenAM}(1)-(4)的定义在有理数上的2-维空间。此时如果多面体$\mathcal{C}$是2维的,那么就张成Neron-Severi群,进而定理\ref{mapbetweenAM}.(3).1的条件就满足。
\end{remark}


下面的引理用定理\ref{mapbetweenAM}.(2)的态射构造对应的除子压缩、小压缩或森纤维空间压缩。在一些情况下 ,两个小压缩组成了一个平转。 
\begin{lemma}\label{mapbetweenAM2}
  \cite[Lemma 3.5]{haconSarkisovProgram2012} 令 $ f:W\dashrightarrow X $和 $ g:W\dashrightarrow  Y $是两个有理压缩使得 $ \mathcal{C}_{A,f} $ 是2维空间 (张成2维线性空间) 且 $ \mathcal{O}=\mathcal{C}_{A,f}\cap \mathcal{C}_{A,g} $ 是 $ 1 $维空间。假设 $ \rho(X)\geqslant \rho(Y) $ 且 $ \mathcal{O} $ 不包含在 $ \mathcal{L}_{A}(V) $。 令 $ D $ 是 $ \mathcal{O} $ 的内点且 $ B=f_*D $。那么存在一个有理压缩态射 $ \pi:X\dashrightarrow Y $ 且 $ g=\pi\circ f $ 满足下列情况之一:
  \begin{enumerate}
    \item $ \rho(X)=\rho(Y)+1 $ 且 $ \pi  $ 是 $ (K_X+B) $-平凡映射。此时下列情况之一成立:
          \begin{enumerate}
            \item $ \pi $ 是双有理态射,并且 $ \mathcal{O} $ 不包含在 $ \mathcal{E}_A(V) $的边界里。此时有两种情况:
                  \begin{enumerate}
                    \item $ \pi $ 是除子压缩,并且 $ \mathcal{O}\neq \mathcal{C}_{A,g} $;
                    \item $ \pi $ 是小压缩,并且 $ \mathcal{O}= \mathcal{C}_{A,g} $。
                  \end{enumerate}
            \item $ \pi $ 是森纤维空间压缩,并且 $ \mathcal{O}=\mathcal{C}_{A,g} $ 包含在 $ \mathcal{E}_{A}(V) $的边界里。
          \end{enumerate}
    \item $ \rho(X)=\rho(Y) $。此时  $ \pi $ 是 $ (K_X+B) $-平转, 并且 $ \mathcal{O}\neq\mathcal{C}_{A,g} $ 不包含在 $ \mathcal{E}_A(V) $的边界里。
      \[ \xymatrix{
          X\ar[rd]_{p} \ar@{.>}[rr]^{\pi} & & Y\ar[ld]^{q} \\
        & S &
        } \]
  \end{enumerate}
\end{lemma}
\begin{remark}
  引理\ref{mapbetweenAM2}.(1)中,态射$\pi$由定理\ref{mapbetweenAM}.(2)给出。引理\ref{mapbetweenAM2}.(1).1对应定理\ref{mapbetweenAM}.(3)成立的情况,而引理\ref{mapbetweenAM2}.(1).2则对应定理\ref{mapbetweenAM}.(3)不成立的情况。

  引理\ref{mapbetweenAM2}.(2)中两个态射$p,q$由定理\ref{mapbetweenAM}.(2)给出,并且都是引理\ref{mapbetweenAM2}.(1).1.2的小双有理态射。
\end{remark}
最后给出引理来保证后续构造的Sarkisov连接中的每一个代数簇,都是$(W, B_{W})$-MMP的结果。
\begin{lemma}
  \cite[Lemma 3.6]{haconSarkisovProgram2012} 令 $f:W\dashrightarrow X $ 是 $\mathbb{Q}$-分解代数簇的双有理压缩,假设 $(W,D)$ 和 $(W,D+A)$ 都具有klt奇点。如果 $f$ 是关于 $(W,D+A)$的丰沛模型,且  $A$ 是丰沛的,那么 $f$是  $(K_{W}+D)$-MMP的结果。
\end{lemma}

\section{构造Sarkisov连接}
这一节构造Sarkisov连接。 对于某些端点 $D \in \mathcal{E}_{A}(V)$,如果$ D=A+B $ 在 $ \mathcal{E}_A(V) $的边界上,并且在 $ \mathcal{L}_A(V) $的内部,那么可以构造对应的Sarkisov连接
\[ \xymatrix{
             & Z=Z^{0}\ar[rd]\ar[ld]_{p}\ar@{.>}[rr]^{\text{关于}K+D \text{的平转}} &   & Z^{1}\ar[ld]\ar@{.>}[r] & \cdots\ar@{.>}[r] & Z^{k}=Z_{1}\ar[rd]^{q}\\
    X  &                                 & Y^{1} & & & & X_{1} } \]
其中$p,q$ 可能是同构,且$X,X_{1},Z^{i},Z,Z_{1}$都是$W$ 关于$D$ 的某种模型。 

令 $ \mathcal{T}_1, \ldots, \mathcal{T}_k $ 是 包含$D$ 的多面体 $ \mathcal{C}_i $中2维的那些。通过重新排序,可以假设 $\mathcal{O}_{0}$是$\mathcal{T}_{1}$和 $ \mathcal{E}_{A}(V) $的边界的交 ( 即 $ \mathcal{T}_{1} \cap \partial \mathcal{E}_{A}(V) = \mathcal{O}_{0}  $),类似的$\mathcal{O}_{k}$是$\mathcal{T}_{k}$和 $ \mathcal{E}_{A}(V) $的边界的交 ( 即 $ \mathcal{T}_{k} \cap \partial \mathcal{E}_{A}(V) = \mathcal{O}_{k}  $),并且 $ \mathcal{O}_i=\mathcal{T}_i\cap\mathcal{T}_{i+1} $ 是1维的。如下图所示:
\begin{center}
  \begin{tikzpicture}
    \draw (0,0)node[below]{$ D $}--(-6,4)node[above]{$ \mathcal{O}_0 $};
    \draw (-2.5,2)node[right]{$ \mathcal{T}_1 $};
    \draw (0,0)--(-1.5,2)node[right]{$ \mathcal{T}_{2} $}--(-3,4)node[above]{$ \mathcal{O}_{1} $};
    \draw (0,0)--(-0.5,2)node[right]{$ \cdots $}--(-1,4);
    \draw (0,0)--(0.5,2)node[right]{$ \mathcal{T}_{k-1} $}--(1,4);
    \draw (0,0)--(3,4)node[above]{$ \mathcal{O}_{k-1} $};
    \draw (2,2)node[right]{$ \mathcal{T}_{k} $};
    \draw (0,0)--(6,4)node[above]{$ \mathcal{O}_{k} $};
  \end{tikzpicture}
\end{center}
令 $ f_i:W\dashrightarrow  X_i $ 是 $ \mathcal{T}_i $ 对应的双有理映射,且$ g_i:W\dashrightarrow  S_i $ 是$ \mathcal{O}_i $对应的有理映射。

记 
\[ f=f_1:W\dashrightarrow X, g=f_k:W\dashrightarrow Y \]
以及 
\[ \phi:X\to S=S_0,\psi:Y\to T=S_k \]
和 $ X'=X_2,Y'=X_{k-1}$。 令 $ W\dashrightarrow R $ 为 关于$ D $的丰沛模型。延续这些记号,构造Sarkisov连接:
\begin{theorem}\label{constructlink}
  \cite[Theorem 3.7]{haconSarkisovProgram2012} 取边界 $ B_W $ 使得$ (W,B_W) $ 是算术光滑的算术终端奇点的代数簇对,且 $ D-B_W $ 是丰沛除子。那么 $ \phi $和 $ \psi $ 是森纤维空间,并且是 $ (K_W+B_W) $-MMP的输出。假设 $ D $ 包含在2个以上的多面体中,那么 $\phi$ 和 $\psi$ 由 Sarkisov连接相连,并且每个 $f_{i}:W \dashrightarrow  X_{i}$是  $(K_{W}+B_{W})$-MMP的结果。
\end{theorem}
\begin{proof}
  假设 $ k\geqslant 3 $,那么有
  $$ \xymatrix{
      X'\ar@{.>}[d]_p\ar@{.>}[rr]&&Y'\ar@{.>}[d]^q\\
      X\ar[d]_{\phi}&&Y\ar[d]^\psi\\
      S\ar[rd]^s&&T\ar[ld]_t\\
      &R&} $$
  注意到 $ \rho(X_i/R)\leqslant 2 $ 且 $ \rho(X/S)=\rho(Y/T)=1 $,那么下列两种情况之一成立:
  \begin{enumerate}
    \item $ s $ 是一致态射 (identity)而 $ p $ 是除子压缩 (除子解压);
    \item $ s $ 是压缩态射而 $ p $ 是平转。
  \end{enumerate}
  同样的情况对 $ q $ 和 $ t $成立。而映射 $X'\dashrightarrow Y'$ 是 $(K+D)$-平转。这样给出了四种Sarkisov连接。
\end{proof}

\section{构造Sarkisov分解}
首先需要如下特别的解消$W$和仿射子空间$V \subset \operatorname{WDiv}(W)$:

\begin{lemma}\label{keylemma}
  \cite[Lemma 4.1]{haconSarkisovProgram2012} 令 $ \phi: (X,B_{X}) \to S $ 和 $ \psi: (Y,B_{Y})\to T  $ 是两个具有klt奇点的 MMP-相关森纤维空间。那么存在光滑射影代数簇 $ W $和两个双有理态射 $ f:W\to X $  和 $ g:W\to Y $,并且有边界除子$(B_{W})$是 $ (W,B_{W}) $具有klt奇点。同时还存在$W$ 上的丰沛 $ \mathbb{Q} $-除子 $ A $ 和2维有理仿射子空间 $ V \subset \mathrm{WDiv}_{\mathbb{R}}(W) $使得:
  \begin{enumerate}
    \item 如果 $ D\in \mathcal{L}_A(V) $, 那么 $ D-B_W $ 是丰沛除子;
    \item $ \mathcal{A}_{A,\phi\circ f} $ 且 $ \mathcal{A}_{A,\psi\circ g} $ 不包含在 $ \mathcal{L}_A(V) $的边界里;
    \item $ V $ 满足定理\ref{mapbetweenAM}.(1-4);
    \item $ \mathcal{C}_{A,f} $ 和 $ \mathcal{C}_{A,g} $是2维的; 
    \item $ \mathcal{C}_{A,\phi\circ f} $ 和 $ \mathcal{C}_{A,\psi\circ g} $是1维的。 
  \end{enumerate}
\end{lemma}

\begin{proof}
  由MMP-相关的假设,存在 $\mathbb{Q}$-faci分解的具有klt奇点的 代数簇对$(W,B_{W})$ 使得 $f:W\dashrightarrow X$ 和 $g:W \dashrightarrow Y$是 $(K_{W}+B_{W})$-MMP的输出。令 $p':W'\to W$ 是解消 $f$ 和 $g$的解消,记
  \[ K_{W'}+B_{W'}=p'^*(K_{W}+B_{W})+E' \]
  其中  $p'_*B_{W'}=B_{W}$,且$E'\geqslant 0$ 和 $B_{W'}\geqslant 0$ 没有公共组成成分, $E'$是例外除子。取 除子 $-F$满足:
\begin{itemize}
  \item 相对于$W$ 丰沛的;
  \item $\operatorname{Supp}F=\operatorname{Exc}p'$;
  \item  $K_{W'}+B_{W'}+F$ 具有klt奇点。 
\end{itemize}
  由于 $p'$ 是 $(K_{W'}+ B_{W'}+F)$-负性的,$(K_{W}+B_{W})$具有 klt奇点, $W$ 是 $\mathbb{Q}$-分解的,所以相对于$W$ 的 $(K_{W'}+B_{W'}+F)$-MMP终结于 $(W,B_{W})$。 用 $(W',B_{W'} +F)$替换 $(W,B_{W})$,可以假设 $(W,B_{W})$ 是算术光滑并且 $f,g$是双有理态射。 

  取$W$ 上 一般的 丰沛$\mathbb{Q}$-除子 $A, H_{1},H_{2},\ldots ,H_{k}$ on $W$ 使得 $H_{1},\ldots , H_{k}$ 生成 $W$ 的 Neron-Severi 群。 记 $H=A+H_{1}+\ldots+ H_{k}$。取$S$ 和$T$ 上充分丰沛的除子 $A_{S}$ 和 $A_{T}$使得 
  \[ -(K_{X}+B_{X})+\phi^*A_{S} \]
  和
  \[ -(K_{Y}+B_{Y})+\psi^*A_{T} \]
  都是丰沛除子。取有理数 $0<\delta<1$使得 
  \[ -(K_{X}+B_{X}+\delta f_*H)+\phi^*A_{S} \]
  和
  \[-(K_{Y}+B_{Y}+\delta g_*H)+\psi^*A_{T} \]
  都是丰沛除子,并且  $f$ 和  $g$ 都是  $(K_{W}+B_{W}+\delta H)$-负性的。通过用 $\delta H$ 替换$H$可以假设 $\delta=1$。接着取 $\mathbb{Q}$-除子 $B_{0}\leqslant B_{W}$ 使得 $A+(B_{0}-B_{W}), -(K_{X}+ f_*B_{0}+f_*H)+\phi^*A_{S}$ 和 $-(K_{Y}+ g_*B_{0}+f_*H)+\psi^*A_{T}$ 都是丰沛除子,并且 $f$ 和  $g$都是  $(K_{W}+B_{0}+\delta H)$-负性的。

  取一般的丰沛 $\mathbb{Q}$-除子 $F_{1}\geqslant 0$ 和 $G_{1}\geqslant 0$ 使得 
    \[ F_{1}\sim_{\mathbb{Q}} -(K_{X}+f_*B_{0}+ f_*H)+\phi^*A_{S} \]
  且
    \[G_{1}\sim_{\mathbb{Q}} -(K_{Y}+g_*B_{0}+ g_*H)+\psi^*A_{T} \]
  并且
  \[ K_{W}+B_{0}+H+F+G \]
  具有klt奇点,其中 $F=f^*F_{1},G=g^*G_{1}$.
  考虑由 $H_{1},\ldots , H_{k},F,G$张成的线性子空间,再被$B_{0}$平移,得到$\operatorname{WDiv}_{\mathbb{R}}(W)$的仿射子空间 $V_{0}$ (即 $V_{0}=B_{0}+ \langle H_{1},\ldots ,H_{k},F,G \rangle  $)。假设 $D=A+B \in \mathcal{L}_{A}(V_{0})$,那么
  \[ D-B_W=(A+B_{0}-B_{W})+(B-B_{0}) \]
  是丰沛除子 (由于 $B-B_{0}$是数值有效的,由$V_{0}$的定义可以得到)。注意到
  \[ B_{0}+F+H \in \mathcal{A}_{A,\phi\circ f}(V_{0}), B_{0}+G+H \in \mathcal{A}_{\psi \circ g}(V_{0}) \]
  且 $f$和 $g$分别是关于  $K_{W}+B_{0}+F+H$和 $K_{W}+B_{0}+G+H$的弱算术典范模型。因此由定理\ref{mapbetweenAM} 推出 $V_{0}$满足定理\ref{mapbetweenAM}.(1)-(4)。

  由于 $H_{1},\ldots ,H_{k}$ 生成$W$的 Neron-Severi群,可以可以找到常熟$h_{1},\ldots ,h_{k}$ 使得 $G \equiv \sum^{k}_{i=1} h_{i}H_{i}$。那么存在 $0< \delta\ll 1$使得 
  \[B_{0}+F+\delta G+H- \delta(\sum_{i=1}^{k} h_{i}H_{i}) \in \mathcal{L}_{A}(V_{0})\]
  且
  \[ B_{0}+F+\delta G+H-\delta (\sum_i^k h_{i}H_{i}) \equiv B_{0}+F+H \]
  因此, $\mathcal{A}_{A,\phi\circ f}$ 不含在 of $\mathcal{L}_{A}(V_{0})$的边界中。类似的, $\mathcal{A}_{A,\psi\circ g}$ 不包含在 $\mathcal{L}_{A}(V_{0})$的边界中。特别的,由于$\rho(X/S)=\rho(Y/T)=1$,所以$\mathcal{A}_{A,\phi\circ f}$ 和   $\mathcal{A}_{A,\psi\circ g}$ 张成 $V_{0}$中的仿射平面。

  考虑 $F+H-A$ 和 $G+H-A$张成的空间,再被 $B_{0}$平移,得到仿射空间 $V_{1}$。由引理\ref{subspace},将 $V_{1}$扰动得到定义在有理数上的仿射空间 $V$,这个空间即满足此引理。
\end{proof}

最后证明Sarkisov分解的存在性:
\begin{proof}
  令 $(W,B_{W}),A $ 和 $V$ 满足引理\ref{keylemma}。 取$ \mathcal{L}_A(V) $中的内点$ D_{0} \in \mathcal{A}_{A,\phi\circ f} $  和 $ D_1\in \mathcal{C}_{A,g} $。由于 $ V $是2维的,去掉 $ D_0 $ 和 $ D_1 $ 将把 $ \mathcal{E}_A(V) $的边界分成两个部分。其中一部分全部由不大的(not big)除子组成,并且包含在 $ \mathcal{L}_A(V) $的内部。考虑这一部分从 $ D_0 $ 到 $ D_1 $的过程,存在有点多个点 $ D_i , 2\leqslant i\leqslant N $ 包含在两个以上的多面体 $ \mathcal{C}_{A,f_{k}}(V) $中。由引理 \ref{constructlink},每一个点 $ D_i $给出一个 Sarkisov 连接。那么双有理映射 $X \dashrightarrow Y$ 分解成这些连接的复合。
\end{proof}


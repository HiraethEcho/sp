\chapter{有限模型法}\label{thirdmethod}
这一章主要参考 \cite{haconSarkisovProgram2012}。 通过有限模型来构造Sarkisov分解和前两种方法不同,这种方法并不通过运行2-ray game来构造Sarkisov连接。下面简要介绍这种方法的思路:

令 $W$为两个MMP-相关的森纤维空间 $X\to S$ 和 $Y\to T$公共奇点解消。 Take a finite dimensional affine subspace $V$ 取 $\operatorname{WDiv}_{\mathbb{R}}(W)$的有限维仿射子空间$V$和丰沛 $\mathbb{Q}$-除子 $A$,那么 $\{\mathcal{A}_{i} =\mathcal{A}_{A,f_{i}}\} $ 是 $\mathcal{E}_{A}(V)$的一个划分,且每个子集 $\mathcal{A}_{i}$ 对应 $W$的一个丰沛模型,相邻的子集对应的丰沛模型之间有压缩态射 (见 定理\ref{mapbetweenAM})。由引理\ref{keylemma},可以取到一个特殊的子空间 $V$ ,使得 
\begin{enumerate}
  \item $V$ 是2维的有理仿射空间(可以定义在有理数上); 
  \item $S,T$ 是 $W$ 关于 $D_{S},D_{T} \in \mathcal{L}_{A}(V) $的丰沛模型;
  \item $D_{S}$ 和 $D_{T}$ 是$\mathcal{L}_{A}(V)$边界 $\partial \mathcal{L}_{A}(V)$上的两个点,并且将边界分为两个部分。在其中一个部分上,有有限多个线段连接 $D_{S}$ 和 $D_{T}$,并且$D_{i}$是线段的端点,此时每个端点 $ D_{i}$对应一个Sarkisov连接。

    (图 \ref{pic}为一个示例图,其中$D_S=D_0,D_T=D_1$)
  \item 双有理映射$X\dashrightarrow Y$是这些Sarkisov 连接的复合。这就是有限模型法构造的Sarkisov分解。
\end{enumerate}
\begin{remark}
  有限模型法的Sarkisov纲领严格来说并不是一个纲领 (program),和前两种方法不同,不是归纳地构造分解中的森纤维空间$X_{i}\to S_{i}$,而是先找出$W$ 关于$D \in \mathcal{L}_{A}(V)$的丰沛模型,然后选择其中的一些。
\end{remark}

\section{构造Sarkisov连接}
这一节构造端点 $D$ 对应的Sarkisov连接。构造的任何一种Sarkisov连接
\[
  \xymatrix{
             & Z=Z^{0}\ar[rd]\ar[ld]_{p}\ar@{.>}[rr]^{\text{关于}K+D \text{的平转}} &   & Z^{1}\ar[ld]\ar@{.>}[r] & \cdots\ar@{.>}[r] & Z^{k}=Z_{1}\ar[rd]^{q}\\
    X\ar[d]  &                                 & Y^{1} & & & & X_{1}
  }
\]
其中$p,q$ 可能是同构,且$X,X_{1},Z^{i},Z,Z_{1}$都是$W$ 关于$D$ 的某种模型。 

首先给出 $\mathcal{E}_{A}(V)$ 关于丰沛模型的划分定理:
\begin{theorem}\label{finitemodel}
  \cite[Corollary 1.1.5]{BCHM10} 令 $\pi:X\to U$ 是正规拟射影代数簇间的射影态射,并且 $V \subset \operatorname{WDiv}_{\mathbb{R}}(X)$是有限维有理子空间,假设存在一个除子 $\Delta_{0} \in V$ 使得 $(X,\Delta_{0})$ 具有klt奇点。令 $A$是 相对于 $U$ 的一般的丰沛 $\mathbb{Q}$-除子,并且它和任何$V$ 中的除子都没有共同的素除子作为组成成分。那么
  \begin{enumerate}
    \item   存在有限多个相对于 $U$ 的双有理映射  $f_{i}:X \dashrightarrow   X_{i}$ 使得:
          \[ \mathcal{E}_{A,\pi}(V) =\bigcup_{i}\mathcal{W}_{i} \]
          其中  $\mathcal{W}_{i}=\mathcal{W}_{A,f_{i}}(V)$ 是定义在有理数上的多面体  (rational polytope)。且如果  $f:X \dashrightarrow  Y$ 是关于对某个 $D \in \mathcal{E}_{A,\pi}(V)$相对于 $U$ 的算术终端模型,那么对某个$i$ ,有  $f=f_{i}$ 。即 $X$ 关于$\mathcal{E}_{A,\pi}(V)$的算术终端模型一定是$f_{i}$中的一个。   
    \item   存在有限多个相对于$U$ 的 有理映射 $g_{j}:X \dashrightarrow  Z_{j}$使得: 
          \[ \mathcal{E}_{A,\pi}(V) =\coprod_{j}\mathcal{A}_{j} \]
          其中$ \{\mathcal{A}_j=\mathcal{A}_{A,\pi,g_j}\} $ 是 $ \mathcal{E}_{A}(V) $的划分。 令 $\mathcal{C}_{j}$ 是 $\mathcal{A}_{j}$ 在$\mathcal{L}_{A,\pi}(V)$中的闭包;
    \item  对每个  $f_{i}$ 都存在某个 $g_{j}$ 和态射 $h_{ij}:Y_{i}\to Z_{j}$ 使得 $\mathcal{W}_{i} \subset \mathcal{C}_{j}$。
  \end{enumerate}
\end{theorem}
如果模型对应的多面体相邻,那么模型之间有态射:
\begin{theorem}\label{mapbetweenAM}
  \cite[Theorem 3.3]{haconSarkisovProgram2012} 令 $W$ 是光滑射影代数簇,并且$ V $ 是 $ \operatorname{WDiv}_{\mathbb{R}}(W) $的有限维有理子空间。取定一个有效丰沛$\mathbb{Q}$-除子 $A$,假设$\mathcal{L}_{A}(V)$中有元素 $D_{0}$使得 $K_{W}+D_{0}$ 具有klt奇点并且是大除子。那么存在有限多个有理压缩映射 $ f_i:W\dashrightarrow X_i $使得 
  \begin{enumerate}
    \item $ \{\mathcal{A}_i=\mathcal{A}_{A,f_i}\} $是 $ \mathcal{E}_{A}(V) $的划分。 $ \mathcal{A}_{i} $ 是有限多个有理多面体的内点的并。 令$\mathcal{C}_{i}$ 是 $\mathcal{A}_{i}$ 在 $\mathcal{L}_{A}(V)$中的闭包。如果 $ f_i $ 双有理映射,那么 $ \mathcal{C}_i$ 是有理多面体;
    \item 如果 $ i,j $ 两个指标满足 $ \mathcal{A}_j\cap \mathcal{C}_i\neq \emptyset $ ,那么存在一个压缩态射 $ f_{ij}:X_i\to X_j $使得 $ f_j=f_{ij}\circ f_i $;
    \item 进一步假设 $ V $ 张成 $W$的 Neron-Severi群,即
          \[ V \to \operatorname{NS}(W) \]
          是满射。 如果某个$ \mathcal{C}_i $的一个连通分支$ \mathcal{C} $ 与$ \mathcal{L}_A(V) $的内点有交,那么下列等价:
          \begin{enumerate}
            \item $ \mathcal{C} $ 张成 $ V $;
            \item 对除子 $ D\in \mathcal{A}_i\cap \mathcal{C} $,那么$ f_i $ 是关于$ D $的算术终端模型;
            \item $ f_i $ 是双有理映射,且 $ X_i $ 是 $ \mathbb{Q} $-分解的。
          \end{enumerate}
    \item 假设 $ V $ 张成 $W$的 Neron-Severi群。取两个指标 $ i,j $,使得 $ \mathcal{C}_i $ 张成 $ V $ 并且 $ D $ 是 $ \mathcal{A}_j\cap \mathcal{C}_i $ 的一般点,并且也是 $ \mathcal{L}_A(V) $的内点,那么 $ \mathcal{C}_i $ 在 $D$ 的一个邻域局部同构于$ \overline{\mathrm{NE}}(X_i/X_j)^*\times \mathbb{R}^k $,其中 $ k\geqslant 0 $。并且 $ \rho(X_i/X_j)=\dim  \mathcal{C}_i-\dim \mathcal{C}_j\cap \mathcal{C}_i   $。
  \end{enumerate}
\end{theorem}

\begin{lemma}\label{subspace}
  \cite[Corollary 3.4]{haconSarkisovProgram2012} If $V$ spans the Neron-Severi group of $W$, then there is a Zariski dense open subset $U$ of the Grassmannian $G(r, V)$ of real affine subspaces of dimension $r$ such that any  $[V']\in U$ defined over the rational numbers satisfies (1-4) of Theorem \ref{mapbetweenAM}.
\end{lemma}

\begin{proof}
  Let $U \subset G(r, V) $ be the set of real affine subspace  $V'$ of $V$ of dimension $r$, which contains no face of any $\mathcal{C}_{i}$ of $\mathcal{L}_A(V)$. In particular, the interior of  $\mathcal{L}_{A}(V')$ is contained in the interior of $\mathcal{L}_{A}(V)$. It is clear that any $V'\in U$ defined over the rationals satisfies (1-4) of Theorem \ref{mapbetweenAM}.
\end{proof}

By the above Lemma, from now on in this section, we always assume that $V$ has dimension $2$ and satisfies (1-4) of Theorem \ref{mapbetweenAM}. The following lemma classifies the morphisms in (2) of Theorem \ref{mapbetweenAM} into a divisorial contraction, a small contraction or a log Mori fibre space. In some cases (Lemma \ref{mapbetweenAM2} (2)), two small contractions form a flop.

\begin{lemma}\label{mapbetweenAM2}
  \cite[Lemma 3.5]{haconSarkisovProgram2012} Let $ f:W\dashrightarrow X $ and $ g:W\dashrightarrow  Y $ be two rational contractions such that $ \mathcal{C}_{A,f} $ is of dimension $ 2 $ and $ \mathcal{O}=\mathcal{C}_{A,f}\cap \mathcal{C}_{A,g} $ is of dimension $ 1 $. Assume $ \rho(X)\geqslant \rho(Y) $ and $ \mathcal{O} $ is not contained in the boundary of $ \mathcal{L}_{A}(V) $. Let $ D $ be an interior point of $ \mathcal{O} $ and $ B=f_*D $. Then there is a rational contraction $ \pi:X\dashrightarrow Y $ and $ g=\pi\circ f $ such that either
  \begin{enumerate}
    \item $ \rho(X)=\rho(Y)+1 $ and $ \pi  $ is $ (K_X+B) $-trivial, and either
          \begin{enumerate}
            \item $ \pi $ is birational and $ \mathcal{O} $ is not contained in the boundary of $ \mathcal{E}_A(V) $, and either
                  \begin{enumerate}
                    \item $ \pi $ is a divisorial contraction and $ \mathcal{O}\neq \mathcal{C}_{A,g} $, or
                    \item $ \pi $ is a small contraction and $ \mathcal{O}= \mathcal{C}_{A,g} $, or
                  \end{enumerate}
            \item $ \pi $ is a log Mori fibre space, and $ \mathcal{O}=\mathcal{C}_{A,g} $ is contained in the boundary of $ \mathcal{E}_{A}(V) $, or
          \end{enumerate}
    \item $ \rho(X)=\rho(Y) $, and $ \pi $ is  a $ (K_X+B) $-flop and $ \mathcal{O}\neq\mathcal{C}_{A,g} $ is not contained in the boundary of $ \mathcal{E}_A(V) $.
  \end{enumerate}
\end{lemma}

\begin{lemma}
  \cite[Lemma 3.6]{haconSarkisovProgram2012} Let $f:W\dashrightarrow X $ be a birational contraction between $\mathbb{Q}$-factorial varieties. Suppose $(W,D)$ and $(W,D+A)$ are both klt. If $f$ is the ample model of $(W,D+A)$ and $A$ is ample, then $f$ is a result of the  $(K_{W}+D)$-MMP.
\end{lemma}

This lemma guarantees that every variety in the Sarkisov links constructed later is a result of the $(W, B_{W})$-MMP.

Finally, we show that there is a Sarkisov link corresponding to certain $D \in \mathcal{E}_{A}(V)$. Let $ D=A+B $ be a point of the boundary of $ \mathcal{E}_A(V) $ in the interior of $ \mathcal{L}_A(V) $. Let $ \mathcal{T}_1, \ldots, \mathcal{T}_k $ be the polytopes $ \mathcal{C}_i $ of dimension $ 2 $ containing $ D $. Possibly re-ordering, we may assume that  the intersection  $ \mathcal{O}_0 $ and $ \mathcal{O}_k $ of $ \mathcal{T}_1 $ and $ \mathcal{T}_k $ with boundary of $ \mathcal{E}_A(V) $ and  $ \mathcal{O}_i=\mathcal{T}_i\cap\mathcal{T}_{i+1} $ are one dimensional. Let $ f_i:W\dashrightarrow  X_i $ be the birational contraction associated to $ \mathcal{T}_i $ and $ g_i:W\dashrightarrow  S_i $ be the rational contraction associated to $ \mathcal{O}_i $.
\begin{center}
  \begin{tikzpicture}
    \draw (0,0)node[below]{$ D $}--(-6,4)node[above]{$ \mathcal{O}_0 $};
    \draw (-2.5,2)node[right]{$ \mathcal{T}_1 $};
    \draw (0,0)--(-1.5,2)node[right]{$ \mathcal{T}_{2} $}--(-3,4)node[above]{$ \mathcal{O}_{1} $};
    \draw (0,0)--(-0.5,2)node[right]{$ \cdots $}--(-1,4);
    \draw (0,0)--(0.5,2)node[right]{$ \mathcal{T}_{k-1} $}--(1,4);
    \draw (0,0)--(3,4)node[above]{$ \mathcal{O}_{k-1} $};
    \draw (2,2)node[right]{$ \mathcal{T}_{k} $};
    \draw (0,0)--(6,4)node[above]{$ \mathcal{O}_{k} $};
  \end{tikzpicture}
\end{center}

Set $ f=f_1:W\dashrightarrow X, g=f_k:W\dashrightarrow Y $ and $ \phi:X\to S=S_0,\psi:Y\to T=S_k $ and $ X'=X_2,Y'=X_{k-1} $ and let $ W\dashrightarrow R $ be the ample model of $ D $. Then
\begin{theorem}\label{constructlink}
  \cite[Theorem 3.7]{haconSarkisovProgram2012} Suppose $ B_W $ is any divisor such that $ (W,B_W) $ is a log smooth terminal pair and $ D-B_W $ is ample. Then $ \phi $ and $ \psi $ are log Mori fibre spaces, which are outputs of the $ (K_W+B_W) $-MMP. Moreover, $ D $ is contained in more than two polytopes, then $\phi$ and $\psi$ are connected by a Sarkisov link, where each $f_{i}$ is a result of running the $(K_{W}+B_{W})$-MMP.
\end{theorem}
\begin{proof}
  We may assume $ k\geqslant 3 $, and we have
  $$ \xymatrix{
      X'\ar@{.>}[d]_p\ar@{.>}[rr]&&Y'\ar@{.>}[d]^q\\
      X\ar[d]_{\phi}&&Y\ar[d]^\psi\\
      S\ar[rd]^s&&T\ar[ld]_t\\
      &R&} $$
  Note that $ \rho(X_i/R)\leqslant 2 $ and $ \rho(X/S)=\rho(Y/T)=1 $. Thus,
  \begin{enumerate}
    \item $ s $ is the identity and $ p $ is a divisorial contraction (extraction), or
    \item $ s $ is a contraction and $ p $ is a flop.
  \end{enumerate}
  The same holds for $ q $ and $ t $. The map $X'\dashrightarrow Y'$ is the composition of the flops. This gives 4 types of links.
\end{proof}

\section{Decomposition into Sarkisov links}
We need a special resolution $W$ and a special affine subspace $V \subset \operatorname{WDiv}(W)$ as follows.

\begin{lemma}\label{keylemma}
  \cite[Lemma 4.1]{haconSarkisovProgram2012} Let $ \phi: X \to S $ and $ \psi: Y\to T  $ be two MMP-related log Mori fibre spaces corresponding to two klt projective varieties $ (X, B_X) $ and $ (Y, B_Y) $. Then we may find a smooth projective variety $ W $, two birational morphisms $ f:W\to X $ and $ g:W\to Y $, a klt pair $ (W,B_{W}) $, an ample $ \mathbb{Q} $-divisor $ A $ on $ W $ and a two-dimensional rational affine subspace $ V $ of $ \mathrm{WDiv}_{\mathbb{R}}(W) $ such that
  \begin{enumerate}
    \item If $ D\in \mathcal{L}_A(V) $ then $ D-B_W $ is ample;
    \item $ \mathcal{A}_{A,\phi\circ f} $ and $ \mathcal{A}_{A,\psi\circ g} $ are not contained in the boundary of $ \mathcal{L}_A(V) $;
    \item $ V $ satisfies (1-4) of Theorem \ref{mapbetweenAM};
    \item $ \mathcal{C}_{A,f} $ and $ \mathcal{C}_{A,g} $ are two-dimensional;
    \item $ \mathcal{C}_{A,\phi\circ f} $ and $ \mathcal{C}_{A,\psi\circ g} $ are one dimensional.
  \end{enumerate}
\end{lemma}

\begin{proof}
  By assumption there is a $\mathbb{Q}$-factorial klt pair $(W,B_{W})$ such that $f:W\dashrightarrow X$ and $g:W \dashrightarrow Y$ are the outputs of the $(K_{W}+B_{W})$-MMP. Let $p':W'\to W$ be any log resolution  that resolves the indeterminacy of $f$ and $g$, then we may write
  \[
    K_{W'}+B_{W'}=p'^*(K_{W}+B_{W})+E'
  \]
  where $E'\geqslant 0$ and $B_{W'}\geqslant 0$ have no common components, and $E'$ is exceptional and $p'_*B_{W'}=B_{W}$. Pick a divisor $-F$ which is ample over $W$ with $\operatorname{Supp}F=\operatorname{Exc}p'$ such that $K_{W'}+B_{W'}+F$ is klt. As $p'$ is $(K_{W'}+
    B_{W'}+F)$-negative and $(K_{W}+B_{W})$ is klt and $W$ is $\mathbb{Q}$-factorial, the $(K_{W'}+B_{W'}+F)$-MMP over $W$ terminates with the pair $(W,B_{W})$. Replacing $(W,B_{W})$ by $(W',B_{W'} +F)$ we may assume that $(W,B_{W})$ is log smooth and $f,g$ are morphisms.

  Pick general ample $\mathbb{Q}$-divisors $A, H_{1},H_{2},\ldots ,H_{k}$ on $W$ such that $H_{1},\ldots , H_{k}$ generate the Neron-Severi group of $W$. Let $H=A+H_{1}+\ldots+ H_{k}$. Pick sufficiently ample divisors $A_{S}$ on $S$ and $A_{T}$ on $T$ such that
  \[
    -(K_{X}+B_{X})+\phi^*A_{S} \text{ and } -(K_{Y}+B_{Y})+\psi^*A_{T}
  \]
  are both ample. Pick a rational number $0<\delta<1$ such that
  \[
    -(K_{X}+B_{X}+\delta f_*H)+\phi^*A_{S} \text{ and } -(K_{Y}+B_{Y}+\delta g_*H)+\psi^*A_{T}
  \]
  are both ample and  $f$ and  $g$ are both  $(K_{W}+B_{W}+\delta H)$-negative. Replacing $H$ by $\delta H$ we may assume that $\delta=1$. Now pick a $\mathbb{Q}$-divisor $B_{0}\leqslant B_{W}$ such that $A+(B_{0}-B_{W}), -(K_{X}+ f_*B_{0}+f_*H)+\phi^*A_{S}$ and $-(K_{Y}+ g_*B_{0}+f_*H)+\psi^*A_{T}$  are all ample and $f$ and  $g$ are both  $(K_{W}+B_{0}+\delta H)$-negative.

  Pick general ample $\mathbb{Q}$-divisors $F_{1}\geqslant 0$ and $G_{1}\geqslant 0$  such that
  \[
    F_{1}\sim_{\mathbb{Q}} -(K_{X}+f_*B_{0}+ f_*H)+\phi^*A_{S} \text{ and } G_{1}\sim_{\mathbb{Q}} -(K_{Y}+g_*B_{0}+ g_*H)+\psi^*A_{T}
  \]
  and
  \[
    K_{W}+B_{0}+H+F+G
  \]
  is klt, where $F=f^*F_{1}$ and $G=g^*G_{1}$.
  Let $V_{0}$ be the affine subspace of $\operatorname{WDiv}_{\mathbb{R}}(W)$ which is the translation by $B_{0}$ of the vector subspace  spanned by $H_{1},\ldots , H_{k},F,G$. Suppose that $D=A+B \in \mathcal{L}_{A}(V_{0})$. Then
  \[
    D-B_W=(A+B_{0}-B_{W})+(B-B_{0})
  \]
  is ample, as $B-B_{0}$ is nef by definition of $V_{0}$. Note that
  \[
    B_{0}+F+H \in \mathcal{A}_{A,\phi\circ f}(V_{0}), B_{0}+G+H \in \mathcal{A}_{\psi \circ g}(V_{0})
  \]
  and $f$, respectively $g$, is a weak log canonical model of $K_{W}+B_{0}+F+H$, respectively $K_{W}+B_{0}+G+H$. Thus, Theorem \ref{mapbetweenAM} implies that $V_{0}$ satisfies (1-4) of Theorem \ref{mapbetweenAM}.

  Since $H_{1},\ldots ,H_{k}$ generated the Neron-Severi group of $W$ we may find constants $h_{1},\ldots ,h_{k}$ such that $G \equiv \sum^{k}_{i=1} h_{i}H_{i}$. Then there is $0< \delta\ll 1$ such that  $B_{0}+F+\delta G+H- \delta(\sum_{i=1}^{k} h_{i}H_{i}) \in \mathcal{L}_{A}(V_{0})$ and
  \[
    B_{0}+F+\delta G+H-\delta (\sum_i^k h_{i}H_{i}) \equiv B_{0}+F+H
    .\]
  Thus, $\mathcal{A}_{A,\phi\circ f}$ is not contained in the boundary of $\mathcal{L}_{A}(V_{0})$. Similarly, $\mathcal{A}_{A,\psi\circ g}$ is not contained in the boundary of $\mathcal{L}_{A}(V_{0})$. In particular $\mathcal{A}_{A,\phi\circ f}$ and   $\mathcal{A}_{A,\psi\circ g}$ span affine hyperplanes of $V_{0}$, since $\rho(X/S)=\rho(Y/T)=1$.

  Let $V_{1}$ be the translation by $B_{0}$ of the two-dimensional vector space spanned by $F+H-A$ and $G+H-A$. Let $V$ be a small general perturbation of $V_{1}$ as in Lemma \ref{subspace}, which is defined over the rationals. This is the affine subspace we need.
\end{proof}
Then we can prove the main theorem

\begin{proof}[Proof of Theorem \ref{main}]
  Let $(W,B_{W}),A $ and $V$ as in the Lemma \ref{keylemma}.  Pick $ D_{0} \in \mathcal{A}_{A,\phi\circ f} $  and $ D_1\in \mathcal{C}_{A,g} $ belonging to the interior of $ \mathcal{L}_A(V) $. As $ V $ is two-dimensional, removing $ D_0 $ and $ D_1 $ divides the boundary of $ \mathcal{E}_A(V) $ into two parts. The part which consists entirely of divisors that are not big is contained in the interior of $ \mathcal{L}_A(V) $. Consider tracing this boundary from $ D_0 $ to $ D_1 $. Then there are finitely many $ 2\leqslant i\leqslant N $ points $ D_i $ which are contained in more than two polytopes $ \mathcal{C}_{A,f_i}(V) $. By Lemma \ref{constructlink},  each point $ D_i $ gives a Sarkisov link.  The birational map $X \dashrightarrow Y$ is the composition of such links.
\end{proof}


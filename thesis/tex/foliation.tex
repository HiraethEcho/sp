\chapter{叶层化Sarkisov纲领}
用子层$\mathcal{F} \subset \mathcal{T}_{X}$ 代替切丛,进而用叶层化典范除子 (foliated canonical divisor) $K_{\mathcal{F}}$代替典范除子$K_{X}$,并且有对应的叶层化极小模型纲领,在高维情况下尤其对代数可积的叶层化代数簇对有一系列结果。本章尝试分析这种情况下的Sarkisov纲领。
\section{定义和引理}
首先给出\textbf{叶层化代数簇对} (或 \textbf{带叶状结构的代数簇对})的基本知识。

\begin{definition}\cite[Definition 2.3-6]{acc_foliation}
  \begin{enumerate}
    \item 令 $X$ 为正规代数簇。$X$ 上的一个叶状结构 (folation)是一个凝聚层 $\mathcal{F} \subset \mathcal{T}_{X}$,满足
          \begin{enumerate}
            \item $\mathcal{T}_{X}/\mathcal{F}$是无扭的 (torsion free);
            \item $\mathcal{F}$在李括号下封闭。 
          \end{enumerate}
          如果$\mathcal{F}=0$,称之为由点定义的叶状结构 (此时此叶状结构的叶子 (leaves)是单点)。
    \item 对$X$ 上素除子 $D$,若$D$ 不是 $\mathcal{F}$-不变的,则定义$\epsilon_{\mathcal{F}}(D)=1 $;反之,如果$D$ 是$\mathcal{F}$-不变的,则定义$\epsilon_{\mathcal{F}}(D)=0$。
    \item 如果除子$K_{\mathcal{F}}$满足$\det \mathcal{F} = \mathcal{O}_{X}(-K_{\mathcal{F}})$,则称之为$\mathcal{F}$的典范除子。
    \item 令$f:X \dashrightarrow Y$为支配有理映射,将$X$ 上的叶状结构$\mathcal{F}$的在$f$ 的拉回记作  $ f^{-1}(F) $ (叶状结构的拉回的定义,参考\cite[3.1]{acss} 或\cite[3.2]{cs21}),并且称其为$\mathcal{F}$诱导的 $\mathcal{Y}$ 上的叶状结构。令$g:X \dashrightarrow X' $为双有理映射,则将$(g^{-1})^{-1}(\mathcal{F})$记为 $g_{*}\mathcal{F}$,并称为$\mathcal{F}$的推出。
    \item 如果存在支配有理映射$h:X \dashrightarrow  Z$和$Y$ 上的由点定义的叶状结构$\mathcal{F}_{Z}=0$  ,使得  $\mathcal{F}=h^{-1}(\mathcal{F}_{Z})$,则称$\mathcal{F}$为代数可积 (algebraically integrable)的叶状结构,并且称$\mathcal{F}$由$h$ 诱导。 
  \end{enumerate}
\end{definition}
用$K_{\mathcal{F}}$代替$K_{X}$,可以定义相应的概念
\begin{definition}
使用\cite[3.4.5,6.2.1]{chlx}、\cite[3.2]{acss}的定义:
  \begin{enumerate}
    \item 叶层化代数对 $(X,\mathcal{F},B)$包括一个正规射影代数簇$X$ ,叶状结构$\mathcal{F}$和边界除子$B$,使得叶层化算术典范除子
  \[ K_{\mathcal{F}}+B \]
  是$\mathbb{R}$-Cartier除子。
    \item 如果$f:Y\to X$是双有理态射,且$\mathcal{F}_{Y} $是$\mathcal{F} $诱导的 $Y$ 上的叶状结构,且$E \subset Y$是$Y$ 上的除子,那么分歧公式
  \[ K_{\mathcal{F}_{Y}}+C=f^{*}(K_{\mathcal{F}}+B) \]
  定义了例外除子$E$的叶层化差异数
  \[ a(E;X,\mathcal{F},B)=- \operatorname{mult}_{E}C \]
    \item 用叶层化差异数可以定义奇点性质:
      \begin{enumerate}
        \item 如果对所有除子$E$,有 $a(E;X,\mathcal{F},B)>-\epsilon(E)$,则称$(X,\mathcal{F},B) $具有klt奇点;
        \item 如果对所有除子$E$,有 $a(E;X,\mathcal{F},B)\geqslant -\epsilon(E)$,则称$(X,B) $具有lc奇点 ;
      \end{enumerate}
    \item 代数可积的叶层化代数对$(X,\mathcal{F},B)$如果满足:
      \begin{enumerate}
        \item $X$最多有商奇点 (quotient singularities);
        \item 存在压缩态射$f:X\to Z$,使得$\mathcal{F}$是由$f$ 诱导的叶状结构; 
        \item 存在一个约化除子$\Sigma$使得$\operatorname{Supp}B \subset \operatorname{Supp} \Sigma $,且 $(X,\Sigma)$是环形的 (toroidal),特别的,$X$是$\mathbb{Q}$-分解的具有klt奇点的代数簇;
        \item 存在约化除子$\Sigma_{Z}$使得$ (Z,\Sigma_{Z})$是算术光滑的,并且
          \[ (X,\Sigma)\to (Z,\Sigma_{Z}) \]
          是等维数环形态射 (equidimensional toroidal morphism)
      \end{enumerate}
      则称为叶层化算术光滑代数对。
    \item 对代数可积叶层化代数簇对$(X,\mathcal{F},B)$,如果有双有理态射$f:Y\to X$,使得
      \[ (Y,\mathcal{F}_{Y}=f^{-1}\mathcal{F},B_{Y}=f^{-1}_{*}B+ \operatorname{Exc}f) \]
      是叶层化算术光滑代数簇对,那么称$f$为叶层化算术解消 (foliated log resolution)。
    \item 对具有lc奇点的代数可积叶层化代数簇对$(X,\mathcal{F},B)$,如果有叶层化算术解消$f:Y\to X$,使得任意$X$ 上例外除子$E \subset Y$,都有
      \[ a(E;X,\mathcal{F},B)>-\epsilon_{\mathcal{F}}(E) \]
      那么称其具有$F$-dlt奇点。
  \end{enumerate}
\end{definition}
在叶层化代数簇对中,还有一种特殊的奇点性质,首先由Ambro, Cascini, Shokurov, Spicer在\cite{acss}中提出:
\begin{definition}[ACSS奇点]\cite[Definition 4.1-3]{acc_foliation}
 设$ (X,B) $为具有lc奇点的代数簇对,且$f: X\to Z$是压缩态射。记$B^{v}$和$ B^{h}$分别为$B$ 在$Z$ 上垂直和水平的部分,即$B^{v}$ 的组成成分素除子$D$满足$f(D)\subsetneq Z $,而$B^{h}$  的组成成分素除子$D$满足$f(D)= Z$。如果满足下列性质:
 \begin{enumerate}
   \item 存在$Z$ 上的约化除子$\Sigma_{Z}$,使得$(Z,\Sigma_{Z}) $是算术光滑代数簇对,且$f(B^{v}=\Sigma_{Z})$;
   \item 对任意闭点$z \in Z$和约化除子$\Sigma \geqslant \Sigma_{Z}$,如果$(Z,\Sigma)$在$z \in Z$附近是算术光滑的,那么$(X,B+f^{*}(\Sigma-\Sigma_{Z}))$在$z \in Z$附近具有lc奇点。 
 \end{enumerate}
 那么称$(X,B)/Z$具有性质$*$。

 对于代数可积的叶层化代数簇对$(X,\mathcal{F},B)$,如果存在压缩态射$f:X \to Z$和约化除子$G$,满足下列条件
 \begin{enumerate}
   \item $(X,B+G)/Z$满足性质$*$;
   \item $\mathcal{F}$由$ f$诱导;
   \item $G$是$\mathcal{F}$-不变除子 (在这种情况下,等价于$G$在$Z$上垂直)
 \end{enumerate}
那么称$(X,\mathcal{F},B)/Z$具有性质$*$。进一步,如果还满足
\begin{enumerate}
  \item $(X,\mathcal{F},B)$具有lc奇点;
  \item $f$是等维数压缩态射
\end{enumerate}
则称$(X,\mathcal{F},B;G)/Z$满足弱ACSS条件,或者具有弱ACSS奇点。如果还满足
\begin{enumerate}
  \item 如果除子$\Sigma \subset Z$满足$\Sigma \geqslant f(G)$且$(Z,\Sigma)$是算术光滑代数簇对,那么
    \[ (X,\mathcal{F},B+G+f^{*}(\Sigma - f(G ))) \]
  \item 如果$W$是lc中心,且$ \eta_{W}$是一般点 (generical point),那么$(X,B+G)$在$\eta_{W}$附近是环形的,且$W$ 是$(X,\mathcal{F},\left\lfloor B \right\rfloor )$的lc中心。
\end{enumerate}
那么称具有ACSS性质,或具有ACSS奇点。
\end{definition}
这种奇点在叶层化MMP下保持:
\begin{proposition}\cite[Lemma 4.8]{acc_foliation}
 设$(X,\mathcal{F},B;G)/Z$具有ACSS奇点,并且是$\mathbb{Q}$-分解的。那么 
 \begin{enumerate}
   \item 任何$(K_{\mathcal{F}}+B)$-负性的极端射线都是相对于$Z$ 的 $(K_{X}+B+G)$-负性的极端射线,因此每一步$(K_{\mathcal{F}}+B)$-MMP都是相对于$Z$ 的$(K_{X}+B+G)$-MMP 
   \item 如果$\phi:(X,\mathcal{F},B;G) \dashrightarrow (X',\mathcal{F}',B';\phi_{*}G)$是一部$(K_{\mathcal{F}}+B)$-MMP,那么$(X',\mathcal{F}',B';\phi_{*}G)$也是$\mathbb{Q}$-分解的具有ACSS奇点的代数可积叶层化代数簇对。
 \end{enumerate}
\end{proposition}


\section{三种方法的推广尝试}


\section{弱化版结果}


\chapter{叶层化Sarkisov纲领}
传统双有理代数几何用典范除子$K_{X}$描述代数簇$X$的性质。如果用子层$\mathcal{F} \subset \mathcal{T}_{X}$ 代替切丛,进而用叶层化典范除子 (foliated canonical divisor) $K_{\mathcal{F}}$代替典范除子$K_{X}$,就对应叶层化双有理代数几何。此时有对应的叶层化极小模型纲领,在高维情况下尤其对代数可积的叶层化代数簇对有一系列结果。本章尝试分析这种情况下的Sarkisov纲领。
\section{定义和引理}
首先给出\textbf{叶层化代数簇对}的基本知识。

\begin{definition}[叶状结构]\cite[Definition 2.3-6]{acc_foliation}
  \begin{enumerate}
    \item 令 $X$ 为正规代数簇。$X$ 上的一个叶状结构 (folation)是一个凝聚层 $\mathcal{F} \subset \mathcal{T}_{X}$,满足
          \begin{enumerate}
            \item $\mathcal{T}_{X}/\mathcal{F}$是无扭的 (torsion free);
            \item $\mathcal{F}$在李括号下封闭。 
          \end{enumerate}
          如果$\mathcal{F}=0$,称之为由点定义的叶状结构 (此时此叶状结构的叶子 (leaves)是单点)。
    \item 对$X$ 上素除子 $D$,若$D$ 不是 $\mathcal{F}$-不变的,则定义$\epsilon_{\mathcal{F}}(D)=1 $;反之,如果$D$ 是$\mathcal{F}$-不变的,则定义$\epsilon_{\mathcal{F}}(D)=0$。
    \item 如果除子$K_{\mathcal{F}}$满足$\det \mathcal{F} = \mathcal{O}_{X}(-K_{\mathcal{F}})$,则称$K_{\mathcal{F}}$为$\mathcal{F}$的典范除子。
    \item 令$f:Y \dashrightarrow X$为支配有理映射,将$X$ 上的叶状结构$\mathcal{F}$在$f$ 的拉回记作  $ f^{-1}(F) $ (叶状结构的拉回的定义参考\cite[3.1]{acss} 或\cite[3.2]{cs21}),并且称其为$\mathcal{F}$诱导的 $Y$ 上的叶状结构。令$g:X \dashrightarrow X' $为双有理映射,则将$(g^{-1})^{-1}(\mathcal{F})$记为 $g_{*}\mathcal{F}$,并称为$\mathcal{F}$的推出。
    \item 如果存在支配有理映射$h:X \dashrightarrow  Z$和$Z$ 上的由点定义的叶状结构$\mathcal{F}_{Z}=0$  ,使得  $\mathcal{F}=h^{-1}(\mathcal{F}_{Z})$,则称$\mathcal{F}$为代数可积 (algebraically integrable)的叶状结构,并且称$\mathcal{F}$由$h$ 诱导。 
  \end{enumerate}
\end{definition}
用$K_{\mathcal{F}}$代替$K_{X}$,可以定义相应的概念
\begin{definition}
使用\cite[3.4.5,6.2.1]{chlx}、\cite[3.2]{acss}的定义:
  \begin{enumerate}
    \item 叶层化代数对 $(X/U,\mathcal{F},B)$包括一个正规射影代数簇$X$ ,叶状结构$\mathcal{F}$和边界除子$B$,一个压缩态射$\pi:X\to U$,使得叶层化对数典范除子
  \[ K_{\mathcal{F}}+B \]
  是$\mathbb{R}$-Cartier除子。当基底$U=pt$,或清楚时,常舍去$U$,简记为$(X,\mathcal{F},B)$。 
    \item 如果$f:Y\to X$是双有理态射,且$\mathcal{F}_{Y} $是$\mathcal{F} $诱导的 $Y$ 上的叶状结构,且$E \subset Y$是$Y$ 上的除子,那么分歧公式
  \[ K_{\mathcal{F}_{Y}}+C=f^{*}(K_{\mathcal{F}}+B) \]
  定义了例外除子$E$的叶层化差异数
  \[ a(E;X,\mathcal{F},B)=- \operatorname{mult}_{E}C \]
    \item 用叶层化差异数可以定义奇点性质:
      \begin{enumerate}
        \item 如果对所有除子$E$,有 $a(E;X,\mathcal{F},B)>-\epsilon(E)$,则称$(X,\mathcal{F},B) $具有klt奇点;
        \item 如果对所有除子$E$,有 $a(E;X,\mathcal{F},B)\geqslant -\epsilon(E)$,则称$(X,B) $具有lc奇点 ;
      \end{enumerate}
    \item 代数可积的叶层化代数对$(X,\mathcal{F},B)$如果满足:
      \begin{enumerate}
        \item $X$最多有商奇点 (quotient singularities);
        \item 存在压缩态射$f:X\to Z$,使得$\mathcal{F}$是由$f$ 诱导的叶状结构; 
        \item 存在一个约化除子$\Sigma$使得$\operatorname{Supp}B \subset \operatorname{Supp} \Sigma $,且 $(X,\Sigma)$是环形的 (toroidal),特别的,$X$是$\mathbb{Q}$-分解的具有klt奇点的代数簇;
        \item 存在约化除子$\Sigma_{Z}$使得$ (Z,\Sigma_{Z})$是对数光滑的,并且
          \[ (X,\Sigma)\to (Z,\Sigma_{Z}) \]
          是等维数环形态射 (equidimensional toroidal morphism);
      \end{enumerate}
      则称为叶层化对数光滑代数对。
    \item 对代数可积叶层化代数簇对$(X,\mathcal{F},B)$,如果有双有理态射$f:Y\to X$,使得
      \[ (Y,\mathcal{F}_{Y}=f^{-1}\mathcal{F},B_{Y}=f^{-1}_{*}B+ \operatorname{Exc}f) \]
      是叶层化对数光滑代数簇对,那么称$f$为叶层化对数解消 (foliated log resolution)。
    \item 对具有lc奇点的代数可积叶层化代数簇对$(X,\mathcal{F},B)$,如果有叶层化对数解消$f:Y\to X$,使得任意$X$ 上例外除子$E \subset Y$,都有
      \[ a(E;X,\mathcal{F},B)>-\epsilon_{\mathcal{F}}(E) \]
      那么称其具有$F$-dlt奇点。
  \end{enumerate}
\end{definition}
在叶层化代数簇对中,还有一种特殊的奇点性质,首先由Ambro, Cascini, Shokurov, Spicer在\cite{acss}中提出:
\begin{definition}[ACSS奇点]\cite[Definition 4.1-3]{acc_foliation}
 设$ (X/U,B) $为具有lc奇点的代数簇对,且$f: X\to Z$是压缩态射。
\begin{enumerate}
  \item 记$B^{v}$和$ B^{h}$分别为$B$ 在$Z$ 上垂直和水平的部分,即$B^{v}$ 的组成成分素除子$D$满足$f(D)\subsetneq Z $,而$B^{h}$  的组成成分素除子$D$满足$f(D)= Z$。如果满足下列性质:
        \begin{enumerate}
          \item 存在$Z$ 上的约化除子$\Sigma_{Z}$,使得$(Z,\Sigma_{Z}) $是对数光滑代数簇对,且$f(B^{v}=\Sigma_{Z})$;
          \item 对任意闭点$z \in Z$和约化除子$\Sigma \geqslant \Sigma_{Z}$,如果$(Z,\Sigma)$在$z \in Z$附近是对数光滑的,那么$(X,B+f^{*}(\Sigma-\Sigma_{Z}))$在$z \in Z$附近具有lc奇点。 
        \end{enumerate}
        那么称$(X/U,B)/Z$具有性质$*$。
  \item 对于代数可积的叶层化代数簇对$(X/U,\mathcal{F},B)=(X,\mathcal{F},B)$,如果存在压缩态射$f:X \to Z$和约化除子$G$,满足下列条件
        \begin{enumerate}
          \item $(X/U,B+G)/Z$满足性质$*$;
          \item $\mathcal{F}$由$ f$诱导;
          \item $G$是$\mathcal{F}$-不变除子 (在这种情况下,等价于$G$在$Z$上垂直)
        \end{enumerate}
      那么称$(X/U,\mathcal{F},B;G)/Z$具有性质$*$。
    \item 如果满足性质$*$ 的$(X/U,\mathcal{F},B;G)/Z$还满足:
          \begin{enumerate}
            \item $(X/U,\mathcal{F},B)$具有lc奇点;
            \item $f$是等维数压缩态射
          \end{enumerate}
          则称$(X/U,\mathcal{F},B;G)/Z$满足弱ACSS条件,或者具有弱ACSS奇点。
    \item 如果具有若ACSS奇点的$(X/U,\mathcal{F},B;G)/Z$还满足:
          \begin{enumerate}
            \item 如果除子$\Sigma \subset Z$满足$\Sigma \geqslant f(G)$且$(Z,\Sigma)$是对数光滑代数簇对,那么
              \[ (X/U,\mathcal{F},B+G+f^{*}(\Sigma - f(G ))) \]
            \item 如果$W$是lc中心,且$ \eta_{W}$是一般点 (generical point),那么$(X,B+G)$在$\eta_{W}$附近是环形的,且$W$ 是$(X/U,\mathcal{F},\left\lfloor B \right\rfloor )$的lc中心。
          \end{enumerate}
          那么称具有ACSS性质,或具有ACSS奇点。
\end{enumerate}
\end{definition}
ACSS奇点在叶层化MMP下保持:
\begin{proposition}\cite[Lemma 4.8]{acc_foliation}\label{acssmmp}
 设$(X/U,\mathcal{F},B;G)/Z$具有ACSS奇点,并且是$\mathbb{Q}$-分解的。那么 
 \begin{enumerate}
   \item 任何$(K_{\mathcal{F}}+B)$-负性的相对于$U$的极端射线都是 $(K_{X}+B+G)$-负性的相对于$Z$ 的极端射线,因此每一步相对于$U$ 的 $(K_{\mathcal{F}}+B)$-MMP都是相对于$Z$ 的$(K_{X}+B+G)$-MMP;
   \item 如果$\phi:(X,\mathcal{F},B;G) \dashrightarrow (X',\mathcal{F}',B';\phi_{*}G)$是一部$(K_{\mathcal{F}}+B)$-MMP,那么$(X',\mathcal{F}',B';\phi_{*}G)$也是$\mathbb{Q}$-分解的具有ACSS奇点的代数可积叶层化代数簇对。
 \end{enumerate}
\end{proposition}
$F$-dlt奇点的叶层化代数簇对具有ACSS性质:
\begin{theorem}\cite[Theorem 17.0.1]{chlx}
 如果$\mathbb{Q}$-分解的代数可积叶层化代数簇对$(X,\mathcal{F},B)$具有$F$-dlt奇点,那么具有ACSS奇点。  
\end{theorem}

\section{三种方法的推广尝试}
首先,由下列定理,在特定奇点设定下,可以运行某些叶层化MMP:
\begin{theorem}[叶层化MMP]\cite[Theorem 2.1.1]{chlx}
设$(X/U,\mathcal{F},B)$是$\mathbb{Q}$-分解的具有$F$-dlt奇点的代数可积的叶层化代数簇对,那么对应的锥定理、翻转的存在性等定理成立,即可以运行相对于$U$ 的 $(K_{\mathcal{F}}+B)$-MMP。  
\end{theorem}
但是只对部分情况下有终结性:
\begin{theorem}[叶层化MMP终结性]\cite[Theorem 2.1.2]{chlx}
设$(X/U,\mathcal{F},B)$是$\mathbb{Q}$-分解的具有$F$-dlt奇点的代数可积的叶层化代数簇对,并且满足$B\geqslant A \geqslant 0$,其中$A$是丰沛的$\mathbb{R}$-除子。假设$H$是丰沛的$\mathbb{R}$-除子,那么可以运行关于$H$标量的$(K_{\mathcal{F}}+B)$-MMP,并且终结。特别的,
\begin{enumerate}
  \item 如果$K_{\mathcal{F}}+B$不是相对于$U$ 伪有效的,那么终结于叶层化森纤维空间;
  \item 如果$K_{\mathcal{F}}+B$是相对于$U$ 伪有效的,那么终结于好极小模型 (good minimal model,定义见\cite[Definition 9.1.1]{chlx})。
\end{enumerate}
\end{theorem}

假设$(W/U,\mathcal{F}_{W},B_{W})$是代数可积的叶层化代数簇对,并且具有$F$-dlt奇点。运行相对于$U$ 的 $(K_{\mathcal{F}_{W}}+B_{W})$-MMP,并且假设有两个不同的叶层化森纤维空间$(X,\mathcal{F},B)\to S$和$(X',\mathcal{F}',B')\to S'$作为叶层化MMP的不同输出。此时有双有理映射$\Phi:X \dashrightarrow X'$,目标是构造这个映射的Sarkisov分解,并且其中每一个$X_{i}$,都有一个叶层化森纤维空间$(X_{i},\mathcal{F}_{i},B_{i})\to S_{i}$。接下来分析三种方法在叶层化代数簇对上能否推广。

\textbf{下降法:}
对于具有$F$-dlt奇点的代数可积叶层化代数簇对,有下列结果;
\begin{lemma}\cite[Lemma 6.2.4]{chlx}
  令$X$ 是正规拟射影代数簇, $f:X \dashrightarrow Z$是支配的有理映射,且$\mathcal{F}$是由$f$ 诱导的代数可积的叶状结构,$B$ 是有效$\mathbb{R}$-除子。那么叶层化代数簇对$(X,\mathcal{F},B)$的叶层化对数解消存在。
\end{lemma}
\begin{theorem}[压缩超例外除子]\cite[Theorem 9.4.1]{chlx}
 令$(X/U,\mathcal{F},B)$是$\mathbb{Q}$-分解的具有ACSS奇点的叶层化代数簇对,且
 \[ K_{\mathcal{F}}+B \sim_{\mathbb{R},U}E \]
其中$E$是相对于$U$ 的超例外除子 (super exceptional divisor),并且是有效$\mathbb{R}$-除子。那么关于某丰沛除子$A$ 标量的相对于$U$的任何$(K_{\mathcal{F}}+B)$-MMP终结于$(X',\mathcal{F}',B')$,是相对于$U$ 的好极小模型,并且$X \dashrightarrow X' $的例外除子恰好是$E$。 
\end{theorem}
注意到当$X\to U$是双有理射影态射时,所有例外除子都是超例外除子。因此用上述定理可以构造叶层化代数簇对的除子解压。同时由于某些丰沛除子标量的MMP终结,用下降法的思路可以构造类似的Sarkisov连接。

但是,由于Fano型叶层化代数簇对的有界性暂时未知,所以这种方法的终结性仍未明确。

\textbf{双标量法和有限模型法:}
对于双标量法,需要构造公共解消$W$和边界除子$B_{W},G_{W},H_{W}$,使得$(W,B_{W}+2G_{W}+2H_{W})$具有终端型奇点,并且需要定义一个序关系 (定义\ref{doubleorder})。这对于叶层化代数簇的的情况是困难的。

另外,这两种方法都依赖于模型的有限性,虽然有叶层化代数簇对的对应模型的定义,但其有限性暂时未知。因此这两种方法暂时难以推广。
\section{弱化版结果}
但是,通过将叶层化代数簇对约化为普通的代数簇对,可以得到较弱的结果。
\begin{theorem}\cite[Proposition 3.6]{acss}或\cite[Proposition 7.3.6]{chlx}  
  令$(X/U,\mathcal{F},B;G)/Z$是具有ACSS奇点的$\mathbb{Q}$-分解的叶层化代数簇对,那么
  \begin{enumerate}
    \item 相对于$Z$ 有线性等价:
      \[ K_{X}+B+G\sim_{\mathbb{R},Z}K_{\mathcal{F}}+B \]
    \item $(X,0)$具有klt奇点,$(X,B+G)$具有lc奇点。
  \end{enumerate}
\end{theorem}
% 用上述结果,可以将叶层化森纤维空间$(X,\mathcal{F},B)\to S$和$(X',\mathcal{F}',B')\to S' $转化为普通的森纤维空间。
用上述结果,可以将叶层化代数簇对$(W,\mathcal{F}_{W},B_{W})$转化为普通的森纤维空间,进而将$X\to S$和$X' \to S'$视作普通的森纤维空间。
为了应用Sarkisov分解,还需要将lc奇点的代数簇对约化为klt奇点:
\begin{lemma}\label{lctoklt}
 令$(X,B+A)$是具有lc奇点的$\mathbb{Q}$-分解的代数簇对,其中$A$是丰沛除子,并且$(X,0)$具有klt奇点。那么存在边界除子$D$,使得
 \[ K_{X}+B+A\sim_{\mathbb{R}}K_{X}+D \]
 并且$(X,D)$具有klt奇点。
\end{lemma}
\begin{proof}
  令$f:Y\to X$为对数解消,考虑分歧公式
  \[ K_{Y}+E_{K}=f^{*}K_{X} \]
  和
  \[ K_{Y}+f^{-1}_{*}B+E_{B}+E_{K}+f^{*}A=f^{*}(K_{X}+B+A) \]
  由$(X,0)$具有klt奇点知$E_{K}< 1 $;由$(X,B+A)$具有lc奇点知$E_{B}\leqslant 1$。由于丰沛是开性质,所以存在充分小$0< \epsilon \ll 1$,使得$A+\epsilon B$是丰沛除子,并且$f^{*}(A+\epsilon B)$是数值有效的大除子,因此存在$Y$ 上的丰沛除子$A_{Y}$和充分小有效除子$E_{Y}$,使得
  \[ f^{*}(A+\epsilon B)\sim_{\mathbb{R}} A_{Y}+E_{Y} \]
  且
  \[ \left(Y,D_{Y}=(1-\epsilon)(f^{-1}_{*}B+E_{B})+E_{K}+A_{Y}+E_{Y}\right) \]
  具有klt奇点。 那么
  \[ K_{Y}+f^{-1}_{*}B+E_{B}+E_{K}+f^{*}A\sim_{\mathbb{R}} K_{Y}+D_{Y} \]
  令$D:=f_{*}D_{Y}$,则
  \[ K_{X}+B+A\sim_{\mathbb{R}} K_{X}+D \]
  且$(X,D)$具有klt奇点。
\end{proof}
对两个$(K_{\mathcal{F}_{W}}+B_{W})$-MMP进行微扰,可以在边界除子$B_{W}$中添加丰沛成分。具体来说,令$\rho:W \dashrightarrow X$和$\rho': W \dashrightarrow X'$为两个不同的$(K_{\mathcal{F}_{W}}+B)$-MMP的过程,因为在有限步内结束,所以都之经过有限次关于某个极端射线$R$ 的 压缩态射或翻转,即每一步中都有
\[ (K_{\mathcal{F}_{i}}+B_{i}).R<0 \]
令$A_{W}$是$W$ 上任意丰沛除子,且$0<\epsilon\ll 1$充分小,使得
\[ (K_{\mathcal{F}_{i}}+B_{i}+\epsilon A_{i}).R<0 \]
恒成立,并且$(W,\mathcal{F}_{W},B_{W}+\epsilon A_{W})$具有$F$-dlt奇点,那么$\rho,\rho'$也是$(K_{\mathcal{F}_{W}}+B+\epsilon A)$-MMP。用$\epsilon A$替换$A$,可以假设$(W,\mathcal{F}_{W},B_{W}+A_{W})$是 具有$F$-dlt奇点的$\mathbb{Q}$-分解的代数可积的叶层化代数簇。 
用上述结果可以证明定理\ref{mainf}
\begin{proof}
  由于$(W,\mathcal{F}_{W},B_{W})$具有$F$-dlt奇点,所以存在除子$G_{W}$和压缩态射$W\to Z$,使得$(W,\mathcal{F}_{W},B_{W};G)/Z$ 具有ACSS奇点,并且有
  \[ K_{\mathcal{F}}+B_{W} \sim_{Z} K_{X}+B+G \]
 取$W$ 上的丰沛除子$A_{W}$,满足
 \begin{enumerate}
   \item $(W,B_{W}+G_{W}+A_{W}) $具有lc奇点;
   \item $A_{W}$不包含$(X,B+G)$的lc中心;
   \item $\rho,\rho' $也是$(K_{\mathcal{F}_{W}}+B_{W}+A_{W})$-MMP的输出;
 \end{enumerate}
 由定理\ref{acssmmp},$(K_{\mathcal{F}_{W}}+B_{W}+A_{W})$-MMP的输出$X,X'$也是相对于$Z$ 的$(K_{W}+B_{W}+G_{W}+A_{W})$-MMP的输出。由引理\ref{lctoklt},存在除子
\[ D_{W}\sim B_{W}+G_{W}+A_{W} \]
使得$(W,D_{W})$是$\mathbb{Q}$-分解的klt奇点的代数簇对,且$(X,D)\to S $和$(X',D')\to S' $是它相对于$Z$ 的两个森纤维空间。由定理\ref{main},双有理映射$\Phi:X\to X'$由Sarkisov分解。 
\[ X=X_{0}\dashrightarrow X_{1}\dashrightarrow \cdots \dashrightarrow X_{N}=X' \]
\end{proof}

\begin{remark}
 这个分解中的Sarkisov连接$X_{i}\dashrightarrow X_{i+1}$是根据上述构造的边界除子$D_{W}$得到的。即存在除子$D_{i}$和$D_{i}'$,使得交换图表
  \[ \xymatrix{
      Z_{i}\ar@{.>}[d]_p\ar@{.>}[rr]^{\tau}&&Z_{i}'\ar@{.>}[d]^q\\
      X_{i}\ar[d]_{\phi}&&X_{i+1}\ar[d]^\psi\\
      S_{i}\ar[rd]^s&&S_{i+1}\ar[ld]_t\\
      &T&} \]
      中,$p,q,\tau$都是关于某个边界除子$D_{i}'$的翻转或平转或无差别除子解压,而$(X_{i},D_{i})\to S_{i}$是普通代数簇对的森纤维空间,不是叶层化森纤维空间。这与所期望的叶层化Sarkisov分解并不一致,因此将定理\ref{main2}称为弱化的结果。
\end{remark}


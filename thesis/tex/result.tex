\chapter{三种方法的总结}

\section{三种方法的比较}

\section{例子}
首先给出一些具体的MMP-相关的森纤维空间的Sarkisov分解的例子。

\subsection{下降法}\label{example1}
记 $ X=\mathbb{P}^2 $ 为射影平面,坐标记为 $ [ x_0:x_1:x_2 ] $ ,记 $ X'=\mathbb{P}^2 $ 为另一个射影平面,坐标为 $ [ y_0:y_1:y_2 ] $。
记 $ B=\{x_0=0\} $ 和 $B'=\{y_{0}=0\} $。
取双有理映射 $ \Phi:X\dashrightarrow X' $ ,按如下定义:
\[ \Phi:[x_0:x_1:x_2]\dashrightarrow [x_0^2:x_0x_1:x_1^2+x_0x_2] \]
那么存在一个公共解消 $\sigma: W\to X$ 和 $\sigma':W\to X'$,两个态射都是三个对点的胀开的复合。
三个胀开如下所示:
\begin{enumerate}
  \item $\pi_{1}:W_{1}\to X$ 是对点 $P_{0} \in B$ 的胀开,点 $P_{0}$是双有理映射  $\Phi$不定义的点。 用相同的符号 $B$ 表示它在  $W_{1}$上的严格双有理变换。将$\pi_{1}$的例外除子记为 $E_{1}$。此时$\Phi$提升到 双有理映射 $W_{1}\dashrightarrow X'$,但仍然不是态射。
  \item $\pi_{2}:W_{2}\to W_{1}$ 是对点 $P_{1}=E_{1} \cap B$的胀开,点 $P_{1}$是双有理映射 $  W_{1}\dashrightarrow X' $
的不定义的点。将$\pi_{2}$的例外除子记为 $E_{2}$,仍然用同样的符号 $B$ 和 $E_{1}$ 表示它们在 $W_{2}$ 上的严格双有理变换。此时$\Phi$提升到 双有理映射 $W_{2}\dashrightarrow X'$,但仍然不是态射。
  \item $\pi_{3}:W=W_{3}\to W_{2}$ 是对点 $P_{2} \in E_{2} \setminus (B\cup E_{1})$的胀开。将$\pi_{3}$的例外除子记为 $E_{3}$,仍然用同样的符号 $B,E_{1},E_{2}$ 表示它们在 $W_{3}$ 上的严格双有理变换。此时$\Phi$提升到 双有理映射 $W_{3}\to  X'$是一个态射。
\end{enumerate}
那么 $ \sigma=\pi_{3}\circ \pi_{2} \circ \pi_{1} $ 和 $ W=W_3 $ 是 $\Phi$的公共解消。对称的, $ \sigma':W\to X' $ 是三个压缩曲线的态射的复合  $W=W'_{3}\xrightarrow{\pi'_{3}} W'_{2}\xrightarrow{\pi'_{2}} W'_{1} \xrightarrow{\pi'_{1}} X'$。双有理态射$\pi'_{3},\pi'_{2},\pi'_{1}$ 依次压缩曲线 $ B,E_2,E_1 $。
\begin{remark}
  显然双有理映射$\Phi$有逆映射
\[ \Phi^{-1}:[y_0:y_1:y_2]\dashrightarrow [y_0^2:y_0y_1:y_1^2-y_0y_2] \]
与$\Phi $只差一个符号。于是$\sigma' $是解消 $\Phi^{-1}$的三个胀开的复合,例外曲线依次是$E_{1},E_{2},B$。可以看到,$B' \subset X'$是例外除子$E_{3} \subset W$在$X'$上的严格双有理变换。在$W$ 上,如果用相同的符号$B'$记它在 $W$ 上的严格双有理变换,那么 除子$E_{3}$和$B$ 是地位相等的。
\end{remark}
下面给出一些代数簇的符号:
\begin{itemize}
  \item 令 $W_{2}\to Z_{0}$ 是  $W_{2}$压缩曲线 $E_{1}$的态射,且 $Z_{0} \to X_{1}$ 是压缩曲线 $B$的态射,那么 $Z_{0}\to X_{0}$ 是  $X$上解压 $E_{2}$除子解压;
  \item Let $W'_{2}\to Z_{1}$ be the contraction of $E_{1}$ on $W'_{2}$, then $Z_{1} \to X_{1}$ is the extraction of $E_{3}$ on $X_{1}$, and $Z_{1}\to X'$ is the contraction of $E_{2}$;
  \item $W\to Z$ be the contraction of $E_{1}$ and $E_{2}$ on $W$, then $Z\to X$ is the extraction of $E_{3}$ and $Z\to X'$ is the contraction of $B$.
\end{itemize}
That is
\[ \xymatrix{&&&W_3\ar[ld]\ar@{=}[r]&W\ar[ddd]\ar@{=}[r]&W_3'\ar[rd]\\
    &&W_2\ar[ld]\ar[rd]&&&&W_2'\ar[rd]\ar[dl]\\
    &W_1\ar[ld]&&Z_0\ar[llld]\ar[rd]&&Z_1\ar[ld]\ar[rrrd]&&W_1'\ar[rd]\\
    X_0&&&&X_1&&&&X'
  } \]
and
\[\xymatrix{
    &W\ar[d]\ar[ddl]_{\sigma}\ar[ddr]^{\sigma'}&\\
    &Z\ar[dl]\ar[dr]\\
    X&&X' }  \]

Consider the pairs $ (X,bB) $ and $ (X',b'B') $, and take the function $\theta$ such that:
\begin{itemize}
  \item $\theta(B)=b$ and $\theta(B')=b'$;
  \item $\theta(E_{1})=\theta(E_{2})=\epsilon$ with $b,b'<\epsilon<1$.
\end{itemize}
Then we have the ramification formulas:
\[ \begin{array}{rllllllllll}
    K_W+B_W & = & \sigma^*(K_X+bB)       & + & (3-2b+b')E_3 & + & (1-b+\epsilon)E_1 & + & (2-2b+\epsilon)E_2  & \\
            & = & \sigma'^*(K_{X'}+b'B') & + & (3-2b'+b)B   & + & (1-b'+e)E_1       & + & (2-2b'+\epsilon)E_2 &
  \end{array} \]
Let $ \mathcal{H}'=|\mathcal{O}_{X'}(1)| $ be the very ample complete linear system on $X'$, then $H\in |\mathcal{O}_{X}(2)|$. 

Different choices of $\theta$ and  $\epsilon$ give different decompositions:
\begin{enumerate}
  \item\label{example1.1} If $ 2b+2b'\geqslant 3\epsilon>0 $, then $\Phi$ is the composition of two Sarkisov links $\psi_{0},\psi_{1}$ of type II:

  \[
    \xymatrix{
    &Z_{0}\ar[rd]\ar[ld] & &Z_{1}\ar[rd]\ar[ld]\\
    X\ar[d]\ar@{.>}[rr]^{\psi_{0}}& &X_{1}\ar@{.>}[rr]^{\psi_{1}}\ar[d]&&X'\ar[d]\\
    \text{pt}&&\text{pt}&&\text{pt}
    }
  \]
  \item If $ 2b+2b'< 3\epsilon $, then $\Phi$ is just one Sarkisov link  of type II:
        \[
          \xymatrix{
            &Z\ar[rd]\ar[ld] & \\
            X_{1}\ar@{.>}[rr]^{\Phi}\ar[d]& &X'\ar[d]\\
            \text{pt}&&\text{pt}
          }
        \]
  \item If $ \epsilon=b=b'=0 $, then $\Phi$ is the composition of four Sarkisov links $\psi_{i}$:
        \[ \xymatrix{
          &&W_2\ar[ld]\ar[rd]&&W_2'\ar[ld]\ar[rd]\\
          &X_1=\mathbb{F}_1\ar@{.>}[rr]^{\psi_{1}}\ar[d]&&X_2=\mathbb{F}_2\ar@{.>}[rr]^{\psi_{2}}\ar[d]&&X_3=\mathbb{F}_1\ar[d]\ar[rd]^{\psi_{3}}\\
          X=X_0=\mathbb{P}^2\ar[d]\ar@{.>}[ru]^{\psi_{0}}&\mathbb{P}^1\ar[ld]\ar@{=}[rr]&&\mathbb{P}^1\ar@{=}[rr]&&\mathbb{P}^1\ar[rd]&X_4=X'=\mathbb{P}^2\ar[d]\\
          \text{pt}&&&&&&\text{pt} } \]
\end{enumerate}

\subsection{Double scaling method}
Notations and assumptions as in Section \ref{example1}, let $B_{W}=\frac{1}{2}(B+E_{1}+E_{3})$ and consider pairs $(X,\frac{1}{2}B)$ and $(X',\frac{1}{2}B')$. Then we have $G=G_{0}\sim_{\mathbb{Q}}\frac{5}{2}B$ and $H'\sim_{\mathbb{Q}}\frac{5}{2}B'$.

\begin{enumerate}
  \item $r_{0}=2$ and $s_{0}=\frac{1}{5}$. $X_{1}$ is a weak log canonical model of $(W,B_{W}+\frac{3}{5}G_{W}+\frac{1}{5}H_{W})$;
  \item  $r_{1}=1$ and $s_{1}=\frac{2}{5}$. $X_{2}= X'$ is a weak log canonical model of $(W,B_{W}+\frac{1}{5}G_{W}+\frac{3}{5}H_{W})$.
\end{enumerate}

This gives the same decomposition as in the case (\ref{example1.1}) in  Section \ref{example1}.

\subsection{Polytope method}
Let  $P,Q$ be two different points on $\mathbb{P}^{2}$ and let $L$ be the line passing through $P$ and $Q$. Let $p:X\to \mathbb{P}^{2}$ be the blow-up at $P$ and $E_{1}$ be the exceptional divisor. Let $q:Y\to \mathbb{P}^{2}$ be the blow-up at $Q$ and $E_{2}$ be the exceptional divisor. Let $W\to \mathbb{P}^{2}$ be the blow-up of $P$ and $Q$, then we have contractions $f:W\to X$ and $g:W\to Y$. Identify $L,E_{1}$ and $E_{2}$ with their strict transforms on  $W$. Let $h:W\to Z$ be the contraction of $L$, then $Z\cong \mathbb{P}^{1} \times \mathbb{P}^{1}=\mathbb{F}_{0}$.
\[
  \xymatrix{
    S\cong \mathbb{P}^{1}&X\cong \mathbb{F}_{1}\ar[l]_{\phi}\ar[r]^{p} & \mathbb{P}^{2} \\
    &W\ar[u]^{f}\ar[r]_{g}\ar[ld]_{h} & Y\cong \mathbb{F}_{1}\ar[u]_{q}\ar[d]^{\psi}\\
    Z\cong \mathbb{F}_{0}\ar[uu]\ar[rr]&&T\cong \mathbb{P}^{1}
  }
\]

Note that $X\cong \mathbb{F}_{1}$, there is a  Mori fibre space $\phi:X\to S \cong \mathbb{P}^{1}$. Similarly, there is another Mori fibre space $\psi: Y\to T\cong \mathbb{P}^{1}$. There is a birational map $\Phi: X\dashrightarrow  Y$ induced by $p$ and $q$. If we take $B_{W}=\frac{1}{4}L$ on $W$, then $f$ and $g$ are two log Mori fibre spaces given by the outputs of $(K_{W}+B_{W})$-MMPs.

Take $A\sim_{\mathbb{Q}}-K_{W}+\frac{1}{4}L$, and let $V$ be the translation by  $\frac{1}{4}L$ of the 2-dimensional vector space spanned by $E_{1}$ and  $E_{2}$. Then we have $\mathcal{L}_{A}(V)=\mathcal{E}_{A}(V)$. Furthermore,  $K_{W}+D\sim_{\mathbb{Q}} \frac{1}{2}L+aE_{1}+bE_{2}$ for $0\leqslant a,b\leqslant 1$ if $D \in \mathcal{E}_{A}(V)$. The partition of $\mathcal{E}_{A}(V)$ is
\begin{figure}
\centering
  \begin{tikzpicture}
    \draw (0,0)--(6,0)--(6,6)--(0,6)--(0,0);
    \draw (0,3)--(6,3);
    \draw (3,0)--(3,6);
    \draw (3,0)--(0,3);
    \draw(4.5,4.5)node{$\mathbb{P}^{2} $};
    \draw(0.9,0.9)node{$Z$};
    \draw(2.1,2.1)node{$W$};
    \draw(4.5,1.5)node{$Y$};
    \draw(1.5,4.5)node{$X$};
    \draw(-0.5,-0.5)node{pt};
    \draw(-1,3)node{$\mathbb{P}^{1}$};
    \draw(3,-1)node{$\mathbb{P}^{1}$};
    \draw(4.5,0.3)node{$D_{1}$};
    \filldraw (4.5,0) circle(0.05);
    \filldraw (3,0) circle(0.05);
    \draw(3.3,0.3)node{$D_{4}$};
    \filldraw (0,0) circle(0.05);
    \draw(0.3,0.3)node{$D_{3}$};
    \filldraw (0,3) circle(0.05);
    \draw(0.3,3.3)node{$D_{2}$};
    \filldraw (0,4.5) circle(0.05);
    \draw(0.5,4.5)node{$D_{0}$};
  \end{tikzpicture}
    \caption{Decomposition of $\mathcal{E}_A(V)$}
    \label{pic}
\end{figure}
Then $D_{0}$ and $D_{1}$ correspond to log Mori fibre spaces $\phi:X\to S$ and $\psi:Y\to T$. $D_{2},D_{3}$ and $D_{4}$ correspond to three Sarkisov links. Therefore, we have a decomposition of $\Phi: X\dashrightarrow  Y$ as

\[
  \xymatrix{
    &W\ar[ld]\ar[rd] &&&&W\ar[ld]\ar[rd] \\
    X\ar[d]& &Z\ar[d]\ar@{=}[rr]&&Z\ar[d]&&Y\ar[d]\\
    \mathbb{P}^{1}&&\mathbb{P}^{1}\ar[dr]&&\mathbb{P}^{1}\ar[dl]&&\mathbb{P}^{1}\\
    &&&\text{pt}
  }
\]

\section{应用}


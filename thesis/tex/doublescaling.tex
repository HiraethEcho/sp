\chapter{双标量法}
 本章介绍\cite[\S 13]{haconMinimalModelProgram2012} 和  \cite{liuSarkisovProgramGeneralized2021}中的双标量法构造 (double scaling)的Sarkisov纲领。下降法用 $X'$上的除子 $H'$ 来作为标量,双标量法在  $X$ 和 $X'$ 上分别构造除子 $G$ 和 $H'$ ,作为双标量。
% \textbf{初始设定:}
令 $W$ 为 MMP-相关的森纤维空间 $f:(X,B)\to S$ 和 $f':(X',B')\to S'$  的公共解消,  取在$ S $ 上的$\mathbb{Q}$-除子 $A$ 使得 $G \sim_{\mathbb{Q}} -(K_{X}+B) +f^*A$ 是一般的丰沛 $\mathbb{Q}$-除子。类似的,取 $S'$ 上的丰沛$\mathbb{Q}$-除子,使得 $H' \sim_{\mathbb{Q}} -(K_{X'}+B') +f'^*A'$ 是一般的丰沛 $\mathbb{Q}$-除子。那么  $(X,B+G)$ 和 $(X',B'+H')$  是$W$ (关于 $B_{W}+G_{W}$ 和 $B_{W}+H_{W}$)的两个弱算术典范模型。双标量Sarkisov纲领将构造 $W$ 的有限多个弱算术典范模型 $(X_{i},B_{i}+g_{i}G_{i}+h_{i}H_{i})$,它们是分别关于 $(W,B_{W}+g_{i}G_{W}+h_{i}H_{W})$的模型,其中 $0\leqslant g_i,h_i\leqslant 1$。这些模型间的双有理映射 $\psi_{i}:X_{i}\dashrightarrow X_{i+1}$是通过$2$-ray game 构造的Sarkisov连接映射。

和下降法相比,这种方法在一个更小的集合内运行MMP和Sarkisov纲领。所有的 $X_{i}$都对应一个$(K_{W}+D_{i})$-非正性的双有理压缩映射 $W\dashrightarrow X_{i}$,其中 $D_{i}$是 $\operatorname{WDiv}_{\mathbb{R}}(W)$ 的一个紧子集 $\mathcal{E}_{A}(V)$。对于这个子集有弱算术典范模型的有限性性质,由此可以得到双标量法的Sarkisov纲领的终结性。
\section{定义和引理}
令 $(W, B_W)$ 是 $\mathbb{Q}$-分解具有 klt奇点的代数簇对,并且 $f:(X, B)\to S$ 和 $f':(X', B')\to S'$ 是两个不同的$(K_{W}+B_{W})$-MMP 的输出,且都是森纤维空间。接下来给出一些所需的记号和引理。 
\begin{definition}
  令 $f: W\dashrightarrow X$ 是正规拟射影代数簇间的双有理映射,如果
  \begin{itemize}
    \item $f$ 不解压除子;
    \item 对所有 在 $W$ 之上的 (包括更高模型上的)除子 $E$ 都有  $a(E;X,B_{X})\geqslant a(E;W,B_{W})$,
  \end{itemize}
  那么记 $(W,B_{W})\geqslant (X,B_{X})$。
\end{definition}
对于具有终端奇点的代数簇对,有下面的引理:
\begin{lemma}\label{terminalorder}
  \cite[Lemma 13.8]{haconMinimalModelProgram2012} 令 $f:W\dashrightarrow X$ 是双有理映射且 $(W,B_{W})$  具有终端奇点。如果
  \begin{itemize}
    \item $f$不解压除子;
    \item $K_X+B$ 是数值有效除子,其中$B=f_*B_W$;
    \item  对所有$E \subset W$之上的除子 $E$ 都有 $a(E;X,B)\geqslant a(E;W,B_{W})$。
  \end{itemize}
  那么有
  \begin{itemize}
    \item $(W,B_{W})\geqslant (X,B_{X})$;
    \item $(X,B_{X})$ 具有klt奇点;
    \item 如果 $Z\to X$ 是 关于$E$的除子解压,其中$a(E;X,B_{X})\leqslant 0$,那么 $E$是 $W$上的除子;
    \item 如果 $Z\to X$是 $(X,B_{X})$的终端化,那么 $W\dashrightarrow Z$ 不解压除子。
  \end{itemize}
\end{lemma}
相反的,从非正性映射出发,可以构造这样的具有终端奇点的更高模型:
\begin{lemma}\label{terminalresolution}
  \cite[Lemma 3.5]{liuSarkisovProgramGeneralized2021} 令 $\sigma:(W,B_W)\dashrightarrow (X,B)$是 $(K_W+B_W)$-非正性双有理映射,使得 $\sigma_*(K_W+B_W)=K_X+B$ 且 $(W,B_W)$ 是 $\mathbb{Q}$-分解的具有 klt奇点的代数簇对,那么存在一个映射解消$\pi:\tilde{W}\to W $ 和 $\tilde{\sigma}:\tilde{W}\to X$使得 
  \begin{itemize}
    \item $(\tilde{W},B_{\tilde{W}})$ 是 $\mathbb{Q}$-分解的具有终端奇点的代数簇对,且 $\tilde{\sigma}_*B_{\tilde{W}}=B$;
    \item $\tilde{\sigma}$ 是 $(K_{\tilde{W}}+B_{\tilde{W}})$-非正性的且 $(\tilde{W},B_{\tilde{W}})\geqslant (X,B)$;
  \end{itemize}
  \[
    \xymatrix{
    & \widetilde{W}\ar[ld]\ar[rd] & \\
      (W,B_{W})\ar@{.>}[rr] & &(X,B)
    }
  \]
  
\end{lemma}

通过应用引理 \ref{terminalresolution},可以调整双标量法Sarkisov纲领的初始设定:
\begin{assertion}\label{doublesetting}
用 $(W,B_W)$ 的奇点解消代替它,使得 $(W,B_{W})$具有终端奇点,且$\sigma:W\to X$ 和$\sigma':W\to X'$ 是 $(K_W+B_W)$-非正性态射,并且 $(W,B_W)\geqslant (X,B),(X',B')$。

  取 $ S $ 和 $ S' $ 上的极丰沛 $\mathbb{Q}$-除子 $ A $ 和 $ A' $使得 $ G\sim_{\mathbb{Q}}-(K_X+B)+f^*A $ 和 $ H\sim_{\mathbb{Q}}-(K_{X'}+B')+f'^{*}A' $ 是丰沛 $\mathbb{Q}$-除子。进一步,可以假设 $ G $ 和 $ H $ 满足 $G_{W}:= \sigma^*G=\sigma^{-1}_*G $ 和 $ H_{W}:=\sigma'^{*}H=\sigma'^{-1}_*H $。因此 $\sigma_{*}(K_{W}+B_{W}+G_{W})=K_{X}+B+G$ 是数值有效的,因此引理 \ref{terminalorder} 的结论满足。
如果必要的话取进一步解消,可以假设
 $(W, B_W+gG_W+hH_W)$ 算术光滑,并且对所有 $0\leqslant g,h\leqslant 2$都具有终端奇点。
\end{assertion}
 在这个设定下,有双标量的Sarkisov纲领:
\begin{theorem}[双标量Sarkisov纲领]\label{main2}
  \cite[Claim 13.12]{haconMinimalModelProgram2012}
  沿用上述记号,有有限多个Sarkisov连接的序列
  \[
    \xymatrix{
    X=X_0\ar@{.>}[r]\ar[d]_{f=f_{0}}&X_{1}\ar@{.>}[r]\ar[d]_{f_{1}}& X_{2}\ar[d]_{f_{2}}\ar@{.>}[r] &\cdots\ar@{.>}[r]&X_N=X'\ar[d]_{f_{N}} \\
    S=S_{0}&S_{1} &S_{2}&&S_N=S'
    }
  \]
  和有理数
  \[
    \begin{aligned}
      1 & =g_0\geqslant g_1 \geqslant \cdots \geqslant g_N   & =0 \\
      % 0 & =h_0\leqslant h_{1} \leqslant \cdots \leqslant h_N &  \\
      0 & =h_0\leqslant h_{1} \leqslant \cdots \leqslant h_N & =1 \\
    \end{aligned}
  \]
  满足
  \begin{enumerate}
    \item 对每个 $i$,双有理压缩映射 $\sigma_i:W\dashrightarrow  X_{i}$都是 $(K_{W}+B_{W}+g_{i}G_{W}+h_{i}H_{W})$-非正性的,并且 $(K_{X_{i}}+B_{i}+g_{i}G_{i}+h_{i}H_{i})=\sigma_{i*}(K_{W}+B_{W}+g_{i}G_{W}+h_{i}H_{W})$ 是数值有效的,且相对于$S_{i}$数值平凡;
    \item $(W,B_{W}+g_{i}G_{W}+h_{i}H_{W})\geqslant (X_{i},B_{i}+g_{i}G_{i}+h_{i}H_{i})$;
    \item 每一个Sarkisov连接 $X_{i}\dashrightarrow X_{i+1}$ 都是一组 $(K_{X_{i}}+B_{i}+g_{i}G_{i}+h_{i}H_{i})$-平凡的映射的复合;
    \item 最后一个森纤维空间 $X_{N} \to S_{N}$ 同构于 $X'\to S'$。
  \end{enumerate}
\end{theorem}
这里平凡映射指
\begin{definition}\label{trivialmap}
  \cite[\S 13.2]{haconMinimalModelProgram2012} 令 $f:X\dashrightarrow Y$ 是相对于$S$ 正规拟射影代数簇间的有理映射,并且 $D$是 $X$  上  $\mathbb{R}$-Cartier $\mathbb{R}$-除子,且$f_*D=D_Y$。   如果 $D$ 是 $S$ 上 某个  $\mathbb{R}$-Cartier 除子的拉回,那么 称$f$ 为 \textbf{$D$-平凡}(\textbf{$D$-trivial })。
\end{definition}

\section{用双标量法构造Sarkisov连接}
这一节将归纳地构造Sarkisov连接。对于 森纤维空间$X=X_{0}\to S=S_{0}$,将$(X_{0},B_{0}+1\cdot G_{0}+0\cdot H_{0}) $看作$(W,B_{W}+1\cdot G_{W}+0\cdot H_{W}) $ 的弱算术典范模型,并把$(X‘,B’+0\cdot G'+1\cdot H') $看作$(W,B_{W}+0\cdot G_{W}+1\cdot H_{W}) $ 的弱算术典范模型。归纳地构造  $f_{i}:X_{i}\to S_{i}$,使得$(X_{i},B_{i}+g_{i}\cdot G_{i}+h_{i}\cdot H_{i}) $是$(W,B_{W}+g_{i}\cdot G_{W}+h_{i}\cdot H_{W}) $的弱算术典范模型。在这个过程中,$g_{i}$ 逐渐减少为$g_{N}=0$,而$h_{i}$逐渐增加为$h_{N}=1$,并且$X_{N}\to S_{N}$与$X'\to S'$同构。

假设已经构造定理\ref{main2}中的映射$\sigma_{i}:W\dashrightarrow X_{i}$,即
\begin{itemize}
  \item $f_{i}:(X_{i},B_{i})\to S_{i}$ 是森纤维空间,且 $\sigma_{i*} B_{W}=B_{i}$;
  \item $\sigma_{i}:W\dashrightarrow  X_{i}$ 是 $(K_{W}+B_{W}+g_{i}G_{W}+h_{i}H_{W})$-非正性的,且
    \[(K_{X_{i}}+B_{i}+g_{i}G_{i}+h_{i}H)=\sigma_{i*}(K_{W}+B_{W}+g_{i}G_{W}+h_{i}H_{W}) \]
    是数值有效的,并且相对于 $S_{i}$ 数值平凡;
  \item $(W,B_{W}+g_{i}G_{W}+h_{i}H_{W})\geqslant (X_{i},B_{i}+g_{i}G_{i}+h_{i}H_{i})$;
  % \item $0\leqslant g_{i},h_{i}\leqslant 1$ 是有理数。
  \item $0\leqslant g_{i},h_{i}\leqslant 1$。
\end{itemize}
那么构造下一个满足定理\ref{main2} 的 Sarkisov 连接 $X_{i}\dashrightarrow X_{i+1}$。类比下降法的Sarkisov次数,引入下列记号:
\begin{definition}\label{doubledegree}
  令 $C_{i}$ 是被 $f_{i}$ 压缩的(也称为 $f$-vertical  )  $X_{i}$ 上的曲线,那么
  \begin{itemize}
    \item $r_{i}:=\frac{H_{i}.C_{i}}{G_{i}.C_{i}}$;
    \item 令 $\Gamma$ 是满足下列条件的 $t\in [0,\frac{g_{i}}{r_{i}}] $的集合
      \begin{itemize}
        \item\label{singularcondition} $\left(W,B_{W}+g_{i}G_{W}+h_{i}H_{W}+t(H_{W}-r_{i}G_{W})\right)\geqslant \left(X_{i},B_{i}+g_{i}G_{i}+h_{i}H_{i}+t\left(H_{i}-r_{i}G_{i}\right)\right)$;
        \item $K_{X_{i}}+B_{i}+g_{i}G_i+h_{i}H+t(H_{i}-r_{i}G_{i})$  数值有效.
      \end{itemize}
          令 $s_{i}=\max\, \Gamma $;
    \item 记 $D_{W,i}=B_{W}+g_{i}G_{W}+h_{i}H_{W}$ 和 $D_{i}=B_{i}+g_{i}G_{i}+h_{i}H_{i}$,并记
      \[D_{W,i}(t)=B_{W}+g_{i}G_{W}+h_{i}H_{W}+t(H_{W}-r_{i}G_{W})\]
      和
      \[D_{i}(t)=B_{i}+g_{i}G_{i}+h_{i}H_{i}+t (H_{i}-r_{i}G_{i})\]
      令 $g_{i+1}=g_{i}-r_{i}s_{i}$ 和 $h_{i+1}=h_{i}+s_{i}$。注意到 $D_{W,i+1}=D_{W,i}(s_{i})$.
  \end{itemize}
\end{definition}
那么有
\begin{enumerate}
  \item $r_{i}>0$;
  \item  $\Gamma=\{0\} $ 或 $\Gamma$ 是闭区间
  \item $g_{i+1}=g_{i} \Leftrightarrow h_{i+1}=h_{i} \Leftrightarrow s_{i}=0$;
\end{enumerate}
(细节见 \cite[Lemma 4.4]{liuSarkisovProgramGeneralized2021} )

\textbf{终结判定:}
如果 $s_{i}=\frac{g_{i}}{r_{i}}$,此时 $g_{i+1}=0$。 令 $N=i+1$ ,并令$f_{N}:X_{N}=X_{i} \to S_{N}=S_{i}$,那么 $h_{N}=1$ 且 $X_{N}\to S_{N}$ 与 $f':X'\to S'$同构 (见命题 \ref{nfi2}),这样双标量Sarkisov纲领停止。此时事实上$X_{i}\cong X_{N},S_{i}\cong S_{N}$,但是对二者的边界不同,而 $X_{N}$的边界$B_{N}$与$(X',B')$的边界$B'$相同。

\begin{proposition}[构造Sarkisov 连接]\label{2Constructlink}
 如果$s_{i}<\frac{g_{i}}{r_{i}}$,那么可以构造Sarkisov 连接 $X_{i}\dashrightarrow X_{i+1}$ 。
\end{proposition}
\begin{proof}
  分两种情况构造:
  \begin{enumerate}
  \item\label{2a} 若 $s_{i}$ 不是集合 $\Gamma$ 的第一个条件
    \[\left(W,B_{W}+g_{i}G_{W}+h_{i}H_{W}+t(H_{W}-r_{i}G_{W})\right)\geqslant \left(X_{i},B_{i}+g_{i}G_{i}+h_{i}H_{i}+t\left(H_{i}-r_{i}G_{i}\right)\right)\]
    的阈值,也就是说存在 $0<\epsilon\ll 1$,使得 $W$ 上任意除子$E$,都有
  \[
    a(E;X_{i},D_{i}(s_{i}+\epsilon))\geqslant a(E;W,D_{W,i}(s_{i}+\epsilon))
  \]
  并且 $K_{X_{i}}+D_{i}(s_{i}+\epsilon)$ 不是数值有效。那么对某个 $0< \delta \ll \epsilon $,存在一个 $2$ 维的 $(K_{X_{i}}+D_{i}(s_{i}+\epsilon)-\delta G_{i})$-负性 极端面 $F$,由$R=\mathbb{R}_{\geqslant 0}[C_{i}]$ 和令一个极端射线$P$张成 (spanned by)。于是存在关于 $F$   的压缩态射$X_{i}\to T_{i}$穿过 $f_{i}$。 
  \[
    \xymatrix{
      X_{i}\ar[d]\ar[rdd]& \\
      S_{i}\ar[rd]&\\
         &T_{i}
    }
  \]
  接着运行相对于 $T_{i}$ 的关于某丰沛除子标量的$(K_{X_{i}}+D_{i}(s_{i}+\epsilon))$-MMP,在有限多个翻转后,会接着一个关于$(K_{X_{i}}+D_{i}(s_{i}+\epsilon))$的森纤维空间,或一个除子压缩,或一个极小模型。
  \begin{enumerate}
    \item  若记$X_{i}\dashrightarrow X_{i+1}$是翻转的复合,并且接着森纤维空间压缩 $X_{i+1}\to S_{i+1}$,那么这是第四型的Sarkisov连接;
      \[
        \xymatrix{
          X_{i}\ar@{.>}[rr]\ar[d]& &X_{i+1}\ar[d] \\
          S_{i}\ar[rd]& & S_{i+1}\ar[ld]\\
             &T_{i}&
        }
      \]
    \item 若记$X_{i}\dashrightarrow Z_{i}$是翻转的复合,并接着一个除子压缩态射 $Z_{i}\to X_{i+1}$,那么 令 $S_{i+1}=T_{i}$,则 $X_{i+1}\to S_{i+1}$是一个森纤维空间。这是第三型的Sarkisov连接。
      \[
        \xymatrix{
          X_{i}\ar@{.>}[r]\ar[d]& Z_{i}\ar[rd] \\
          S_{i}\ar[rd]& & X_{i+1}\ar[d]\\
               &T_{i}\ar[r]^{\sim}& S_{i+1}
        }
      \]
    \item 若记 $X_{i}\dashrightarrow X_{i+1}$是翻转的复合,且 $X_{i+1}\to T_{i}$是关于 $\left(X_{i},D_{i}\left(s_{i}+\epsilon\right)\right)$ 相对于 $T_{i}$的极小模型。 令  $C'$ 是  $C_{i}$ 在 $X_{i+1}$上的双有理变换,则有 $(K_{X_{i+1}}+D_{i+1}(\epsilon)).C'=0$ 和 $(K_{X_{i+1}}+B_{i+1}).C'<0$,因此有压缩态射$X_{i+1} \to S_{i+1}$ 相对于 $T_i$。这个态射是森纤维空间,并且这是第四型Sarkisov连接。
      \[
        \xymatrix{
          X_{i}\ar@{.>}[rr]\ar[d]& &X_{i+1}\ar[d] \\
          S_{i}\ar[rd]& & S_{i+1}\ar[ld]\\
             &T_{i}&
        }
      \]
  \end{enumerate}
  注意到$ (K_{X_{i}}+D_{i}(s_{i})) $及其严格双有理变换都是相对于 $S_{i}$上数值平凡的,由锥定理\ref{conethm},此时的$X_{i}\dashrightarrow X_{i+1}$是一系列$ (K_{X_{i}}+D_{i}(s_{i}))$-平凡映射的复合。
    \item 如果 $s_{i}$ 集合$\Gamma$的第一个条件\ref{singularcondition}的阈值,也就是说 存在  $0<\epsilon \ll 1$ 和 $W$ 上的  $\sigma_{i}$-例外除子 $E_{i}$ 使得
  \[ a(E_{i};X_{i},D_{i}(s_{i}+\epsilon))< a(E_{i};W,D_{W,i}(s_{i}+\epsilon)) \]
  此时有
  \[ a(E_{i};X_{i},D_{i}(s_{i}))= a(E_{i};W,D_{W,i}(s_{i}))=-\operatorname{mult}_{E_{i}}(D_{W,i}(s_{i}))\leqslant 0 \]
  令 $p_{i}:Z_{i}\to X_{i}$ 是推论 \ref{extraction} 中除子 $E_{i}$的除子解压,并且记
  \[K_{Z_{i}}+D_{Z_{i}}(s_{i})=K_{Z_{i}}+B_{Z_{i}}+g_{i+1}G_{Z_{i}}+h_{i+1}H_{Z_{i}}=p_{i}^*\left(K_{X_{i}}+D_{i}\left(s_{i}\right)\right)\]
  取充分小的 $\delta$ 使得 $0<\delta \ll \epsilon \ll 1$ 且
  \[ K_{Z_{i}}+\Delta_{i}=p_{i}^*(K_{X_{i}}+D_{i}(s_{i}+\epsilon)-\delta G_{i}) \]
  是klt奇点。接下来运行相对于$S_{i}$ 的 $(K_{Z_{i}}+\Delta_{i})$-MMP。由于$Z_{i}$被$(K_{Z_{i}}+\Delta_{i})$-负性的曲线覆盖,所以$(K_{Z_{i}}+\Delta_{i})$相对于$S_{i}$不是伪有效的,并且这个MMP终结于森纤维空间。同时,对一任意$0<\delta'\leqslant\delta$,这也是$p_{i}^*(K_{X_{i}}+D_{i}(s_{i}+\epsilon)-\delta'G_{i})$-MMP。在有限多步翻转之后,要么终结于$(K_{Z_{i}}+\Delta_{i})$的森纤维空间,要么$(K_{Z_{i}}+\Delta_{i})$的除子压缩。
  \begin{enumerate}
    \item 若记 $Z_{i}\dashrightarrow X_{i+1}$是有限步翻转的复合,并终结于森纤维空间
      \[X_{i+1}\to S_{i+1}\]
      那么是第一型的Sarkisov连接。 
      \[
        \xymatrix{
        &Z_{i}\ar@{.>}[r]\ar[ld] &X_{i+1}\ar[d] \\
          X_{i}\ar[d]& &S_{i+1}\ar[lld]\\
          S_{i}   & &
        }
      \]
      在这种情况下有 $\rho(X_{i+1})=\rho(X_{i})+1$。
    \item 若记$Z_{i}\dashrightarrow Z_{i+1}'$是有限步翻转的复合,并且接着相对于 $S_{i}$的 除子压缩
      \[q_{i}:Z_{i}'\to X_{i+1}\]
     那么$X_{i+1}\to S_{i}=:S_{i+1}$是森纤维空间,这是第二型Sarkisov连接。
      \[
        \xymatrix{
        &Z_{i}\ar@{.>}[r]\ar[ld] &Z_{i}'\ar[rd] \\
          X_{i}\ar[d]& & &X_{i+1}\ar[d]\\
          S_{i}\ar[rrr]^{\sim}   & & & S_{i+1}
        }
      \]
  \end{enumerate}
  注意到$ (K_{X_{i}}+D_{i}(s_{i})) $及其严格双有理变换都是相对于 $T_{i}$上数值平凡的,由锥定理\ref{conethm},此时的$X_{i}\dashrightarrow X_{i+1}$是一系列$ (K_{X_{i}}+D_{i}(s_{i}))$-平凡映射的复合。
\end{enumerate}
\end{proof}
类似于下降法中的Sarkisov次数,定义\ref{doubledegree}中的不变量也有变化:
\begin{assertion}\label{behavior2}
  由 \cite[Lemma 13.14-17]{haconMinimalModelProgram2012} 和 \cite[Lemma 4.2]{liuSarkisovProgramGeneralized2021}可以得到:
  \begin{enumerate}
    \item $r_{i}\leqslant r_{i+1}$。在(1).1的情况,还有 $r_{i}<r_{i+1}$。
    \item 在情况 (1) 中由于 $X_{i}\dashrightarrow X_{i+1}$ 是相对于 $T_{i}$的 $(K_{X_{i}}+D_{i}(s_{i}))$-平凡映射,所以对任意的$W$ 之上的除子$E$ ,都有
      \[a(E;X_{i},D_{i}(s_{i}))= a(E;X_{i+1},D_{i+1})\]
      和不等式
      \[ a(E;X_{i+1},D_{i+1})\geqslant a(E;W,D_{W,i+1}) \]
    \item 在情况 (2) 中由于 $X_{i}\dashrightarrow X_{i+1}$ 是相对于 $S_{i}$的 $(K_{X_{i}}+D_{i}(s_{i}))$-平凡映射,所以同样对任意的$W$ 之上的除子$E$ ,都有
      \[a(E;X_{i},D_{i}(s_{i}))= a(E;X_{i+1},D_{i+1})\]
      和不等式
      \[ a(E;X_{i+1},D_{i+1})\geqslant a(E;W,D_{W,i+1}) \]
    \item\label{2adicrepancy} 在情况 (1)中,任意 $W$之上的除子$E$ 和 $0<\epsilon\ll 1$,都有
      \[a(E;X_{i},D_{i}(s_{i}+\epsilon))\leqslant a(E;X_{i+1},D_{i+1}(\epsilon))\] 
      并且因为 $X_{i} \not\cong X_{i+1}$,所以 有$W$ 之上的除子$F$使得 
      \[a(F;X_{i},D_{i}(s_{i}+\epsilon))< a(F;X_{i+1},D_{i+1}(\epsilon))\]
    \item\label{2bdiscrepancy} 在情况(2)中,任何$ W$之上的除子$E$ ,对所有$0<\epsilon\ll 1$都有
      \[a(E;X_{i},D_{i}(s_{i}+\epsilon)-\delta G_{i})\leqslant a(E;X_{i+1},D_{i+1}(\epsilon)-\delta G_{i+1})\]
      并且因为 $X_{i} \not\cong X_{i+1}$,所以 有$W$ 之上的除子$F$使得 
      \[a(F;X_{i},D_{i}(s_{i}+\epsilon)-\delta G_{i})< a(F;X_{i+1},D_{i+1}(\epsilon)-\delta G_{i+1}) \]
    \item  $h_{i}\leqslant 1$,且 $h_{i}=1$ 当且仅当 $g_{i}=0$;
  \end{enumerate}
  
\end{assertion}

\section{双标量Sarkisov纲领的终结性}
\begin{lemma}\label{termination2}
  \cite[Lemma 13.18-19]{haconMinimalModelProgram2012} (或者 \cite[Lemma 4.9]{liuSarkisovProgramGeneralized2021}) 假设按上节所述构造了 Sarkisov 连接的序列:
  \[
    \xymatrix{
    X=X_0\ar@{.>}[r]\ar[d]_{f_0}&X_{1}\ar@{.>}[r]\ar[d]_{f_1}& X_{2}\ar[d]_{f_2}\ar@{.>}[r] &\cdots\ar@{.>}[r]&X_{i}\ar[d]_{f_i}\ar@{.>}[r] &\cdots\\
    S=S_{0}&S_{1} &S_{2}&&S_{i}
    }
    ,\]
  那么
  \begin{enumerate}
    \item 在同构意义下,只有有限多种森纤维空间 $f_{i}:X_{i}\to S_{i}$;
    \item 对$(G_{W},H_{W})$双标量的Sarkisov纲领终结,即存在整数 $N>0$ 使得 $g_{N}=0$。
  \end{enumerate}
\end{lemma}

\begin{proof}
  \begin{enumerate}
    \item 第一条本质上来自弱算术典范模型的有限性 (定理 \ref{finitewlcm})。  如下构造 $\operatorname{WDiv}_{\mathbb{R}}(W)$的子空间 $V$:
      \begin{enumerate}
        \item 假设对某个$k$ 有  $h_{k}>0$。 由于 $H_{W}$ 是 数值有效 且是大除子,那么存在丰沛 $\mathbb{Q}$-除子 $A_{W}$ 和有效 $\mathbb{Q}$-除子 $C_{W}$  使得 $H_{W}\sim_{\mathbb{Q}}A_{W}+C_{W}$。 此时取 $V$ 为  $B_{W},G_{W},H_{W},C_{W}$的组成成分的素除子张成的仿射空间。此时对于 $i>k$,定义$\Delta_{i}$为
              \[
                  B_{W}+g_{i}G_{W}+h_{i}H_{W}\sim_{\mathbb{Q}} h_{k}A_{W}+B_{W}+g_{i}G_{W}+(h_{i}-h_{k})H_{W}+h_{k}C_{W}=:\Delta_{i} \in \mathcal{L}_{h_{k}A_{W}}(V)
              \]
        \item 假设对所有$k$ 都有$h_{k}=0$,那么  $h_{i}\equiv 0$ 且 $g_{i}\equiv 1$。由于  $G_{W}$ 是 数值有效 且是大除子,那么存在丰沛 $\mathbb{Q}$-除子 $A_{W}$和有效 $\mathbb{Q}$-除子 $C_{W}$ 使得 $G_{W}\sim_{\mathbb{Q}}A_{W}+C_{W}$。此时取 $V$ 为  $B_{W},C_{W}$的组成成分的素除子张成的仿射空间。此时定义 $\Delta_{i}$为
              \[
                  B_{W}+G_{W}\sim_{\mathbb{Q}} A_{W}+B_{W}+C_{W}=:\Delta_{i} \in \mathcal{L}_{A_{W}}(V)
              \]
      \end{enumerate}
          那么每一个 $X_{i}$ 都是$(W,\Delta_{i})$的弱算术典范模型。由弱算术典范模型的有限性,在同构意义下,存在有限多个 $\sigma_{i}: W\dashrightarrow X_{i}$。

          这样只证明了 $X_{i}$ 的有限性,此时需要森纤维空间 $X_{i}\to S_{i}$的有限性。接下来证明 在构造的Sarkisov连接的序列中,出现的$X_{i}$的森纤维空间在同构意义下只有有限多个。
          不妨设存在无穷子列 $\{i_{k}\}_{k=0}^{\infty} $使得 $X_{i_{k}}\cong X_{i_{0}}$。为简化记号,不妨设$X_{i_{0}}=X_{0}=X$ (注意,做此替换后$X_{0}$可能并不是Sarkisov纲领的初始起点,但为符号简便,仍将其记为$X$)。记$f_{i_{k}}$是关于极端射线 $R_{k} \subset \overline{\operatorname{NE}}(X_{i_{k}})= \overline{\operatorname{NE}}(X) $ 的森纤维空间压缩态射。 那么有
          $(K_{X}+B).R_{i_{k}}<0  $ 且 $(K_{X}+B+g_{i_{k}}G+h_{i_{k}}H).R_{k}=0$, 并且 $H$ 和 $G$ 相对于 $S_{i_{k}}$ 是丰沛的。分三种情况证明:
      \begin{enumerate}

        \item 如果对某$k$ 有  $h_{i_{k}}=1$,那么Sarkisov纲领终结,原序列是有限序列,断言显然成立。
        \item 假设$h_{i_{k}}=0$ 对所有 $i_{k}$ 成立,那么 $g_{i_{k}}=1$。 因为 $G_{i_{k}}$ 是大除子,所以存在丰沛$\mathbb{Q}$-除子 $A$ 和有效$\mathbb{Q}$-除子 $E$ 使得 $G=A+E$。  取充分小的 $\epsilon$ 使得$(X,D)$具有klt奇点,其中$D=B+(1-\epsilon)G+\frac{\epsilon}{2} E$,那么 $(K_{X}+D).R_{k}<0$且 $(K_{X}+D+\frac{\epsilon}{2} A).R_{k}<0$ 对所有$k$ 成立。由锥定理\ref{conethm},有
              \[
                \overline{\operatorname{NE}}(X)=\overline{\operatorname{NE}}(X)_{K_{X}+D+\frac{\epsilon}{2}A_{k}\geqslant 0} +\sum_{\alpha \in\Lambda\text{有限集}}R_{\alpha}.
              \]
              由于 森纤维空间$X=X_{i_{k}}\to S_{i_{k}}$ 是关于极端射线 $R_{\alpha}$的压缩态射,这样的射线有限,所以弱算术典范模型 $X=X_{i_{0}}$ 在构造的Sarkisov连接的序列中出现的森纤维空间在同构意义下有限;
            \item 假设对某个 $k>0$ 有  $h_{i_{k}}>0$,那么通过截短子列,不妨设$0<h_{i_{0}}=h_{0}<1$ (注意,做此替换后$X_{0}$并不是Sarkisov纲领的初始起点,并且$g_{0}\neq 1$,但为符号简便,仍将其记为$X$  )。 因为 $H$ 是大除子,所以存在丰沛$\mathbb{Q}$-除子 $A$ 和有效$\mathbb{Q}$-除子 $E$ 使得 $H=A+E$。  取充分小的 $\epsilon$ 使得$(X,D)$具有klt奇点,其中$D=B+(1-\epsilon)H+\epsilon E$。那么 $(K_{X}+D).R_{k}<0$ 且 $(K_{X}+D+\epsilon A).R_{k}<0$ 对所有 $k$成立。同样由锥定理\ref{conethm},有
              \[
                \overline{\operatorname{NE}}(X)=\overline{\operatorname{NE}}(X)_{K_{X}+D+\epsilon A}\geqslant 0} +\sum_{\alpha \in\Lambda\text{有限集}}R_{\alpha}.
              \]
              同上,由于 森纤维空间$X=X_{i}_{k}\to S_{i_{k}}$ 是关于极端射线 $R_{\alpha}$的压缩态射,这样的射线有限,所以弱算术典范模型 $X=X_{i_{0}}$ 在构造的Sarkisov连接的序列中出现的森纤维空间在同构意义下有限;
      \end{enumerate}
    \item 若上述构造的 Sarkisov连接的序列是无限的,那么由(1) ,不妨设存在无穷子列 $\{i_{k}\}_{k=0}^{\infty} $使得 森纤维空间$X_{i_{k}}\to S_{i_{k}}$同构于$X_{i_{0}}\to S_{i_{0}}$。为简化记号,不妨设$X_{i_{0}}=X_{0}=X$ (同样的,做此替换后$X_{0}$可能并不是Sarkisov纲领的初始起点,并且可能有$g_{0}\neq 1,h_{0}\neq 0$,但为符号简便,仍将其记为$X$)。此时曲线$C_{i_{k}} \subset X$是相同的曲线,由$g_{i_{k}}$和$h_{i_{k}}$的构造可以得到 $g_{0}=g_{i_{k}}$ 和 $h_{0}=h_{i_{k}}$。由于序列 $h_{i}$ 和 $g_{i}$是单调的,所以 $h_{i}=h_{0}$ 和  $g_{i}=g_{0}$ 是常值,并且$r_{i}=r_{0},s_{i}=0$都是常值。

      若序列中存在命题\ref{2Constructlink}中(1)构造的第三或第四型Sarkisov连接,不妨设为 $X\dashrightarrow X_{1}$,那么由断言\ref{behavior2}的(4),下一个Sarkisov连接也是 按(1)的构造,继而之后所有连接都是如此。再考虑Picard数,由于$X\cong X_{i_{1}}$, 显然之后都是第四型Sarkisov连接。由断言\ref{behavior2}的(4),对任意 $ W$上的除子 $E$, 存在不等式
      \[ a(E;X,D(s_{0}+\epsilon))\leqslant a(E;X_{1},D_{1}(\epsilon))\] 
      但是同样由断言\ref{behavior2}的(4),还存在$W$ 之上的除子 $F$,使得 
      \[a(F;X,D(s_{0}+\epsilon))< a(F;X_{1},D_{1}(\epsilon))\]
      于是对$ X = X_{i_{0}}\cong X_{i_{1}} $ 有
      \[a(F;X,D(\epsilon))< a(F;X_{i_{1}},D_{i_{1}}(\epsilon))\]
      矛盾。

      假设Sarkisov连接的序列全是由命题\ref{2Constructlink}中(2)构造的第一第二型Sarkisov连接。同样由Picard数可以得到,所有Sarkisov连接都是第二型。类似的,由断言\ref{behavior2}的(5)可以推出矛盾。

      综上,命题\ref{2Constructlink}构造的Sarkisov序列是有限的。
  \end{enumerate}
\end{proof}
最后还要证明双标量Sarkisov纲领终结的森纤维空间$X_{N}\to S_{N}$与 $(X',B')\to S'$同构。这一部分与下降法Sarkisov纲领的判别法定理\ref{nfi}类似。
\begin{proposition}\label{nfi2}
  \begin{enumerate}
    \item $h_{N}=1$。
    \item $X_{N}\to S_{N}$ 同构于 $X'\to S'$。
  \end{enumerate}
\end{proposition}
\begin{proof}
事实上此处 在$X'$上的$\mathbb{Q}$-除子 $h'H'=1\cdot H'$代替了定理\ref{nfi}中$X'$ 上的$\mathbb{Q}$-除子$ \frac{1}{\mu'}H'$,而$h_{N}H_{N}$对应定理\ref{nfi}中的$ \frac{1}{\mu}H$。用同样的方法可以证明$ h_{N}=h'=1 $。

(2)也可以用同样的方法证明:
 取 $X_{N}$ 上  极丰沛除子 $ L $,且$L'  $ 是在$ X' $上的严格双有理变换。那么  $ L' $ 是 $ f' $-丰沛的,所以存在$ 0<d\ll1 $使得:
          \begin{itemize}
            \item $ K_{X_{N}}+B_{N}+H_{N}+dL $ 是丰沛除子;
            \item $ K_{X'}+B'+H'+dL' $ 是丰沛除子;
            \item $(W,B_{W}+H_{W}+dD_{W})$具有klt奇点 (由双标量Sarkisov纲领的初始设定断言\ref{doublesetting}可以得到)。
          \end{itemize}
          因此 $X_{N}$ 和 $X'$ 都是 $(W,B_{W}+H_{W}+dD_{W})$的算术典范模型,由算术典范模型的唯一性, $X_{N}\cong X'$。更进一步, $f_{N}$ 和  $f'$压缩相同的曲线数值等价类,所以两个森纤维空间同构。
\end{proof}
